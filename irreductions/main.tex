\documentclass[a4paper, 12pt]{book}
%\documentclass[a5paper, 12pt]{report}

%% Разметка документа
\usepackage{geometry}
\geometry{right=1.5cm, left=2.5cm, top=2cm, bottom=2cm}
 
%% Язык и кодировка
\usepackage[english,russian]{babel}
\usepackage[utf8x]{inputenc}
\usepackage[T2A]{fontenc}
\frenchspacing
\raggedbottom
\renewcommand{\arraystretch}{1.6}
\columnsep2em

%% Представление
\usepackage{graphicx}
\usepackage[colorinlistoftodos]{todonotes}
\usepackage[colorlinks=true, allcolors=blue]{hyperref}
\usepackage{hyperref}
%\usepackage{tempora} % гарнитура Tempora
%\usepackage{dejavu} % гарнитура DejaVu
%\usepackage{droid} % шрифты Droid
%\usepackage{paratype} % шрифты Paratype

\usepackage{ucs}

%% Содержание
\usepackage{tocloft} % точки в строках содержания
\renewcommand{\cftchapleader}{\cftdotfill{\cftdotsep}} % для глав

%% Гиперссылки
\hypersetup{
    colorlinks=true,
    linkcolor=blue,
    filecolor=magenta,      
    urlcolor=cyan,
    bookmarks=true,
    pdftitle={Ирредукции},
	pdfsubject={},
	pdfauthor={Брюно Латур},
	pdfkeywords={}
}
            
\usepackage{amssymb}
\renewcommand{\labelitemi}{$\square$}

\clubpenalty=10000
\widowpenalty=10000
\sloppy

\title{Ирредукции}
\author{Брюно Латур}
\date{}

\begin{document}
\maketitle

\tableofcontents

\chapter{От слабости к~могуществу (потенции)}\hypertarget{chap1}{}

\paragraph{1.1.1}\hypertarget{par:1.1.1}{} Ничто само по себе не~является редуцируемым или нередуцируемым к~чему-либо еще. 
	\begin{itemize}
	\item {Я назову это <<принципом нередуцируемости>>, но это государь, который не~правит, так как это было бы~противоречием самому себе (\hyperlink{par:2.6.1}{2.6.1})}.
	\end{itemize}

\paragraph{1.1.2}\hypertarget{par:1.1.2}{} Существуют только испытания сил, слабостей. Или проще~--- существуют только испытания. Вот пункт моего отправления: глагол, <<испытывать>>.

\paragraph{1.1.3}\hypertarget{par:1.1.3}{} Поскольку ничто само по себе не~является редуцируемым или нередуцируемым к~чему-либо еще, постольку существуют только испытания (сил, слабостей). То, что не~является ни~редуцируемым и~ни нередуцируемым должно быть проверено, сосчитано и~измерено. Другого пути нет.

\paragraph{1.1.4}\hypertarget{par:1.1.4}{} Все может быть сделано мерой всего остального.

\paragraph{1.1.5}\hypertarget{par:1.1.5}{} Реально то, что сопротивляется испытаниям.
	\begin{itemize}
	\item Глагол <<сопротивляться>> не~является привилегированным словом. Он используется для обозначения всей совокупности глаголов и~прилагательных, орудий и~приборов, которые вместе определяют способы быть реальным. Мы равным образом можем сказать <<свертываться>>, <<сгибаться>>, <<загораживать>>, <<обострять>>, <<скользить>>. Существует множество альтернатив.
	\end{itemize}

\paragraph{1.1.5.1}\hypertarget{par:1.1.5.1}{} Реальное не~вещь среди других вещей, но скорее градиент (уровень, шкала) сопротивления.

\paragraph{1.1.5.2}\hypertarget{par:1.1.5.2}{} Не существует различия между <<реальным>> и~<<нереальным>>, <<реальным>> и~<<возможным>>, <<реальным>> и~<<воображаемым>>. Скорее все это различия между теми, кто сопротивляется длительное время и~теми, кто этого не~делает, между теми, кто отважно сопротивляется и~теми, кто нет, между теми, кто знает, как вступить в~союз или изолировать себя, и~теми, кто этого не~знает.

\paragraph{1.1.5.3}\hypertarget{par:1.1.5.3}{} Ни одна сила не~может, как это часто говорится, <<познавать реальность>> кроме как через различия, которые она создает в~сопротивлении другим силам.

\paragraph{1.1.5.4}\hypertarget{par:1.1.5.4}{} Ничто не~может быть познано~--- только сделано реальным (realized).

\paragraph{1.1.6}\hypertarget{par:1.1.6}{} Форма это линия фронта (передовая) испытания сил, которая де-формирует, транс-формирует, ин-формирует и~пер-формирует его. Конечно, стоит форме стабилизироваться, она уже более не~выглядит как испытание сил.

\paragraph{1.1.7}\hypertarget{par:1.1.7}{} Что такое сила? Кто это? На что она способна? Это субъект, текст, объект, энергия или вещь? Как много сил существует? Кто силен, а~кто слаб? Это битва? Это игра?
Это рынок? Все эти вопросы определяются и~деформируются в~последующих
испытаниях.
	\begin{itemize}
	\item Вместо <<силы>> можно говорить о~<<слабостях>>, <<энтелехиях>>, <<монадах>>, или проще~--- <<актантах>>.
	\end{itemize}

\paragraph{1.1.8}\hypertarget{par:1.1.8}{} Ни один актант не~является настолько слабым, что не~может завербовать другого. Тогда двое объединяются вместе и~становятся одним для третьего актанта, которого поэтому им легче привести в~движение. Образовался небольшой вихрь, и~он растет, пополняясь многими другими.
\begin{itemize}
	\item Является ли актант сущностью (essense) или отношением? Мы не~можем сказать без испытания (\hyperlink{par:1.1.5.2}{1.1.5.2}). Для того, чтобы избежать уничтожения, сущности могут связать себя отношениями со многими союзниками, а~отношения [связать себя] со многими сущностями.
	\end{itemize}

\paragraph{1.1.9}\hypertarget{par:1.1.9}{} Актант может обрести силу только путем ассоциации с другими. Таким образом он говорит от их имени. Почему другие не~говорят сами за себя? Потому что они немы, потому что их заставили замолчать, потому что, говоря в~одно и~тоже время, они стали неслышны. Поэтому некто интерпретирует их и~говорит вместо них. Но кто? Кто говорит? Они или он? {\itshape Traditore~--- traduttore}. Один равняется нескольким. Это невозможно определить. Если лояльность актанта ставится под сомнение, он может продемонстрировать, что он только лишь повторяет то, что хотели сказать другие. Он предлагает экзегетику расстановки сил, которая не~может быть оспорена даже временно без другого альянса.

\paragraph{1.1.10}\hypertarget{par:1.1.10}{} Действуй так, как тебе этого хочется до тех пор, пока нельзя будет легко повернуть вспять. В результате работы актантов, определенные вещи не~возвращаются в~их исходное состояние. Форма сложена, как сгиб. Это можно назвать ловушкой, храповником, необратимостью, демоном Максвелла, реификацией. Конкретное слово не~имеет значения до тех пор, пока оно обозначает асимметрию. Тогда вы
уже не~можете действовать так, как Вы этого хотите. Существуют победители и~проигравшие, существуют приказы, и~некоторые из них обладают большей силой, чем другие.

\paragraph{1.1.11}\hypertarget{par:1.1.11}{} Все пока еще поставлено на~карту. Однако, так как много игроков пытаются сделать игру необратимой и~делают все, что они могут для того, чтобы гарантировать, что все не~является одинаково возможным, игра окончена.
	\begin{itemize}
	\item Мое почтение Мастерам Го\footnote{{\itshape Yasunari Kawabata}. The Master of Go (New York: Alfred Knopf, 1972).}.
	\end{itemize}

\paragraph{1.1.12}\hypertarget{par:1.1.12}{} Для того, что создать асимметрию актанту необходимо опереться на~силу чуть более прочную, чем он сам. Даже если это различие невелико, его достаточно для того, чтобы создать градиент сопротивления, который сделает их более реальными
для другого актанта~(\hyperlink{par:1.1.5}{1.1.5}).

\paragraph{1.1.13}\hypertarget{par:1.1.13}{} Мы не~можем сказать, что актант следует правилам, законам или структурам, но мы также не~можем сказать, что он действует без всего этого. Извлекая уроки из того, что делают другие актанты, он постепенно разрабатывает правила, законы и~структуры. Затем он стремится заставить других играть по этим правилам, которые
как он утверждает, он выучил, соблюдал или принял. Если он побеждает, то тогда он верифицирует эти правила и~таким образом применяет их. 
	\begin{itemize}
	\item Является ли какой-либо порядок конвенцией, социальной конструкцией, законом природы или творением человеческого духа? Мы не~можем сказать. Но в~любви, как и~на~войне, все средства хороши в~попытке приписать правила чему-то более прочному, чем породившему их моменту.

	\end{itemize}

\paragraph{1.1.14}\hypertarget{par:1.1.14}{} Ничто само по себе не~является упорядоченным или не~упорядоченным, единственным или множественным, гомогенным или гетерогенным, текучим или инертным, человеческим или нечеловеческим, полезным или бесполезным. Никогда само по себе, но всегда посредством других.
	\begin{itemize}
	\item Спиноза сказал давным-давно: до тех пор, пока {\itshape формы} имеют значение, давайте не~будем антропо{\itshape{морфичными}}. Каждая слабость распределяет полный набор ролей. В зависимости от того, чего она ожидает от других, она отделяет стабильное и~упорядоченное от бесформенного и~движущегося. Но поскольку все другие также распределяют роли, получается красивая путаница. Тем не~менее, понятно почему энтелехии могут принять за бесформенную материю тех, кого они сломали, разорвали на~части или соблазнили.

	\end{itemize}

\paragraph{1.1.14.1} Порядок возникает не~из беспорядка, а~из приказов.
	\begin{itemize}
	\item Мы всегда совершаем одну и~ту же~ошибку. Мы проводим различия между варварским и~цивилизованным, сконструированным и~разобранным (dissolved), упорядоченным и~беспорядочным. Мы всегда оплакиваем упадок и~разложение морали. Скатертью дорога! Аттила говорит на~греческом и~латыни, панки одеваются с такой же~тщательностью, как и~Коко Шанель; у бактерий чумы такие же~искусные стратегии, как и~у IBM; представители племени Азандэ фальсифицируют свои верования с попперовским смаком. Не важно как далеко мы заберемся, там всегда уже будут формы; в~каждой рыбе пруды полные рыбы. Некоторые верят в~то, что они лекало, в~то время как другие сырой материал, но это форма элитизма. Для того чтобы завербовать силу мы должны сговориться с ней. Она не~может быть выкована как листовое железо или отлита как гипсовый слепок (по образцу).

	\end{itemize}

\paragraph{1.1.15}\hypertarget{par:1.1.15}{} Утверждения <<все детерминировано>> и~<<все контингентно>> означают одно и~то же~--- то есть ничего. Слова <<детерминированный>> и~<<контингентный>> приобретают
значение, только когда используются в~разгар описания градиентов сопротивления~--- то есть, реальности.
	\begin{itemize}
	\item Длина носа Клеопатры не~является ни~значительной, ни~незначительной. Обстоятельства определяют, на~время, относительную важность того, что их составляет. Роли случая и~необходимости не~могут быть определены заранее.

	\end{itemize}

\paragraph{1.1.16}\hypertarget{par:1.1.16}{} Что одинаково и~что различно? Что с кем? Что противопоставлено или соединено или близко? Что продолжается, останавливается, отвергается, торопится или привязывается? Это общие вопросы, да, общие для всех испытаний прислуживаемся ли мы, пробуем на~вкус, распутываем, плетем, присоединяемся, стираем или же~обращаемся.

\paragraph{1.2.1}\hypertarget{par:1.2.1}{} Ничто само по себе не~является таким же~или отличным от чего-либо еще. То есть, не~существует эквивалентов, только переводы.

Другими словами все происходит только один раз и~в~одном месте.

Если между актантами существуют тождества, то это потому, что они были сконструированы за высокую плату. Если эквивалентности существуют, то это потому, что они были с большим трудом выстроены из разрозненных элементов, и~поддерживаются с помощью силы. Если существуют обмены, то они всегда неравные и~стоят целого состояния для того, чтобы устраивать равно, как и~поддерживать их.
	\begin{itemize}
	\item Я называю это <<принципом относительности>>. Также как для одного наблюдателя невозможно общаться с другим быстрее скорости света, лучшее, что может быть сделано с актантами~--- это перевод одного актанта в~другой. Между несоизмеримыми и~нередуцируемыми силами нет ничего: ни~эфира, ни~непосредственности. Верно то, что данный принцип относительности имеет своей целью восстановить неэквивалентность актантов, в~то время как другой принцип был создан для того, чтобы возродить эквивалентность всех наблюдателей. В обоих из них, однако, мы должны привыкнуть к~тому, чтобы дышать в~отсутствии эфира. Вещи, о~которых я говорю, редки, разбросаны и~преимущественно пусты. Скопления, проявления насыщенности и~полноты редки и~разбросаны, как большие города на~карте страны.

	\end{itemize}

%\subparagraph {Интерлюдия 1: Объяснить в~псевдоавтобиографическом стиле цели автора.}

\paragraph{1.2.2}\hypertarget{par:1.2.2}{} Энтелехии ни~о чем не~договариваются и~могут договориться обо всем, поскольку ничто по своей природе и~без связи с чем-либо еще не~является ни~соизмеримым, ни~несоизмеримым. Каким бы~ни было согласие, всегда будет нечто, что может дать пищу для разногласий. Какой бы~ни была дистанция, всегда будет нечто, на~чем может быть основано согласие. Иными словами, все может стать предметом переговоров.
	\begin{itemize}
	\item <<Переговоры>> не~плохое слово пока подразумевается, что все может стать предметом переговоров, а~не~только форма стола и~имена делегатов. Также необходимо принять решения о~том, что является предметом переговоров, когда можно будет сказать, что они начались или завершились, на~каком языке они будут проводиться, и~каким образом будет определяться то, были ли поняты или нет. Была ли это битва, церемония, дискуссия или игра? Это тоже предмет спора, спора, который продолжается до тех пор, пока все энтелехии не~будут определены и~не~определят других сами. Для изображения этих переговоров мне~нужно <<Поле золотой парчи>>.
	\end{itemize}

\paragraph{1.2.3}\hypertarget{par:1.2.3}{} Сколько существует актантов? Это невозможно определить до тех пор, пока они не~будут измерены друг относительно друга.
	\begin{itemize}
	\item Я еще не~сказал как много нас: 50 миллионов французов, одна экосистема, 20 миллиардов нейронов, три ли четыре типа характеров, один <<me, I, me, I>>. Мы не~можем сосчитать число сил, решить, что существует единственная субстанция, два социальных класса, три хариты, четыре стихии, семь смертных грехов или двенадцать апостолов. Мы не~можем подсчитать итог. В этой своеобразной арифметике никто никогда не~вычитает. Мы прибавляем столько промежуточных сумм, сколько существует счетоводов.
	\end{itemize}

\paragraph{1.2.3.1}\hypertarget{par:1.2.3.1}{} Не существует ни~целых, ни~частей. Также как не~существует ни~гармонии, ни~структуры, ни~интеграции, ни~системы (\hyperlink{par:1.1.14}{1.1.14}). То, каким образом нечто сохраняется
(hold together) определяется на~поле битвы, поскольку никто не~согласен относительно того, кто будет командовать, а~кто подчиняться, кто будет частью, а~кто целым. 
	\begin{itemize}
	\item Предустановленной гармонии, несмотря на~Лейбница, не~существует, гармония устанавливается после локально и~посредством экспериментальной починки.
	\end{itemize}

\paragraph{1.2.4}\hypertarget{par:1.2.4}{} Мы не~знаем, где может быть найден какой-либо актант. Определение его положения это изначальная борьба, в~ходе которой многое может быть потеряно. Мы можем сказать, что некто/нечто определяет местоположение, а~другим определяют место.

\paragraph{1.2.4.1}\hypertarget{par:1.2.4.1}{} Хотя места удалены, нередуцируемы и~несовместимы, их, тем не~менее, связывают
вместе, объединяют, складывают, выравнивают и~подвергают всевозможным
операциям. Если бы~не~эти операции, ни~одно место не~вело бы~к другому.

\paragraph{1.2.5}\hypertarget{par:1.2.5}{} Силы, которые вступают в~союз в~ходе испытания, считаются надежными. Каждая энтелехия создает времена для других, вступая с ними в~союз или предавая их. <<Время>> возникает в~конце этой игры, игры в~которой большинство проиграло то, что поставило.
Мгновение до или мгновение после? Является ли нечто/некто одержимым, пророческим, устаревшим, упадочным, современным, временным или вечным? Это невозможно определить заранее. Об этом необходимо договориться.

\paragraph{1.2.5.1}\hypertarget{par:1.2.5.1}{} Время это отдаленные последствия стремлений каждого из акторов создать необратимые события в~собственных интересах (\hyperlink{par:1.1.10}{1.1.10}). Таким образом проходит время.

\paragraph{1.2.5.2}\hypertarget{par:1.2.5.2}{} Время не~проходит. Времена являются ставкой в~борьбе между силами. Конечно, одна сила может овладеть другими, но это может быть только локально и~временно, потому что постоянство стоит слишком дорого и~требует слишком много союзников.

\paragraph{1.2.5.3}\hypertarget{par:1.2.5.3}{} Во Франции часто говорится, что <<существуют>> революции, но они лишь акторы, которые выиграли время и~историю у других акторов и~таким образом обошли других, и~сделали их устаревшими. Конечно, побежденные иногда добиваются отмщения и, таким образом, опрокидывают ход времен еще раз.
	\begin{itemize}
	\item Кто тогда более современен~--- Шах; Хомейни, Мусульманин из другой эпохи; или Бани-Садр, Президент, который нашел убежище в~Париже? Никто не~знает, и~именно поэтому они так много борются за то, чтобы прийти в~свое время (to make their time).
	\end{itemize}

\paragraph{1.2.5.4}\hypertarget{par:1.2.5.4}{} Свободнейшая из всех демократий властвует между мгновениями. Ни одно мгновение не~может короновать, калечить, оправдать, заменить или ограничить какое-либо другое. Не существует последнего мига, чтобы осудить все те, что были до этого.
	\begin{itemize}
	\item Времена нередуцируемы и~поэтому <<смерть>> всегда оказывалась побежденной. Цель не~оправдывает средства. Также как и~смерть не~отрицает (condemn) жизнь.
	\end{itemize}

\paragraph{1.2.6}\hypertarget{par:1.2.6}{} Пространство и~время не~создают (frame) энтелехии. Они лишь становятся системой координат (framework) описания тех актантов, которые подчинились, локально и~временно, гегемонии другого. 
	\begin{itemize}
	\item Следовательно, существует время времен и~пространство пространств и~так далее до тех пор, пока все не~будет оговорено (has been negotiated). Мое почтение <<Клио>> Шарля Пеги\footnote{{\itshape Charles P{\'e}guy}. Clio: dialogues de l'{\^a}me pa{\"\i}enne et l'{\^a}me charnelle (Paris: Ple{\"\i}ade, 1914).}.
	\end{itemize}

\paragraph{1.2.7}\hypertarget{par:1.2.7}{} Каждая энтелехия определяет: что находится внутри, а~что снаружи; каким другим акторам она поверит, когда она решит, что принадлежит ей, а~что нет; и~какие испытания она будет использовать для того, чтобы решить верить или нет этим третейским судьям. 
	\begin{itemize}
	\item Лейбниц был прав, сказав, что у монад нет ни~дверей, ни~окон, поскольку они никогда не~выходят за пределы самих себя. Однако, они болтуны, поскольку они без конца ведут переговоры о~своих границах, о~том, кто будет вести переговоры, и~что они должны делать. В конечном итоге они как химеры, неспособные определить, где дверь, а~где окно, где право, а~где лево.
	\end{itemize}

\paragraph{1.2.7.1}\hypertarget{par:1.2.7.1}{} Не существует внешнего референта. Все референты всегда внутренние по отношению к~силам, которые используют их в~качестве критериев.

\paragraph{1.2.7.2}\hypertarget{par:1.2.7.2}{} Принцип реальности~--- это другие (люди).
	\begin{itemize}
	Интерпретация реального не~может быть отличена от реального как такового, потому что реальное это градиенты сопротивления (\hyperlink{par:1.1.5}{1.1.5}). Актант следовательно не~прекращает вести переговоры о~числе, градиентах и~природе этих различий; о~числе, авторитете и~весе тех, кто ведет переговоры; числе, качестве и~надежности критериев, которые они используют для того, чтобы оценить степень доверия судьям.
	\end{itemize}

\paragraph{1.2.8}\hypertarget{par:1.2.8}{} Каждая энтелехия создает для себя целый мир. Она определяет место себе и~всем остальным; она решает, из каких сил она состоит; она создает свое собственное время; она указывает того, кто будет первопричиной реальности этого мира. Она переводит все остальные силы в~своих интересах, и~стремится заставить их принять текст о~ней самой, который она бы~хотела, чтобы они перевели.
	\begin{itemize}
	\item Ницше называл это <<оценкой>>, а~Лейбниц <<выражением>>.
	\end{itemize}

\paragraph{1.2.9}\hypertarget{par:1.2.9}{} Это сила, о~которой мы говорим? Это говорит сила? Это актор, которого заставил говорить другой? Это интерпретация или сам объект? Это текст или мир? Мы не~можем сказать, потому что это то, за что мы боремся, построение целого слова (мира? word).
	\begin{itemize}
	\item То, что те, кто используют герменевтику, экзегетику или семиотику говорят о~текстах, может быть сказано обо всех слабостях. Долгое время было принято считать, что отношение между одним текстом и~другим всегда является вопросом интерпретации. Почему бы~не~признать, что это верно также в~отношениях между так называемыми текстами и~так называемыми объектами и~даже между самими так называемыми объектами?
	\end{itemize}

\paragraph{1.2.10}\hypertarget{par:1.2.10}{} Ничто не~может избежать примордиальных испытаний. До переговоров у нас никакого представления о~том, каковы будут испытания~--- можно ли помыслить как конфликт, игру, любовь, историю, экономику или жизнь. Мы также не~знаем примордиальны ли они или вторичны до того, как мы прибудем на~место действия. В конце концов, мы не~можем сказать до окончания [переговоров], являются ли они результатом переговоров, унаследованы от рождения, или выскочили на~коже сами по себе.

\paragraph{1.2.11}\hypertarget{par:1.2.11}{} Мы не~должны верить заранее, что мы знаем, говорим ли мы о~субъектах или объектах, людях или богах, животных, атомах или текстах. Я еще не~сказал, поскольку это именно то, что поставлено на~карту между силами: кто говорит и~о чем?
	\begin{itemize}
	\item Нам не~следует спешить отделять <<природу>> от <<культуры>>. Морские гребешки также считают, что природа суровый патрон~--- враждебный, питательный, расточительный~--- потому, что у рыбы, рыбаков, камней к~которым они прикрепляются цели иные, чем у морских гребешков.
	\end{itemize}

\paragraph{1.2.12}\hypertarget{par:1.2.12}{} Ничто само по себе не~является познаваемым или непознаваемым, высказываемым или невысказываемым, далеким или близким. Все переводится. Что может быть проще?

\paragraph{1.2.13}\hypertarget{par:1.2.13}{} Если все о~чем мы должны написать подлежит обсуждению и~переводу, тогда нам нужна, как говорил Декарт, временная мораль. Когда мы говорим об испытаниях сил, мы должны избегать использовать термины, которые закрепляют отношения в~пользу одной или другой стороны. Если это невозможно нам, по крайней мере, следует попытаться написать текст, который не~забирает время и~пространство, но вместо этого предоставляет их.

\paragraph{1.3.1}\hypertarget{par:1.3.1}{} Все энтелехии могут измерять и~быть мерой всех других энтелехий (\hyperlink{par:1.1.14}{1.1.14}). Однако некоторые силы постоянно пытаются измерять, нежели быть измеряемыми и~переводить, нежели быть переведенными. Они хотят действовать, нежели испытывать действие. Они хотят быть сильнее, чем другие.
	\begin{itemize}
	\item Я сказал <<некоторые>>, а~не~<<все>> как в~ницшевском воинственном мифе. Большинство актантов слишком разрозненны или безразличны для того, чтобы бросить вызов, слишком недисциплинированны или сбивчивы (devious), чтобы долгое время следить за теми, кто говорит от их имени; и~слишком счастливы или горды для того, чтобы командовать другими. В этой работе я говорю только о~тех слабостях, которые хотят увеличить свою силу. Другим нередуцируемым нужны скорее поэты, чем философы.
	\end{itemize}

\paragraph{1.3.2}\hypertarget{par:1.3.2}{} Учитывая, что актанты несоизмеримы, и~каждый создает мир столь же~обширный и~завершенный, как и~любой другой, как получается, что они становится больше, чем другой? Посредством заявления о~том, что он равняется нескольким (to be several), посредством ассоциирования (\hyperlink{par:1.1.9}{1.1.9}).

\paragraph{1.3.3}\hypertarget{par:1.3.3}{} Поскольку ничто по своей природе и~безотносительно к~чему-либо еще не~является эквивалентным или неэквивалентным (\hyperlink{par:1.2.1}{1.2.1}), две силы не~могут вступать в~ассоциацию без недопонимания. 
	\begin{itemize}
	\item Договор, соглашение, компромисс, переговоры, схема, комбинация~--- все эти термины могут быть использованы. Те, кто находит их унизительными и~верит в~то, что они вступают в~противоречие с более совершенными формами ассоциаций, не~могут понять, что невозможно сделать лучше, как потому, что не~существует эквивалентностей (\hyperlink{par:2.2.1}{2.2.1}), так и~потому что, ничто само по себе не~является редуцируемым или нередуцируемым к~чему-либо еще (\hyperlink{par:1.1.1}{1.1.1}).
	\end{itemize}

\paragraph{1.3.4}\hypertarget{par:1.3.4}{} Хотя энтелехии <<одинаково>> активны, они могут представать в~двух состояниях: доминирующими или доминируемыми, действующими или претерпевающими действие. Для того чтобы энтелехию можно было назвать пассивной, ей достаточно (необходимо) лишь не~отвечать.
	\begin{itemize}
	\item Я не~утверждаю, что существуют активные силы и~силы пассивные, но только то, что одна сила может действовать так, как если бы~другая была пассивна и~покорна (\hyperlink{par:1.1.14}{1.1.14}). Точка зрения пассивной силы, конечно же, совершенно другая. Существует тысяча причин для того, чтобы симулировать покорность, десять тысяч~--- для того, чтобы желать над собой власти и~сто тысяч~--- для того, чтобы оставаться безмолвным. Причины, о~которых никогда не~подозревают те, кто верит в~то, что им служат.
	\end{itemize}

\paragraph{1.3.5}\hypertarget{par:1.3.5}{} Так как какой-либо актант может стать больше, чем другой, только будучи одним [состоящим] из нескольких, и~поскольку данная ассоциация всегда является непониманием, тот, кто определяет природу ассоциации, не~встречая возражений, берет управление в~свои руки. 
	\begin{itemize}
	\item Когда две силы провозглашают собственное единство, говорит только одна; когда две силы осуществляют обмен, который они полагают равным, она всегда решает, кто определяет вещи для обмена, каким образом измеряется качество, и~где должен состояться обмен.
	\end{itemize}

\paragraph{1.3.6}\hypertarget{par:1.3.6}{} Поскольку ничто не~является эквивалентным, быть сильным означает сделать эквивалентным то, что таковым не~было. Таким образом несколько действуют как один.
	\begin{itemize}
	\item <<Дозволено не~все>>. Дискурсы и~ассоциации не~эквивалентны, потому что союзники и~аргументы завербованы в~точности так, чтобы одна ассоциация оказалась сильнее другой. Если все дискурсы представляются эквивалентными, если все выглядит так, что существуют <<языковые игры>> и~ничего больше, тогда кто-то был неубедительным. Это слабое место релятивистов. Они говорят только о~силах, которые не~в состоянии объединиться с другими для того, чтобы убедить и~победить. Повторяя <<все дозволено>>, они не~замечают работы, которая создает неэквивалентности и~асимметрию (\hyperlink{par:1.1.11}{1.1.11}).
	\end{itemize}

\paragraph{1.3.7}\hypertarget{par:1.3.7}{} Поскольку ничто не~является соизмеримым или несоизмеримым (\hyperlink{par:1.1.4}{1.1.4}), более активен тот, кому удается определять механизмы измерения.
	\begin{itemize}
	\item Существуют {\itshape акты} дифференциации и~идентификации, а~не~различия и~тождества~(\hyperlink{par:1.1.16}{1.1.16}). Слова <<тот же~самый>> и~<<другой>> являются следствием испытания сил, поражений и~побед. Они не~могут сами по себе описывать эти связи.
	\end{itemize}

%\subparagraph{Интерлюдия II: Показывающая какое это облегчение перестать редуцировать вещи.}

\paragraph{1.4.1}\hypertarget{par:1.4.1}{} Некоторые актанты испытывают свою силу против других, объявляют их пассивными и~заключают с ними союз, который они сами же~определяют. Навязывая эквивалентности, которыми они управляют, они распространяются шаг за шагом от пассивного актора к~пассивному актору.
	\begin{itemize}
	\item Мы слишком часто склонны начинать с <<обменов>>, <<равенств>>, <<трансферов>> эквивалентов. Но мы никогда не~говорим о~предварительной работе, в~ходе которой эти эквиваленты были изобретены. Это как если бы~мы говорили дорожных сетях, но никогда о~гражданском строительстве. Однако, разница между эквивалентом и~тем, чтобы сделать нечто эквивалентным такая же, как между вождением автомобиля и~строительством шоссе.
	\end{itemize}

\paragraph{1.4.2}\hypertarget{par:1.4.2}{} Когда одна слабость вербует другие, она формирует сеть до тех пор пока она в~состоянии удерживать привилегию определять их ассоциацию. 
	\begin{itemize}
	\item В сети некоторые очень отдаленные точки могут обнаружить себя соединенными, в~то время как другие, которые были соседями, удалены далеко друг от друга. Хотя каждый актор локален, он может перемещаться от места к~месту, по крайней мере, до тех пор, пока он способен договариваться об эквивалентностях, которые делают одно место таким же, как другое. Сеть, таким образом, может быть <<в действительности повсеместной>> не~становясь при этом <<универсальной>>. Какой бы~разреженной и~витиеватой сеть ни~была, она, несмотря на~это, остается локальной и~ограниченной, тонкой и~хрупкой, с вкраплениями пустоты (interspersed by space). Мы должны представить волокнистые энтелехии, растянутые и~переплетенные друг с другом (\hyperlink{par:1.2.7}{1.2.7}), которые не~могут достичь гармонии, потому что каждая определяет размер, темп и~оркестровку этой гармонии.
	\end{itemize}

\paragraph{1.4.3}\hypertarget{par:1.4.3}{} Между одной сетью и~другой так же, как между одним актантом и~другим (\hyperlink{par:1.2.7}{1.2.7}), ничто само по себе не~является соизмеримым или несоизмеримым. Таким образом, мы никогда не~покидаем сеть независимо от того, как далеко она простирается.
	\begin{itemize}
	\item Именно по этой причине кто-то может быть Комендантом Аушвица, оливковым деревом на~Корфу, водопроводчиком в~Рочестере, чайкой на~островах Сцилли, физиком в~Стэнфорде, гейсом в~Минас Жераис, китом в~водах Земли Адели, одной из бацилл Коха в~Дамиетте и~так далее. Каждая сеть создает для себя целый мир, мир, внутренности которого являются не~чем иным как внутренними выделениями тех, кто его соорудил. Ничто не~может проникнуть в~галереи такой сети, не~будучи вывернутым внешней стороной внутрь. Если бы~мы подумали, что термиты были лучшими философами, чем Лейбниц, мы могли бы~сравнить сеть с термитником~--- до тех пор, пока мы понимаем, что снаружи нет никакого солнца, которое бы~напротив ослепило (затемнило) галереи сети. Никогда невозможно будет видеть более ясно, никогда невозможно будет забраться дальше <<наружу>>, чем термит, а~наиболее широко принимаемая эквивалентность может в~ходе испытания предстать не~более крепкой, чем глиняная стена.
	\end{itemize}

\paragraph{1.4.4}\hypertarget{par:1.4.4}{} Сила прокладывает путь, делая другие силы пассивными. Затем она может двигаться в~другие места, которые ей не~принадлежат и~относиться к~ним так, как если бы~они были ее собственные.
	\begin{itemize}
	\item Я готов разговаривать (talk) о~<<логике>> (\hyperlink{par:2.0.0}{2.0.0}), но только, если она рассматривается как отрасль общественных работ или гражданского строительства. Говорить (speak) таким образом более правильно, чем болтать (talk), как Ульрих, о~Генеральном Секретариате Точности и~Души\footnote{{\itshape Роберт Мусил}. Человек без свойств. Гл. 116}.
	\end{itemize}

\paragraph{1.4.5}\hypertarget{par:1.4.5}{} Энтелехия желающая быть сильнее можно сказать создает {\itshape линии сил}. Они держать других в~строю. Они делают их более предсказуемыми. 
	\begin{itemize}
	\item Термин <<линия силы>> еще более смутный чем <<сеть>>, <<путь>>, <<галерея>> или <<логика>>, но он превосходен. Читатель не~должен быть в~состоянии определить, говорю ли я социальных существах, напечатанных схемах, причинах, машинах, театрах или привычках. Эта неясность именно тот эффект, к~которому я стремлюсь, поскольку, возможно, мы никогда не~повстречаем объекты, классифицированные таким образом.
	\end{itemize}

\paragraph{1.4.6}\hypertarget{par:1.4.6}{} Как только какому-либо актанту удается убедить других встать в~строй (fall into line), он тем самым увеличивает свою силу и~становится сильнее тех, кого он выстроил в~ряд и~убедил (\hyperlink{par:1.5.1}{1.5.1}). Этот рост может быть измерен несколькими способами. Можно сказать, что А {\itshape соединено} с другими. Хотя в~принципе любое соединение равным образом возможно, теперь связать В с А стало легче, чем А с С. А, можно также сказать, {\itshape командует} другими. Хотя в~принципе эти другие отдали свою силу А (ссудили А своей силой), они позволяют себе быть управляемыми А. Также можно сказать, что А {\itshape переводит} желания других. Хотя другие, возможно, хотят сказать что-то еще, они согласны с тем, что то, что говорит А это то, что они хотели сказать, но не~смогли выразить это в~словах. Силу А можно измерить, сказав, что оно может покупать других. Хотя в~принципе другие не~стоят одинаково (\hyperlink{par:1.2.1}{1.2.1}), E или F согласны быть эквивалентными тому, что А готово заплатить. Наконец, можно сказать, что А {\itshape объясняет} других. Хотя другие не~редуцируют себя к~А, они согласны быть его следствиями, предикатами или приложениями (\hyperlink{par:2.0.0}{2.0.0}).

В конечном счете, работа по установлению цены и~установлению эквивалентности означает, что А сильнее чем другие, несмотря на~их несоизмеримость. Оно переводит, объясняет, понимает, управляет, покупает, решает, убеждает и~заставляет их работать. 
	\begin{itemize}
	\item Иногда это накопление эквивалентов или знаков (tokens) называется <<капиталом>>, но капитал был не~первым шагом. Сначала было необходимо создать эквивалентности (\hyperlink{par:1.3.7}{1.3.7}), покорить силы и~удержать их в~одном месте достаточно долго для того, чтобы определить их масштаб и~измерить. Только потом стало возможным подсчитать прибыль (\hyperlink{par:1.3.5}{1.3.5}). Рынок это только следствие установления сетей; он не~объясняет их формирование.
	\end{itemize}

\paragraph{1.4.6.1}\hypertarget{par:1.4.6.1}{} Абсолютная сила это сила, которая была бы~способна все объяснить, все перевести, все произвести, все купить и~возместить и~быть причиной действия всего. Как универсальный эквивалент, способный заменить собой все, и~универсальное провидение способное дать жизнь всему, она могла бы~быть перводвигателем и~первопричины, из которой может быть выведено все остальное. 
	\begin{itemize}
	\item Некоторые люди говорят о~Боге, когда думают о~силе способной спасти мир с помощью своего Сына, об объяснении происхождения и~сотворения, о~переводе на~Его язык (слово) того, что каждое создание, одушевленное и~неодушевленное, желает в~глубине~своего сердца, о~том, вести нас по окольным путям Провидения к~тому, чего мы все желает. Поскольку ничто само по себе не~является редуцируемым или не~редуцируемым (\hyperlink{par:1.1.1}{1.1.1}), эта абсолютная сила также является абсолютно чистым выражением ничтожества. Именно из-за своей чистоты она всегда очаровывала мистиков, военачальников, капитанов индустрии и~ученых ищущих первопричины. <<О!>>,~--- говорят все оно сами себе,~--- <<овладейте единственной силой (городом, потир (чашу), аксиомой, банком) и~все остальное будет отдано нам>>. Чтобы избежать паники редукции, мы всегда должны говорить: <<То, что оставлено, так это все (Интерлюдии I--II). Великий Пан мертв>>.
	\end{itemize}

\paragraph{1.4.6.2}\hypertarget{par:1.4.6.2}{} Актор расширяется, пока он может убедить других, что он их включает в~себя, защищает, спасает, понимает. Он расширяется быстрее и~дальше, если он может заполучить акторов, которые уже сделали себя эквивалентными многим другим.
	\begin{itemize}
	\item Часто говорилось, что <<капитализм>> был радикальным новшеством, неслыханным прорывом, <<детерриториализацией>> доведенной до предела. Как всегда, различие это мистификация. Как и~Бог, капитализм не~существует. Не существует эквивалентов (\hyperlink{par:1.2.1}{1.2.1}), они должны быть сделаны, и~они дорогостоящи, не~ведут далеко и~их не~хватает надолго. Мы, в~лучшем случае, может создавать протяженные сети (\hyperlink{par:1.4.2}{1.4.2}). Капитализм все еще остается маргинальным даже сегодня. Скоро люди осознают, что он универсален лишь в~воображении его врагом и~адвокатов (Интерлюдия IV). Также как римские католики верят в~универсальность своей религии, даже если она течет только по римским каналам, противники и~сторонники капитализма верят, возможно, в~одну из чистейших мистических грез: в~то, что абсолютная эквивалентность достигнута. Даже Соединенные Штаты, страна подлинного капитализма, не~могут жить согласно его идеалам. Несмотря на~усилия профсоюзов и~ассоциаций работодателей, роение сил не~может стать эквивалентным без работы (\hyperlink{par:3.0.0}{3.0.0}). Мое почтение Фердинанду Броделю\footnote{{\itshape Fernand Braudel}. The Perspective of the World, Fifteenth to Eighteenth Century (New York: Harper and Row, 1985).}, который не~скрывает этот факт, и~показывает, как можно достичь удаленного контроля посредством разреженных сетей.
	\end{itemize}

\paragraph{1.5.1}\hypertarget{par:1.5.1}{} Силе не~могут быть приписаны (cannot be given) те силы, которые она выстраивает в~боевой порядок и~убеждает. По определению она может только заручиться их поддержкой (\hyperlink{par:1.3.4}{1.3.4}). Несмотря на~это, она будет претендовать на~то, что ей не~принадлежит и~добавит их силы к~своим в~новой форме: так рождается потенция.
	\begin{itemize}
	\item Когда энтелехия содержит в~себе энтелехии, которые в~ней не~содержатся, мы говорим, что она содержит их в~себе <<потенциально>>. Исток потенции лежит в~следующей путанице: теперь уже не~возможно отличить актора от союзников, которые сделали его сильным. С этого момента мы начинаем говорить, что аксиома заключает в~себе ее доказательство потенциально (in potentia); мы начинаем говорить о~государе, что он могущественный, что бытие-в-себе включает бытие для себя, тем не~менее, только <<потенциально>>. Вместе с потенцией также начинается несправедливость, потому что, не~считая немногих счастливых~--- государей, принципов, истоков, банкиров и~директоров~--- другие энтелехии, то есть все оставшиеся, становятся деталями, последствиями, приложениями, последователями, слугами, агентами~--- короче говоря, рядовыми. Монады рождаются свободными (\hyperlink{par:1.2.8}{1.2.8}), но повсюду они остаются в~цепях.
	\end{itemize}

\paragraph{1.5.1.1}\hypertarget{par:1.5.1.1}{} Разговор о~возможностях (possibilities) это иллюзия акторов, которые путешествуют, позабыв о~стоимости транспорта.
	\begin{itemize}
	\item Производство возможностей так же~дорогостояще, локально, практично, как и~производство специальных сплавов или лазеров. Возможности покупаются и~продаются, как и~все остальное. Они не~отличны по своей природе. Они, например, не~являются <<нереальными>>. Нет такой вещи как бесплатная возможность. Картотеки консультантов дорогостоящи~--- спросите тех, кто стал банкротом, потому что они произвели слишком много возможностей, но не~продали их в~достаточном количестве.
	\end{itemize}

\paragraph{1.5.2}\hypertarget{par:1.5.2}{} Если актор содержит в~себе многих других потенциально (in potentia), это впечатляет, потому что, даже находясь в~одиночестве, он предстает толпой. Поэтому ему легче завербовать других акторов и~заручиться их поддержкой.
	\begin{itemize}
	\item Хотя это начинается как блеф с утверждения о~владении тем, что было лишь одолжено, это становится реальным. Так как реально то, что сопротивляется (\hyperlink{par:1.1.5}{1.1.5}), кто способен сопротивляться энтелехии превратившейся в~толпу? Власти, троны и~господства раздают чин за чином, хотя они ни~выросли, ни~переместились и~столь же~слабы, как и~те, кто позволяет им действовать.
	\end{itemize}

\paragraph{1.5.3}\hypertarget{par:1.5.3}{} Властью никогда не~владеют. Она либо есть у нас потенциально, но тогда мы не~обладаем ей; либо она у нас есть <<действительно>> (in actu), но тогда наши союзники единственные, кто переходят к~действиям.
	\begin{itemize}
	\item Философы и~социологи власти льстят господствующим, которых, как утверждается, они критикуют. Они объясняют действия господствующих в~терминах мощи власти, однако эта власть является эффективной только как результат соучастий, потворств, компромиссов и~смешиваний (\hyperlink{par:3.4.0}{3.4.0}), которые не~объясняются властью. Понятие <<власти>> это усыпляющее действие мака, которое вызывает в~критиках дремоту как раз в~тот момент, когда безвластные государи вступают в~союз с другие, которые равным образом слабы для того, чтобы стать сильными.
	\end{itemize}

\paragraph{1.5.4}\hypertarget{par:1.5.4}{} Хотя они не~могут ни~сосчитать, ни~суммировать других, все меньшее и~меньшее количество сил, не~имеющих ничего своего, приписывают потенцию всех других мощностей (powers) себе. Это {\itshape reductio ad absurdum} целого к~ничто. Государи, которые практически ничего собой не~представляют, действуют так, как если бы~остальные, которые являются всем, уже ничего собой не~представляли.
\chapter{Социологики}

\paragraph{2.1.1}\hypertarget{par:2.1.1}{} Всякая аргументация имеет одну и~ту же~форму: одно предложение следует за~другим. Затем третье заявляет, что они идентичны, несмотря на~то, что они не~совпадают друг с~другом. Впредь второе употребляется вместо первого, а~пятое утверждает, что второе и~четвертое идентичны, несмотря на~то, что{\ldots} и~так далее, до~тех пор, пока одно предложение не~замещают, делая вид, что оно осталось на~месте и~не переводят, делая вид, что осталось верным исходному значению (faithful).

\paragraph{2.1.2}\hypertarget{par:2.1.2}{} Никогда не~существовало такой вещи как дедукция. Одно предложение {\itshape следует} за~другим, а~затем третье утверждает, что второе неявно или потенциально уже было внутри первого (\hyperlink{par:1.5.1}{1.5.1}).


\paragraph{2.1.3}\hypertarget{par:2.1.3}{} Когда устанавливается эквивалентность многих предложений, они все сворачивают в~первое, о~котором говориться, что оно <<заключает их~всех в~себе>>. Эта единственная фраза затем обсуждается, и~утверждается, что все остальные могут быть выведены из~нее <<посредством простой (pure) дедукции>>.

\paragraph{2.1.3.1}\hypertarget{par:2.1.3.1}{} Те, кто осуществляют доказательство в~присутствии других и~утверждают, что выводят (extract) одну фразу из~{\itshape другой} в~лучшем случае фокусники, а~в~худшем мошенники. Долгие годы они проделывали свои фокусы, используя кроликов и~шляпы, взятые у~зрителей.

\paragraph{2.1.3.2}\hypertarget{par:2.1.3.2}{} Только учителя могут заявлять, что они способны вывести одно предложение из~в~другого средствами <<простой, формальной дедукции>>. Они заранее знают вывод из~аргументации, которую, как утверждается, они разворачивают. Систематичные доводы, выученные {\itshape медленно} и~{\itshape беспорядочно} развертываются ими на~высокой скорости, один за~другим, скрывая то, что происходит за~кулисами позади классной доски, беспорядочную историю, которая ведет к~тому, что эта пропозиция должна быть соединена с~той. Они
предлагают то, что потенциально (in potentia) содержит в~себе все следствия ради поклонения своих учеников, которые горячо верят в~то, что они дедуцировали одно из~другого.
	\begin{itemize}
	\item Без школы никто бы~не верил в~эту религию дедукции. Мы~также можем сказать, что пропозиции спинозовой <<Этики>> содержаться <<все в>> первой пропозиции, или что десерт содержится в~антреме\footnote{Блюдо подаваемое между рыбой и~жарким.~--- {\itshape Прим. перев.}}. Но~школяры всегда были зачарованы абсолютной (совершенной) шпаргалкой предложенной принципом Лапласа: держать все знание в~ладони нашей руки, достав его из~каблука нашей обуви.
	\end{itemize}

\paragraph{2.1.4}\hypertarget{par:2.1.4}{} Аргументы образуют систему или структуру, только если мы~забыли проверить их. Что? Если бы~я~атаковал один элемент, столпились бы~все остальные вокруг меня, не~мешкая ни~секунды? Вряд ли! Всякое собрание актантов включает в~себя ленивых, трусливых, двойных агентов, мечтателей, равнодушных и~диссидентов. Да, я~согласен с~вами, что страх от~того, что они увидели как А, В, или Е~идут на~помощь может настолько впечатлить людей, что они сдадутся. Но~если они выстоят, вероятнее всего, что В~будет
диссоциирован, потому что С~шел слишком медленно, Е~подавлен, F~--- предатель, а~G не~мог помочь потому, что пытался предотвратить предательство F. 
	\begin{itemize}
	\item 
	Как хорошо известно, альянс между логиками и~армией привел генерала Штумма к~тому, чтобы проверить надежность структур в~библиотеке в~Вене\footnote{{\itshape Роберт Мусил}. Человек без свойств. Гл.~85.}. Он~был очень разочарован. В~Париже все еще верим в~структуры, потому что мы~не заботимся о~том, чтобы проверять их~лояльность.
	\end{itemize}

\paragraph{2.1.5}\hypertarget{par:2.1.5}{} Комментарии никогда не~бывают преданными. Либо это повторение, которое не~является комментарием, либо это комментарий, который произносится {\itshape по-иному}. Другими словами это перевод и~предательство. Несмотря на~это, экзегеты не~устают приписывать толкования текстам. Текст набухает толкованиями, которые он~потенциально (in potentia) должен содержать в~себя для того, чтобы оправдаться все эти прочтения. 
	\begin{itemize}
	\item 
	Тексты никогда не~бывают преданными друг другу, но~они всегда держат друг друга на~некоторой дистанции.
	\end{itemize}

\paragraph{2.1.6}\hypertarget{par:2.1.6}{} Мы~говорим <<кто управляет причиной, управляет следствием>>, как если бы~следствие потенциально содержалось внутри причины. Однако, ни~одно слово не~может стать причиной другого. Слова {\itshape следуют} друг за~другом в~рассказе. И~только дальше в~рассказе один герой становится <<причиной>>, а~другой <<последствием>>. Единственное следствие, которое следует принимать во~внимание~--- это эффект производимый на~публику тем или иным альянсом слов: <<Нет, он~преувеличивает>>, или <<это написано хорошо>>, или еще <<очень понятно>>, <<очень убедительно>>, <<как самовлюбленно>>, <<какая скука>>.

\paragraph{2.1.7}\hypertarget{par:2.1.7}{} Теорий нет. Есть тексты, которым мы, как ленивые властелины, почтительно приписываем те~вещи, которых они не~совершали, подразумевали, предвидели или вызывали. Теории никогда не~бывают одни, также как на~открытой местности не~бывает дорожных развязок без шоссе, которые соединяют и~перенаправляют.

\paragraph{2.1.7.1}\hypertarget{par:2.1.7.1}{} В~теории, теории существуют. На~практике, нет. 
	\begin{itemize}
	\item 
	Никто никогда не~выводил всю геометрию их~аксиом и~постулатов Евклида. Но~<<в~теории>>, говорят они, <<всякий и~везде может>> вывести <<в любое время>> <<всю>> геометрию из~аксиом <<одного>> Евклида. На~практике, {\itshape такого не~случалось ни~с~кем}. Но~никому не~требовалось приходить к~такому заключению, поскольку <<в теории>> обратное возможно. А~колдунов презирают, потому что, как говорят, они не~способны принять факты, даже когда факты противоречат им~каждый день на~протяжении столетий!
	\end{itemize}

\paragraph{2.1.7.2}\hypertarget{par:2.1.7.2}{} Не~существует метаязыков, только инфраязыки. Иными словами существуют только языки. Мы~не можем более редуцировать один язык к~другому, иначе как построив Вавилонскую башню. 
	\begin{itemize}
	\item 
	Те, кто говорит о~метаязыке, я~думаю, должно быть имеют в~виду диалект (пиджин) специалистов, который слишком скуден даже для того, чтобы перевести то, что говорится на~кухне.
	\end{itemize}

\paragraph{2.1.7.3}\hypertarget{par:2.1.7.3}{} Ежедневная практика не~нуждается в~теоретике, который бы~обнаружил <<лежащую в~ее основании структуру>>. <<Сознание>> не~лежит в~основании практики, но~оно является чем-то еще, где-нибудь еще в~другой сети. Практика ни~в~чем не~испытывает недостатка. 
	\begin{itemize}
	\item 
	{\itshape Где} находятся бессознательные структуры примитивных мифов? В~Африке? В~Бразилии? Нет! Они среди архивных карточек кабинета Леви-Стросса. Если они выходят за~пределы Коллеж де~Франс на~рю де~Эколь, то~это благодаря его книгам и~ученикам. Если их~находят в~Баия\footnote{Штат на~северо-востоке Бразилии.~--- {\itshape Прим. перев.}}, то~это потому что они там преподаются.
	\end{itemize}

\paragraph{2.1.8}\hypertarget{par:2.1.8}{} До~тех пор, пока форма имеет значение (\hyperlink{par:2.1.1}{2.1.1}), все аргументы одинаково хороши.
Все что нам нужно это серия предложений, а~затем мы~скажем, что одни из~них одинаковы, а~другие различны (\hyperlink{par:2.1.2}{2.1.2}). Предложения в~таком случае сплетаются в~косы, локоны, гирлянды, венки и~паутины. Это можно сделать {\itshape всегда}, не~так ли? В~результате {\itshape некоторые} перемещения становятся легче, а~некоторые труднее. 
	\begin{itemize}
	\item 
	Никто не~может классифицировать аргументы с~точки зрения их~{\itshape формальных} качеств. Если вы~настаиваете, мы~можем выстроить их~в~ряд с~точки зрения их~{\itshape материальных} качеств.
	\end{itemize}

\paragraph{2.1.8.1}\hypertarget{par:2.1.8.1}{} Ничто само по~себе не~является логичным или нелогичным. Тропа всегда куда-либо ведет. Все что нам нужно знать это куда она ведет, и~какой транспорт (груз) по~ней должен проходить. Кто бы~был столь глуп, что назвал бы~шоссе <<логичными>>, дороги <<нелогичными>>, а~козьи тропы <<абсурдными>>?

\paragraph{2.1.8.2}\hypertarget{par:2.1.8.2}{} Ни~один набор предложений не~является сам по~себе последовательным или непоследовательным (\hyperlink{par:1.1.14}{1.1.14}); все, что нам необходимо знать~--- это кто проверяет его с~какими союзниками и~как долго. Последовательность чувствуется (\hyperlink{par:1.1.2}{1.1.2}); это не~диплом, не~медаль или торговая марка.

\paragraph{2.1.8.3}\hypertarget{par:2.1.8.3}{} Нить {\itshape доказательства} никогда не~бывает прямой. Те, кто говорят о~<<логике>>, никогда не~видели как что-либо прядут, заплетают, выстраивают в~линию, ткут или дедуцируют. Бабочка летит по~более прямой линии, чем рассуждающий разум. (Иногда, конечно, вытканные орнаменты могут представлять собой, радующую глаз, прямую
линию.)

\paragraph{2.1.8.4}\hypertarget{par:2.1.8.4}{} <<Разум>> применяется для работы (\hyperlink{par:2.5.4}{2.5.4}) по~распределению согласия и~несогласия между словами. Это дело вкуса и~чувства, ноу-хау и~понимания, класса и~статуса. Мы~оскорбляем, хмуримся, дуемся, сжимаем кулаки, приходим в~восторг, плюем, вздыхаем и~мечтаем. Кто рассуждает?
	\begin{itemize}
	\item 
	Антрополог, изучающий язык тела может нарисовать эскиз мышления преподавателя из~Кембриджа или банкира с~Уолл Стрит.
	\end{itemize}

\paragraph{2.1.9}\hypertarget{par:2.1.9}{} Так как {\itshape количество} тождеств и~различий, которое мы~вынуждены {\itshape делить вместе} (share) остается постоянным (\hyperlink{par:2.1.8}{2.1.8}), это не~в~нашей власти быть нелогичными и~иррациональными (\hyperlink{par:2.1.8.1}{2.1.8.1}). Тем не~менее, существует множество способов расставить всевозможные <<вследствие>>, <<из-за>>, <<в противоречии с>>, <<несмотря на>>. Никто так невнимателен к~<<non sequiturs>> как логики, волшебники и~режиссеры. Когда придумываются эффекты, мы~с~особой тщательностью должны выбрать, что за~чем последует. Мы~должны определить, когда станет известно имя предателя или аксиомы и~приготовиться к~выходу актеров на~сцену, который произведет наибольшее впечатление на~аудиторию. Мы~должны определит единицы измерения времени и~пространства, причины и~принципы. По~мере того, как мы~со вкусом отобрали теоремы и~реплики в~сторону, нам необходимо выбрать писать ли~<<more geometrico>> или <<more populo>>. Короче, убеждение зависит от~жанра, который мы~избрали.
	\begin{itemize}
	\item 
 Мы~забываем, что среди Азанде столько же~скептиков, любителей силлогизмов, попперианцев и~рационалистов, сколько их~среди коперинков и~сцилардов\footnote{Лео Сцилард (1898---1964)~--- американский физик. В~1934 обнаружил (совместно с~Т.\,Чалмерсом) эффект разрушения химической связи под действием нейтронов. В~1939 наряду с~другими показал возможность осуществления цепной ядерной реакции при делении ядер урана. Вместе с~Э.\,Ферми определил критическую массу урана-235 и~принял участие в~создании первого ядерного реактора (1942). --- {\itshape Прим. перев.}}. Так как количество согласий и~несогласий постоянно, мы~не можем {\itshape чисто} отделить мифические фикции от~научных исследований. Это может быть сделано только грязным способом и~потому это настоящая бойня. Художник, который предпочитает лишь оттенки серого, является художником в~не меньшей степени, чем тот, кто использует кричащие краски. Существуют доказательства такие же~суровые как зима и~эластичные (springlike) доказательства, но~все они, тем не~менее, доказательства.
	\end{itemize}

\paragraph{}\hypertarget{par:}{} Так как ничто ничему не~присуще, то~диалектика это небылица. Противоречия являются предметом переговоров как и~все остальное. Они не~даны, а~выстраиваются.

\paragraph{}\hypertarget{par:}{} Если магия это форма практики, которая дает некоторым словам возможность воздействовать на~<<вещи>>, тогда мир логики, дедукции и~теории должен быть назван <<магическим>>: но~это {\itshape наша} магия. 
	\begin{itemize}
	\item 
	Так же~как греки называли высокие языки парфян, абиссинцев или сарматов <<варварскими>>, мы~называем превосходные аргументы (\hyperlink{par:2.1.8}{2.1.8}) тех, кто верит в~другие возможности дедукции, <<нелогичными>>.
	\end{itemize}

\paragraph{2.2.1}\hypertarget{par:2.2.1}{} Сказать что-либо, значит сказать это другими словами. Другими словами, это значит перевести.
	\begin{itemize}
	\item 
	Слово произносится вместо другого, с~которым оно не~совпадает. Третье слово говорит, что оно одинаковы (\hyperlink{par:2.1.1}{2.1.1}). А~это не~А, а~В и~С. Рим больше не~в~Риме, но~на Крите и~у Саксов. Это называется <<утверждением>> (предикацией). {\itshape То~есть}, мы~не можем говорить правильно, переходя от~{\itshape одного} (same) {\itshape к~тому же~самому} (same), но~только небрежно двигаясь от~одного {\itshape к~другому}.
	\end{itemize}

\paragraph{2.2.2}\hypertarget{par:2.2.2}{} Поскольку ничто не~является редуцируемым или нередуцируемым к~чему-либо еще (\hyperlink{par:1.1.1}{1.1.1}) и~не существует эквивалентностей (\hyperlink{par:1.2.1}{1.2.1}), о~каждой паре слов можно сказать, что они тождественны или не~имеют ничего общего. Таким образом, нет очевидного способа отделить буквальный смысл от~{\itshape переносного}\footnote{{\itshape Mary Hesse}. The Structure of~Scientific Inference (London: Macmillan, 1974).}. Каждая группа слов может быть скабрезной, точной, метафоричной, аллегоричной, технической, правильной, или надуманной.

\paragraph{2.2.3}\hypertarget{par:2.2.3}{} Ничто само себе не~является <<высказываемым>> или <<невысказываемым>>. Все переводится (\hyperlink{par:1.2.12}{1.2.12}). С~того момента как одно слово всегда одалживает свой смысл другому, от~которого оно, однако, отличается, не~в~нашей власти более говорить о~том, что верно или ошибочно, кроме как уберечь маленькую мельницу сказки от~вымучивания сути.

\paragraph{2.2.4}\hypertarget{par:2.2.4}{} Либо сказано то~же самое и~тогда ничего не~сказано, либо нечто сказано, но~это нечто другое. Выбор должен быть сделан. Все зависит от~расстояния, которое мы~готовы пройти и~сил, которые мы~готовы уговорить, в~попытке сделать эквивалентными слова,
которые бесконечно далеки друг от~друга. 

\paragraph{2.2.5}\hypertarget{par:2.2.5}{} Нас могут понимать, то~есть окружать, отвлекать, предавать, замещать, передавать, но~нас никогда не~понимают {\itshape надлежащим} образом. Если сообщение транспортируется, то~оно трансформируется. Нет сообщения, которое было бы~просто прочитано.

\paragraph{2.3.1}\hypertarget{par:2.3.1}{} Мы~никогда не~заговариваем словами, которые свободно ассоциируются, но~скорее на~своем родном языке (\hyperlink{par:2.2.2}{2.2.2}). 
	\begin{itemize}
	\item 
	Другие уже поиграли словами, когда мы~начали говорить (1.1.10). Год за~годом, век за~веком другие делали определенные ассоциации звуков, слогов, фраз и~аргументов возможными или невозможными, правильными или варварскими, приличными или вульгарными, ложными или элегантными, точными или бессмысленными. Хотя ни~одно из~этих образований не~является таким прочным, как утверждается (\hyperlink{par:2.1.4}{2.1.4}), если мы~пожелаем их~разрушить и~переделать, мы~станем объектом, который получает удары, плохие оценки, ласки, артиллерийский огонь или аплодисменты.
	\end{itemize}

\paragraph{2.3.2}\hypertarget{par:2.3.2}{} Хотя не~существует точного или переносного смысла, возможно присвоить слово, редуцировать его значения и~альянсы и~жестко связать его службой другому [слову]. 
	\begin{itemize}
	\item 
	Даже все ароматы Аравии не~сдобрят эту маленькую метафору, чтобы сделать ее~фигуральной (\hyperlink{par:2.2.2}{2.2.2}). 
	\end{itemize}

\paragraph{2.3.3}\hypertarget{par:2.3.3}{} Все ассоциации звуков, слов, и~предложений эквивалентны (\hyperlink{par:2.1.8}{2.1.8}), но~так как они вступают в~ассоциации в~точности так, что они {\itshape более не} эквивалентны друг другу (\hyperlink{par:1.3.6}{1.3.6}), в~конечном счете есть победители и~побежденные, сильные и~слабые, смысл и~бессмыслица, и~выражения, которые буквальны и~метафоричны.

\paragraph{2.3.4}\hypertarget{par:2.3.4}{} Ничто само по~себе не~является логичным или нелогичным (\hyperlink{par:1.2.8}{1.2.8}), но~не все равным образом является убедительным. Есть только одно правило: <<Дозволено все>>; говори все что угодно до~тех пор, пока те, к~кому ты~обращался, не~будут убеждены. Ты~сказал, что для того, чтобы попасть из~В~в С~необходимо пройти через D~и E? Если другие не~выразили несогласие и~не предложили другие пути, то~вы были убедительны. Они идут из~В~в~С~по предложенному пути, даже если никто не~хочет покидать В~ради и~есть много других маршрутов, которые можно было бы~избрать. Те, кого вы~стремились убедить, уступили. Для них больше нет <<Дозволено все>>. Это придется сделать, поскольку {\itshape невозможно сделать что-либо лучше} (\hyperlink{par:1.2.1}{1.2.1}).

\paragraph{2.3.5}\hypertarget{par:2.3.5}{} Мы~можем говорить все, что пожелаем, и~все же~не можем. Как только мы~произнесли и~вновь собрали (rallied) слова, образование других союзов стало более легким или более трудным. Асимметрия растет вместе с~потоком слов; в~потоке смысла пороги и~плато скоро разрушаются. На~поле брани между словами заключаются альянсы. Нам верят, нас ненавидят, нам помогают, нас предают. Мы~больше не~управляем игрой. Новые значения предлагаются, в~то время как другие отбрасываются; нас комментируют, дедуцируют, понимают или игнорируют. Именно так: мы~более не~можем говорить то, что пожелаем.

\paragraph{2.4.1}\hypertarget{par:2.4.1}{} Каким образом одна серия предложений становится настолько <<сильнее>> чем другая, что последняя становится <<нелогичной>>, <<абсурдной>>, <<противоречивой>>, <<вымышленной>> или <<несерьезной>>? Как и~сила (\hyperlink{par:1.3.2}{1.3.2}) аргумент становится сильным, лишь используя все, что оказывается под рукой. Таким способом мы~можем заставить какой-либо актант признать, что это предложение является <<противоречивым>> или <<абсурдным>>, до~тех пор, пока не~найдется никого, кто бы~и~далее считал этот аргумент нелогичным.
	\begin{itemize}
	\item 
	Риторика не~может рассматриваться как сила согласования предложений, потому что если нечто названо <<риторикой>>, тогда оно слабо и~уже проиграло (\hyperlink{par:1.3.6}{1.3.6}). Логика не~может рассматриваться как сила, так как она приписывает победу, которая является результатом нескольких предложений, <<формальным>> качествам общим для всех аргументов (\hyperlink{par:2.1.0}{2.1.0}). В~таком случае семиотика вновь остается неадекватной, потому что упорно продолжает принимать во~внимание только тексты или символы вместо того, чтобы также иметь дело с~<<вещами в~себе>>.
	\end{itemize}

\paragraph{2.4.2}\hypertarget{par:2.4.2}{} Слова никогда не~бывают одни и~они не~окружены только словами; они были бы~неслышимы.
	\begin{itemize}
	\item 
	Актант может сделать союзником все что угодно, так ничто само по~себе не~является редуцируемым ил~нередуцируемым (\hyperlink{par:1.1.1}{1.1.1}) и~поскольку нет эквивалентностей без работы по~установлению эквивалентов (\hyperlink{par:1.4.0}{1.4.0}). Слово, таким образом, может в~ступить в~партнерство со~смыслом, с~последовательностью слов, утверждением, нейроном, жестом, стеной, машиной, лицом\ldots со~всем, до~тех пор, пока различия в~сопротивлении позволяет одной силе быть более устойчивой, чем другая. Где написано, что слова могут вступать в~ассоциации только с~другими словами? Всякий раз, когда цепочка слов проверяется на~прочность, мы~измеряем {\itshape преданность} стен, нейронов, чувств, жестов, сердец, умов, бумажников~--- то~есть гетерогенного множества союзников, наемников, друзей и~куртизанок. Но~мы не~выносим эту нечистоту и~беспорядочность.
	\end{itemize}

\paragraph{2.4.3}\hypertarget{par:2.4.3}{} Мы~не можем различить те~моменты, когда мы~сильны и~когда мы~правы. 
	\begin{itemize}
	\item 
	Испытания сил лишь иногда принимают форму демонстрации насилия (\hyperlink{par:1.1.2}{1.1.2}); они также предстают во~многих других обличиях. На~одном полюсе актанты действуют столь мирно, что они смешиваются с~фоном и~становятся частью (flow) природы. Их~деятельность настолько спокойна, что кажется, никакая сила вообще не~используется (\hyperlink{par:1.1.6}{1.1.6}). На~другом полюсе кровопролитие~--- тотальная война без ритуалов, цели или подготовки. Бывает ли~так когда-либо? Где-то между, я~полагаю, находится великая риторическая игра, где сила слова может ослабить альянсы и~показать что-либо, где очень, очень редко, {\itshape при прочих равных условиях}, кто-либо говорит и~убеждает. Мы~всегда ограничиваем себя разговорами об~этих трех случаях из~учебников; я~хочу поговорить обо всех других случаях также.
	\end{itemize}

\paragraph{2.4.4}\hypertarget{par:2.4.4}{} Языки ни~господствуют, ни~находятся в~подчинении, ни~существуют, ни~не существуют. Они такие же~энтелехии как все другие. Они ищут союзников в~своих интересах и, как и~другие актанты, выстраивают из~них целый мир с~такими же~запретами и~привилегиями.
	\begin{itemize}
	\item 
	Только лингвисты могут верить в~то, что слова вступают в~ассоциации только с~другими словами с~тем, чтобы образовать лингвистическую структуру. Они забывают о~трудностях, возникших у~них в~попытке отвязать слова от~их союзников, когда они изобретали свои структуры. То, что слова это силы подобные другим со~своими временами и~пространствами, своим <<габитусом>> и~своими дружбами (дружескими отношениями), удивительно только для тех, кто верит, что <<люди>> существуют или владеют языком. Вы~когда-либо боролись со~словом? Разве ваш язык не~уставал от~разговоров? Все, что сопротивляется~--- реально (\hyperlink{par:1.1.5}{1.1.5}). Кто поверит, что существует чистая история одних лишь слов?
	\end{itemize}

\paragraph{2.4.5}\hypertarget{par:2.4.5}{} Невозможно надолго отделить тех актантов, которые собираются играть роль <<слов>> от~тех, которые будут играть роль <<вещей>>. Если мы~говорим только о~языках и~<<языковых играх>>, мы~уже проиграли, поскольку мы~пропустили момент перераспределения ролей и~костюмов. 
	\begin{itemize}
	\item 
 Не~так давно существовала тенденция наделять язык привилегией. Долгое время считалось, что он~прозрачен и~является единственным среди актантов, лишенным как плотности, так и~насилия. Затем начали расти сомнения относительно его прозрачности. Выражалась надежда, что его прозрачность может быть восстановлена посредством очищения языка, так же~как мы~протираем окно. Язык был столь привилегированным, что его стала единственным достойным занятием для поколений кантов и~витгенштейнов. Потом в~пятидесятые пришло осознание, что язык был непрозрачным, плотным и~тяжелым. Это открытие, однако, не~означало, что он~утратил свое привилегированное положение и~был приравнен к~другим силам, которые переводят и~переводятся им. Напротив, была совершена попытка редуцировать все силы к~означающему. Текст был превращен в~<<объект>>. Это были <<поворотные шестидесятые>>, от~Леви-Стросса к~Лакану через Барта и~Фуко. Что за~суета! Все, что сказано об~означающем верно, но~это же~должно быть сказано также о~любой другой энтелехии (\hyperlink{par:1.2.9}{1.2.9}). В~языке нет ничего особенного, что бы~позволяло хоть сколько-нибудь еще отделять его от~всех остальных. 
	\end{itemize}

\paragraph{2.4.6}\hypertarget{par:2.4.6}{} Устойчивость альянса определяется тем количеством акторов, которое необходимо собрать для того, чтобы разъединить его (\hyperlink{par:2.1.8.2}{2.1.8.2}). Следовательно, мы~должны проверить его, если мы~хотим знать, с~чем мы~имеем дело~--- если мы~хотим знать {\itshape откуда} в~действительности проистекает та~эффективность, которую так часто приписывают отдельному слову, единичному тексту или знаку в~небесах.
	\begin{itemize}
	\item 
	Они говорят, <<Вы не~может перейти от~В~к D, не~миновав С~или Е>>. <<Если вы~неуверенны в~С, тогда вы~также сомневаетесь в~В и~D>>. <<Если вы~в~В, то~вы, следовательно, должны идти в~D>>. Каждое из~этих утверждений может с~одинаковым успехом быть сделано относительно проблем геометрии, генеалогии, подземных работ, драки между мужем и~женой или рисунке (varnish) на~каноэ. Каждое из~них может быть высказано относительно любой прочной (устойчивой) формы (\hyperlink{par:1.1.6}{1.1.6}).Вот поэтому <<логика>> является отраслью общественных работ (\hyperlink{par:1.4.4}{1.4.4}). Мы~скорее не~сможем больше вести машину по~шоссе, чем сомневаться в~законах Ньютона. {\itshape В~каждом случае причины одни и~те же}: удаленные точки были соединены путями, которые поначалу были узкими, а~затем были расширены и~надлежащим образом вымощены. К~этому времени ничто кроме революции или природного катаклизма не~заставит тех, кто использует эти пути, предложить путешественнику другой маршрут. Одна логика разрушается другой так же, как бульдозер сносит лачугу. В~этой перестановке нет ничего сверхъестественного, хотя она может быть опасной, если экспроприированные отомстят за~себя.
	\end{itemize}

\paragraph{2.4.7}\hypertarget{par:2.4.7}{} Гетерогенные альянсы, которые делают определенные цепочки слов согласованными (\hyperlink{par:2.1.8.0}{2.1.8.0}) образуют сети, которые могут быть длинными и~несоизмеримыми~--- до~тех пор, пока они не~решать снять друг с~друга мерку. <<Можешь ли~ты сомневаться в~связи, которая соединяет В~с С?>>. Нет, не~могу, до~тех пор, пока я~не буду готов потерять свое здоровье, репутацию или кошелек>>. <<Можешь ли~ты ослабить узы, которые связывают D~и Е?>> <<Да, но~только с~помощью золота, терпения и~злобы>>. Необходимое и~контингентное (\hyperlink{par:1.1.5}{1.1.5}), возможное и~невозможное, твердое и~мягкое (\hyperlink{par:1.1.6}{1.1.6}), реальное и~нереальное (\hyperlink{par:1.1.5.2}{1.1.5.2})~--- все они увеличиваются таким образом. Для энтелехии существуют только {\itshape более сильные} или {\itshape более слабые} интеракции, с~помощью которых можно построить мир.

\paragraph{2.4.8}\hypertarget{par:2.4.8}{} Предложение сохраняет свое единство (hold together) не~потому, что оно истинно, но~мы говорим, что оно <<истинно>>, поскольку оно сохраняет свое единство. За~что оно держится? За~множество вещей. Почему? Потому что оно связало свою судьбу с~чем-то, что было под рукой и~что было надежнее, чем она сама. В~результате, никто не~может поколебать его (shake itshape loose), не~сотрясая все остального. 
	\begin{itemize}
	\item 
	Ничего больше, верующие; ничего меньше, релятивисты.
	\end{itemize}

\paragraph{2.5.1}\hypertarget{par:2.5.1}{} Недостаточно быть самым сильным; они еще хотят быть самыми лучшими. Недостаточно победить; они также хотят быть правыми. 
	\begin{itemize}
	\item 
	<<Самые сильные доводы всегда уступают доводам самых сильных>>. Вот это приложение в~виде доброты это то, от~чего я~бы хотел избавиться. Аргументация самых сильных просто самая сильная. <<Этот подлунный мир>> сильно бы~изменился, если бы~мы избавились от~этого не~существующего приложения, если бы~мы лишили победителей этого маленького дополнения. Для начала, он~перестал бы~быть подлым (base) миром.
	\end{itemize}

\paragraph{2.5.2}\hypertarget{par:2.5.2}{} Власть это пламя, которое заставляет нас путать силу с~теми союзниками, которые сделали ее~сильной (\hyperlink{par:1.5.1}{1.5.1}). Если бы~мы надели маску сварщика, мы~бы могли смотреть на~место сварки, не~будучи при этом ослепленными. Я~не хочу больше принимать блеск щита (кадр на~экране) за~лицо сероглазой Афины, по~крайне мере до~тех пор, пока я~этого не~захочу.

\paragraph{2.5.3}\hypertarget{par:2.5.3}{} Мы~можем избежать того, чтобы быть запуганными теми, кто присваивает слова и~претендует на~то, чтобы быть <<у власти>>. 
	\begin{itemize}
	\item 
 В~ночь Шабаша ведьмы летают in~potentia, в~то время как их~тела спят на~соломе. Никто не~верит этому сейчас, но~магия продолжает существовать, магия тех, кто верит, что они могут путешествовать {\itshape дальше}, чем их~тела и~{\itshape за~пределы} действия их~силы. Шабаш магов разума проходит каждый день и~эта магия не~нашла своих скептиков (\hyperlink{par:4.0.0}{4.0.0}).
	\end{itemize}

\paragraph{2.5.4}\hypertarget{par:2.5.4}{} Мы~не думаем и~не рассуждаем. Мы~скорее {\itshape работаем} с~хрупкими материалами~--- текстами, записями, следами или рисунками~--- с~другими людьми. Эти материалы ассоциируются и~диссоциируются благодаря отваге и~усилию; у~них нет смысла, ценности или связности вне узкой сети, которая на~какое-то время удерживает их~вместе. Конечно, мы~можем {\itshape расширить} эту сеть, рекрутируя других акторов, и~мы можем сделать {\itshape усилить} ее, заручившись поддержкой более прочных материалов. Однако мы~не можем покинуть ее~даже во~сне.
	\begin{itemize}
	\item 
	Торговля мясом простирается так далеко, как простирается искусство мясника, их~прилавки, их~холодильники, их~пастбища и~их бойни. Рядом с~мясником~--- у~бакалейщика, например~--- торговли мясом нет. Также обстоит дело с~психоанализом, теоретической физикой, философией, бухгалтерским делом, социальной безопасностью, короче, со~всеми ремеслами. Однако {\itshape некоторые ремесла }утверждают, что они могут потенциально или <<в теории>> распространятся за~пределами сетей, в~которых они практикуют. Мясник никогда бы~не поддержал идею редукции теоретической физики к~искусству мясника, но~психоаналитик утверждает, что может редуцировать торговлю мясом к~убийству отца, а~эпистемолог с~успехом говорит об~<<основаниях физики>>. Хотя все сети одного размера, высокомерие распределено не~равномерно.
	\end{itemize}

\paragraph{2.5.5}\hypertarget{par:2.5.5}{} Мы~не можем освободиться от~властвующих средствами <<мысли>>, но~мы освободимся от~власти, когда мы~превратим <<мысль>> в~работу. 
	\begin{itemize}
	\item 
	Разговорные выражения, которые мы~используем для обозначения работы мысли (ломать голову, шевелить мозгами, переваривать идеи) являются не~метафорами, но~указывают на~работу рук и~тел общую для всех ремесел. Почему тогда это ремесло мышления в~отличие от~всех других не~считается ручным. Потому что иначе оно должно будет лишить привилегии выходить за~пределы его сетей. Стало бы~более не~возможно распространяться сверх простой практики ремесленников (\hyperlink{par:2.1.7.2}{2.1.7.2}). Все предпочитают выделять интеллектуалов (даже если только для того, чтобы высмеять их), чем признать что они работают. Даже если верующие не~извлекают выгоды из~этих бесплатных путешествий, они не~хотят, чтобы другие лишались привилегии парить вне времени и~пространства.
	\end{itemize}

\paragraph{2.5.6}\hypertarget{par:2.5.6}{} Нет разницы с~одной стороны между теми, кто редуцирует и~с другой~--- теми, кто желает дополнения в~виде души. Две группы одинаковы. Когда они сводят все к~ничему, они чувствуют, что все остальное ускользает от~них. Поэтому они стремятся ухватиться за~это с~помощью символов. 
	\begin{itemize}
	\item 
	Символическое это магия тех, кто потерял мир. Это единственный способ, который они нашли, чтобы сохранить <<в дополнение>> к~<<объективным вещам>> <<духовную атмосферу>>, без которой вещи были бы~<<лишь>> <<естественными>>.
	\end{itemize}

\paragraph{2.5.6.1}\hypertarget{par:2.5.6.1}{} Мы~можем быть уверены, что всякий раз, когда они заговаривают о~символах, они пытаются путешествовать не~заплатив. Они надеются двигаться, не~покидая дома, надеются связать два актанта без грузовиков, газа и~автострады. 
	\begin{itemize}
	\item 
	Тех, кто говорит о~<<символическом>> поведении должно исследовать как волшебников. Они говорят, что волшебство постигается через слова, что не~может быть достигнуто с~помощью <<действенной практики>>. Но~это определение должно быть приложено {\itshape к~ним самим}. Не~будучи способными овладеть силами посредством их~испытания, они изобрели <<символы>>, которые, будучи <<добавленными к~реальности>>, ничего не~стоят и~не потребляют.
	\end{itemize}

\paragraph{2.5.6.2}\hypertarget{par:2.5.6.2}{} Так как реально все, что сопротивляется, <<символическое>> нельзя добавить к~<<реальному>>. До~того, как к~ним <<добавили>> символы, актанты не~испытывали недостатка в~чем-либо. Поэтому, если мы~прекратим редуцировать их, это ненужное добавление, в~свою очередь, превратиться в~ничто. 
	\begin{itemize}
	\item 
	Если бы~только мы~освободились от~символического, <<реальное>> было бы~нам возвращено. Я~готов согласиться с~тем, что рыбы могут быть божествами, звездами или едой, что рыба может вызвать у~меня болезнь и~играть разные роли в~мифах о~происхождении. Они живут своей жизнью, а~мы своей. В~действительности же, наши жизни перекрывают и~используют друг друга так долго, что в~каждом ките по~Ионе и~по киту в~каждом фолианте Мельвиля. Кто остановит переводы рыболовства, океанографии, дайвинга~--- всего того, что мы~и~рыбы используем для того, чтобы измерить друг друга? Этот человек еще не~родился. (Интерлюдия IV). Те, кто хочет {\itshape отделить} <<символическую>> рыбу от~ее <<реального>> двойника сами должны быть изолированы и~заключены в~ тюрьму (\hyperlink{par:3.0.0}{3.0.0}).
	\end{itemize}

\paragraph{2.5.6.3}\hypertarget{par:2.5.6.3}{} Мы~не страдаем от~нехватки духа. Мы~напротив страдаем от~{\itshape слишком большого количества} неприкаянных духов, которым никогда не~предлагалось достойного погребения. Они бродят повсюду средь бела дня как несчастные призраки. Я~хочу изгнать этих духов и~убедить их~оставить нас наедине с~живыми.

\paragraph{2.6.1}\hypertarget{par:2.6.1}{}\hypertarget{par:2.6.1}{Всякое исследование оснований и~ истоков поверхностно, так как оно надеется выявить энтелехии, которые потенциально содержат в~себе другие [энтелехии].} Это невозможно. Если мы~хотим проникнуть вглубь, мы~должны последовать за~силами в~их сговорах и~переводах. Мы~должны следовать за~ними, куда бы~они ни~вели, и~составить список их~союзников, какими бы~многочисленными и~широко распространенными они не~были. 
	\begin{itemize}
	\item 
	Те, кто ищут основания~--- редукционисты по~определению и~горды этим. Они всегда пытаются редуцировать некоторое количество сил к~одной силе, от~которой все остальные могут быть произведены. Чем больше их~успех, тем более незначительной становится избранная [сила]. Наиболее глубокий является {\itshape также} наиболее поверхностным. Мы~могли бы~с~таким же~успехом относиться к~Королеве Елизавете как к~Соединенному Королевству или к~первому предложению (\hyperlink{par:1.1.1}{1.1.1}) как к~настоящему тексту.
	\end{itemize}

\paragraph{2.6.2}\hypertarget{par:2.6.2}{} Те, кто пытаются обладать тем, чего у~них нет (\hyperlink{par:1.5.1}{1.5.1}), быть там, где их~нет, и~редуцировать то, что не~редуцируется несчастливы, потому что они владеют потенцией только потенциально и~имеют теорию только в~теории.
	\begin{itemize}
	\item 
	Теперь мы~можем перейти к~морали более постоянного типа (\hyperlink{par:1.2.13}{1.2.13}). Мы~не будет выискивать истоки, редуцировать практики к~теориям, теории к~языкам, языки к~метаязыкам и~так далее так как это было описано в~Интерлюдии I. Мы~будем работать внутри узких сетей, которые не~могут быть редуцированы к~другим, имея привилегии или ответственности не~больше, чем кто-либо еще. Как и~все остальные мы~будет искать союзников и~благоприятных возможностей, и~когда-нибудь мы~их найдем. <<Это не~такая уж~далеко идущая мораль, не~так ли?>>. Именно так: она не~заведет нас очень далеко. Она отказывается идти мысленно в~места, где ее~нет. Когда она движется, она платит по~счетам. Мы~больше не~будем подражать Титану, и~нести мир на~наших плечах, подавленные бесконечной задачей понять, устроить, оправдать и~объяснить все.
	\end{itemize}

\paragraph{2.6.3}\hypertarget{par:2.6.3}{} Поскольку не~существует буквального или переносного смысла (\hyperlink{par:2.2.2}{2.2.2}), ни~одно употребление метафоры не~может доминировать над остальными употреблениями. Без правильности нет неправильности. Каждое слово точно и~обозначает именно те~сети, которые оно обнаруживает, выкапывает и~по которым путешествует. Мы~не должны бояться того, что один значение <<истинно>> а~другое <<метафорично>>. Между словами тоже существует демократия. Нам нужна эта свобода, чтобы победить потенцию.

\paragraph{2.6.4}\hypertarget{par:2.6.4}{} Как мы~назовем эту свободу переходить из~одних владений в~другие, это расширение сетей, это обследование? Имя этому ремеслу~--- философия и~старейшие традиции определяли философов как тех, у~кого нет особого поля, территории или области. 
Конечно, мы~можем обойтись и~без философии, и~без философов, но~тогда может и~не быть пути перебраться из~одной провинции в~соседнюю, из~одной сети в~другую.

\paragraph{2.6.5}\hypertarget{par:2.6.5}{} Существует только два способа обнаружения сил. Первое, мы~можем сказать, что с~одной стороны силы, а~с другой~--- {\itshape все прочее} ({\itshape other things}). Это равносильно отрицанию первого принципа (\hyperlink{par:1.1.1}{1.1.1}). Таким образом получаются <<реальные>> эквивалентности, <<реальные>> обмены, <<реальные>> сущности и~мир упорядочен начиная с~господ (государей/принцев, принципов, представителей, истоков, оснований, причин, капитала) нисходя к~тем, кто находится в~подчиненном положении (подразумеваемым, объясненным, дедуцированным, купленным, произведенным, оправданным, вызванным). Второе, мы~можем придерживать первого принципа до~конца. Если мы~так поступим, то~не будет больше никаких эквивалентностей, редукций, или властей (authorities) до~тех пор, пока не~будет выплачена надлежащая цена, а~работа господства не~станет публичной. 
	\begin{itemize}
	\item 
	Первый способ работы религиозный по~сути, монотеистический по~необходимости и~гегельянский по~методу. Он~питает отвращение к~магии, но, тем не~менее, подражает ее~методам. Второй способ работы оставляет локальным то, что локально и~деконструирует {\itshape потенцию}. Он~ведет к~скептицизму относительно всех видов магии, включая нашу собственную.
	\end{itemize}


%\subparagraph{Интерлюдия III: Избавляющая от~противоречия, которое по~мнению автора, возможно поставило читателя в~тупик.}
\chapter{Антропологики}

\paragraph{3.1.1}\hypertarget{par:3.1.1}{} Как обстоят дела? (How do~things stand?) О~каких актантах мы~говорим? Эти энтелехии, чего они хотят? Они сами борются за~то, что бы~ответить на~эти вопросы. Выбрать ответ, значит усилить одного и~ослабить другого. 
	\begin{itemize}
	\item 
	Каждый актант создает для себя целый мир (\hyperlink{par:1.2.8}{1.2.8}). Кто мы? Что мы~можем знать? На~что мы~можем надеяться? Ответы на~эти громкие вопросы определяют и~изменяют их~формы и~границы (\hyperlink{par:1.1.6}{1.1.6}).
	\end{itemize}

\paragraph{3.1.2}\hypertarget{par:3.1.2}{} Я~не~знаю, как обстоят дела. Я~не~знаю ни~кто я, ни~чего я~хочу, но~{\itshape другие} говорят от~моего имени, что знают, другие, которые определяют меня, связывают меня, заставляют меня говорить, интерпретируют то, что я~говорю, и~вербуют меня. Будь я~штормом, крысой, скалой, озером, львом, ребенком, рабочим, геном, рабом, бессознательным или вирусом они шепчут мне, они предлагают, они навязывают интерпретацию того, кто я, и~кем бы~мог быть.

%\subparagraph{Интерлюдия IV: Объясняющая, почему вещи-в-себе хорошо обходятся без какой-либо помощи с~нашей стороны.}

\paragraph{3.1.3}\hypertarget{par:3.1.3}{} Те, кто говорит всегда говорят о~тех, кто не~говорит сам. Они говорят о~нем, об~этом, о~нас, о~вас \ldots о~том, кто это, что оно хочет, когда случится нечто иное. Те, кто говорят, связаны с~теми, о~ком они говорят многими способами. Они действуют как делегаты, переводчики, аналитики, толкователи, предсказатели, наблюдатели, журналисты, гадалки, социологи, поэты, представители, родители, опекуны, пастыри, любовники. 
	\begin{itemize}
	\item 
	Гоббс говорил о~<<персоне>>, <<маске>> или <<актере>>, когда рассуждал о~тех, кто говорит от~имени безмолвных. Существует множество масок и~не~все из~них известны хранителям Музеев Антропологии.
	\end{itemize}

\paragraph{3.1.4}\hypertarget{par:3.1.4}{} Каждый актант решает, кто будет говорить и~когда. Есть те, кому он~позволяет говорить, те, от~имени которых он~говорит, те, к~кому он~обращается. Наконец, есть те, кого заставили молчать или те, кому позволяют общаться исключительно жестами или
симптомами. 
	\begin{itemize}
	\item 
	Энтелехии не~могут быть поделены на~<<одушевленные>> и~<<неодушевленные>>, <<людей>> и~<<не-человеков>>, <<объект>> и~<<субъект>>, поскольку это разделение является одним из~тех самых способов, с~помощью которого одна сила может соблазнить другие. Мы~можем заставить каменных истуканов ходить, отрицать наличие души у~черных, говорить от~имени китов, или заставить Полюса проголосовать. Акторов всегда {\itshape можно заставить сделать} нечто подобное, даже если то, что они бы~сделали или сказали, если бы~их предоставили самим себе, является тайной. (Возможно, они вообще не~были бы~<<черными>>, <<китами>>, <<Полюсами>> или <<истуканами>>.)
	\end{itemize}

\paragraph{3.1.5}\hypertarget{par:3.1.5}{} Сила практически всегда окружена властями (мощностями, powers)~--- голосами, которые говорят от~имени безмолвных толп (\hyperlink{par:1.5.0}{1.5.0}). Эти власти определяют, соблазняют,
используют, проектируют, перемещают, считают, инкорпорируют и~прерывают эту силу. Вскоре становится невозможным различить (\hyperlink{par:1.5.1}{1.5.1}) между тем, что говорит сама сила, что она говорит о~себе, что о~ней говорят власти, и~что представляемые этими властями толпы хотели бы, чтобы они сказали. 
	\begin{itemize}
	\item 
	Заключая союзы со~словами, текстами, бронзой, сталью, местами или эмоциями, мы~приходим к~различению форм, которые можно классифицировать, по~крайней мере в~мирное время. Но~эти классификации не~могут надолго устоять перед мародерством других акторов, которые раскладывают вещи иначе.
	\end{itemize}

\paragraph{3.1.6}\hypertarget{par:3.1.6}{} Все что угодно может быть редуцировано к~тишине и~все можно заставить говорить. Таким образом, любая сила может обратиться к~неистощимому источнику акторов, {\itshape от~лица которых можно говорить}.
	\begin{itemize}
	\item 
	Этнологи показали нам, как можно заставить говорить останки, свернувшееся молоко, дым, предков или ветер. Их~робость помешала им~увидеть, как другие гораздо ближе к~дому заставляют говорить окаменелости, химические осадки, промокательную бумагу, гены и~торнадо. Разумеется, психоаналитики говорят о~словоохотливом <<бессознательном>>, но~его репертуар обедняется и~комбинируется в~соответствии с~очень малым количеством правил. Вдобавок ко~всему, психоаналитики склонны говорить, что подсознательное имеет только <<субъективный>> смысл. Таким образом, все, что нам нужно сделать, это почитать <<Таймс>> и~увидеть как много акторов помимо бессознательного, которых заставляют говорить бесконечным количеством иных способов. Тут легионы ангелов мобилизованы для того, чтобы пресечь зло, там тысячи страниц распечатаны на~компьютере для того, чтобы остановить АЭС; на~следующей странице молчаливое большинство доведено до~крика от~имени нерожденных детей; несколькими страницами ранее мертвых возвернули к~жизни для того, чтобы остановить осквернение кладбища; на~последней полосе представители китов прервали смертельную миссию Японского судна.
	\end{itemize}

\paragraph{3.1.7}\hypertarget{par:3.1.7}{} По~определению не~существует {\itshape преданных} (верных) представителей (\hyperlink{par:2.2.1}{2.2.1}), поскольку они говорят то, что их~избиратели не~говорили и~говорят вместо них (\hyperlink{par:3.1.3}{3.1.3}). Всякая власть, поэтому, может быть {\itshape редуцирована к~своему простейшему выражению}. Все что необходимо, так это то, чтобы все акторы, от~имени которых говорит власть, высказались по-очереди. Тогда каждый из~акторов скажет сам, чего он~хочет без цензуры и~без подсказки. У~сил, которые редуцируют и~провоцируют друг друга, нет жалости. <<Ты говоришь от~их имени, но~что если я~поговорю с~ними сам, что они скажут мне?>>

\paragraph{3.1.8}\hypertarget{par:3.1.8}{} Существует только один способ, с~помощью которого актор может доказать свою власть. Он~должен заставить тех, от~чьего имени он~говорит, {\itshape говорить} и~показывать, что они говорят {\itshape одно и~тоже}. Как только это будет сделано, актор может сказать, что он~не говорит, но~четно <<выражает>> (канализирует, транслирует) взгляды других. 
	\begin{itemize}
	\item 
	Профсоюзы проводят демонстрации при помощи своих членов так же, как лаборатория Скиннера проводит демонстрации при помощи крыс. В~каждом из~случаев должно быть видно, что участники демонстрации и~крысы сами говорят то~же самое как и~тогда, когда их~побуждали говорить. А~что касается ангелов и~бесов, то~есть тысяча способов обнаружить их~знаки~--- очевидцы, стигматы или чудеса~--- которые смягчат  ожесточенное сердце.
	\end{itemize}

%\paragraph{3.1.9}\hypertarget{par:3.1.9}{}

\paragraph{3.1.10}\hypertarget{par:3.1.10}{} Поскольку представитель всегда говорит {\itshape нечто другое}, чем те, которых он~побуждает говорить, и~поскольку всегда существует необходимость вести переговоры относительно сходства и~отличия (\hyperlink{par:1.2.1}{1.2.1}), постольку {\itshape всегда есть место} для спора о~правильности всякой интерпретации. Сила всегда может втереться между спикером и~теми, кого он~заставляет говорить. Она всегда может заставить их~сказать что-нибудь еще. 
	\begin{itemize}
	\item 
	Участники демонстрации не~говорили, что они хотят сорокачасовую рабочую неделю~--- просто их~пришло тысячи; крысы не~говорили, что у~них есть условные рефлексы~--- они просто цепенели от~ударов током. Следовательно, другие могут вмешаться. Присутствие рабочих можно перевести, сказав, что <<им заплатили профсоюзы>>, а~неподвижность крыс можно интерпретировать как <<экспериментальный артефакт>>.
	\end{itemize}

\paragraph{3.1.11}\hypertarget{par:3.1.11}{} У~этих споров нет {\itshape естественного завершения}. Они всегда могут возобновиться (\hyperlink{par:3.1.6}{3.1.6}). Единственный способ прекратить эти споры, это не~дать другим актантам сбить с~пути тех, кто был завербован, и~превратить их~в предателей. В~конечном счете, интерпретации всегда стабилизируются посредством массива {\itshape сил}.

\paragraph{3.1.12}\hypertarget{par:3.1.12}{} Сила становится могущественной, только если она может {\itshape говорить за} других; если она может заставить тех, кого она заглушила, {\itshape говорить} тогда, когда она вынуждена продемонстрировать свою мощь; и~если она может вынудить тех, кто бросил ей~вызов, {\itshape признать}, что она действительно говорит то, что сказали бы~ее союзники. 
	\begin{itemize}
	\item 
	Профсоюзы не~могут запретить своим противникам, правым интерпретировать демонстрации по-другому. Скиннер не~может воспрепятствовать тому, что его <<дорогие коллеги>> интерпретировали его эксперимент иным способами. Если бы~они могли, они, конечно же, сделали бы~это, но~поскольку все так, как есть, они не~могут. Другие бы~уничтожили их, если бы~они попытались.
	\end{itemize}

\paragraph{3.2.1}\hypertarget{par:3.2.1}{} Каково положение дел? Что будет дальше? Каково соотношение сил? Используя множества, которые они заставляют говорить, некоторые актанты становятся достаточно влиятельными для того, чтобы определять, на~недолгий срок и~локально, для чего все это. Они делят актантов, сортируют их~по ассоциациям, называют сущности, наделяют эти сущности волей или функцией, направляют эти воли или функции к~цели, принимают решение о~том, как определить, что эти цели были достигнуты и~так далее. Мало-помалу они все увязывают вместе. Все придает силы энтелехии, у~которой нет сил, и~целое становится <<логичным>> и~<<устойчивым>>~--- другими словами {\itshape сильным} (\hyperlink{par:2.1.8}{2.1.8}).
	\begin{itemize}
	\item 
 Я~не~пытаюсь избежать ответа на~вопрос: <<Каков баланс сил?>> Однако мы~должны расчистить почву для того, чтобы {\itshape все} ответы смогли проявиться.
	\end{itemize}

\paragraph{3.2.2}\hypertarget{par:3.2.2}{} Ни~один из~актантов, мобилизованных для закрепления союза, не~прекращает действовать в~своих интересах (\hyperlink{par:1.3.1}{1.3.1}, \hyperlink{par:1.3.4}{1.3.4}). Все они продолжают плести свои интриги, создавать свои группы, служить другим господам (хозяевам), волям и~функциям. 
	\begin{itemize}
	\item 
	Силы всегда склонны к~восстанию (\hyperlink{par:1.1.1}{1.1.1}); они одалживают себя, но~не отдают (\hyperlink{par:1.5.1}{1.5.1}). Это верно для дерева, которое прорастает снова; для саранчи, которая уничтожает урожай; для рака, который поражает других их~же собственным оружием, для мулл, которые разрушили Персидскую империю; для сионистов, которые ослабили власть мулл; для треснувшего бетона электростанции; для голубых акриловых красок, которые поглощают другие пигменты, для льва, который не~следует пророчествам оракула~--- все они имеют иные цели и~судьбы, которые не~могут быть {\itshape суммированы}. В~тот момент, когда мы~поворачиваемся спиной, наши ближайшие друзья встают под другие знамена.
	\end{itemize}

\paragraph{3.2.3}\hypertarget{par:3.2.3}{} Каким образом можно удержать от~разговоров тех, от~чьего имени мы~говорим? Каким образом те, кто был рекрутирован благодаря счастливой случайности, могут быть сцементированы в~единый блок? Как можно усмирить мятежников и~диссидентов? Есть ли~где-нибудь хоть {\itshape одна} сущность, которая не~должна решать эти проблемы? Ответ всегда один и~тот же, поскольку существует только один источник силы: тот, что проистекает из~объединения (\hyperlink{par:1.3.2}{1.3.2}). Но~как могут быть ассоциированы мятежники? [Этого можно добиться] {\itshape Путем поиска дополнительных союзников}, которые заставят других держаться вместе, и~так далее, до~тех пор, пока градиент неопределенных объектов не~подойдет к~концу и~не~сделает авангард альянса способным к~сопротивлению и~таким образом реальным (\hyperlink{par:1.1.2}{1.1.2}).
	\begin{itemize}
	\item 
	Понятие системы бесполезно для нас, поскольку система это конечный продукт наладки, а~не~исходная точка (\hyperlink{par:2.1.4}{2.1.4}). Для того, чтобы системы существовали, сущности должны быть четко определены, тогда как на~практике никогда не~бывает так; функции должны быть ясны, тогда как большинство акторов не~определились, хотят ли~они командовать или подчиняться; обмен эквивалентами между сущностями и~подсистемами должен быть оговорен, тогда как повсюду ведутся споры о~пропорции и~направлении обмена. Системы не~существуют, но~систематизация довольно распространена; повсюду есть силы, которые обязывают других играть так, как они играли всегда (\hyperlink{par:1.1.13}{1.1.13}). 
	\end{itemize}

\paragraph{3.2.4}\hypertarget{par:3.2.4}{} У~каждого актора, когда он~ассоциирует элементы, есть выбор: расширяться дальше, не~боясь (risking) диссидентства и~диссоциации, или усилить согласованность и~прочность, но~не идти слишком далеко. 

\paragraph{3.2.5}\hypertarget{par:3.2.5}{} Вполне определенное положение дел~--- это работа {\itshape многих сил}. Они не~согласны ни~в чем и~ассоциируются только посредством длинных сетей, в~которых они непрестанно говорят, не~будучи в~состоянии резюмировать друг друга. Они смешиваются, но~они не~могут выйти за~собственные пределы, чтобы понять то, что связывает их, противостоит им, и~резюмирует их. Однако, несмотря ни~на что, сети усиливают друг друга и~сопротивляются разрушению. Надежные и~все же~хрупкие, изолированные и~все же~переплетенные, ровные и~все же~скрученные вместе энтелехии их~странных фабрик. Вот как мы~представляли себе <<традиционные миры>>, как бы~далеко назад мы~ни смотрели. 
	\begin{itemize}
	\item 
 Я~не~говорю о~<<культуре>>, потому что это слово было припасено Западом для описания одной из~обособленных сущностей, конституирующей <<человека>>. Силы не~могут быть поделены на~<<человеческие>> и~<<нечеловеческие>>. Я~не~говорю об~<<обществе>>, поскольку ассоциации, которые меня интересуют, не~ограничены теми немногими, которые допускает понятие <<социального>>. Кроме того, я~не~говорю о~<<природе>>, потому что тех, кто говорит от~имени групп крови, хромосом, водяных испарений, тектонических плит или рыб можно лишь временно и~локально отличить от~тех, кто от~имени крови, мертвых, реки, ада, и~рыбы. Я~бы~согласился на~слово <<бессознательное>>, если бы~мы были достаточно восприимчивыми для того, чтобы обозначать им~вещи-в-себе.
	\end{itemize}

\paragraph{3.3.1}\hypertarget{par:3.3.1}{} Для того, чтобы распространятся далеко, не~теряя связности, актанту нужны верные союзники, которые принимают то, что им~говорят, идентифицируют себя со~своей причиной, выполняют все функции, которые им~определены, и~без колебаний приходят на~помощь к~актанту, когда их~зовут. Поиск таких идеальных союзников занимает пространство и~время тех, кто хочет быть сильнее, чем другие. Как только актор нашел {\itshape в~известной степени более верного союзника}, он~может заставить другого союзника в~свою очередь быть {\itshape более верным}. Он~создает градиент, который обязывает других союзников принять форму и~сохранять ее~до поры до~времени (\hyperlink{par:1.1.12}{1.1.12}).
	\begin{itemize}
	\item 
 Мы~потратили много времени на~поиски чего-либо оказывающегося жестче для того, чтобы придать форму тому, что мягче~--- камня, который служит наковальней, биопробы измеряющей уровень эндорфинов в~крови, языка коровы, позволяющего вирусу проникнуть в~костный мозг, закона, сдерживающего аппетит лоббистов, лоббистов для того, чтобы изменить закон. Слово <<технология>> неудовлетворительно, потому что оно слишком долго было ограничено исследованием тех силовых линий, которые принимали форму гаек и~винтов.
	\end{itemize}

\paragraph{3.3.2}\hypertarget{par:3.3.2}{} Если мы~хотим не~дать силам трансформироваться, в~тот момент, когда мы~поворачиваемся спиной, мы~должны избегать того, чтобы поворачиваться спиной! Власти всегда мечтают о~том, чтобы быть повсюду, даже если они далеко или их~уже давно нет. Как они могут присутствовать, когда другие силы вытеснили их~(\hyperlink{par:1.2.5}{1.2.5})? Как они могут расширяться, когда все локализует их? Как они могут быть там и~где-либо еще, сейчас и~навсегда? О, потенция мифа о~потенции! Все, что в~[данный момент] помогает данной структуре сохраняться после момента ухода силы, будет помогать [и далее].

\paragraph{3.3.3}\hypertarget{par:3.3.3}{} Когда сила нашла союзников, которые позволяют ей~надежным способом скрепить ряды других сил, она может вновь расширяться. Именно потому, что верные [союзники] скреплены такими прочными связями, сила может уходить без страха. Даже, когда ее~там нет, все произойдет так, как если бы~она там была. В~конечном счете, это просто собрание сил, которые работают на~нее, но~без нее.
	\begin{itemize}
	\item 
	Иногда мы~называем эти махинации сил <<механизмами>>. Это неудачно выбранный термин~--- потому что он~подразумевает, что все силы механические, тогда как большинство из~них таковыми не~является; потому что он~придает особое значение несущим конструкциям (hardware) за~счет более мягких отношений; и~потому что он~предполагает, что они сделаны человеком и~искусственны, несмотря на~то, что их~генеалогия является именно тем, что поставлено на~карту.
	\end{itemize}

\paragraph{3.3.3.1}\hypertarget{par:3.3.3.1}{} Обретение могущества (potency) всегда является делом противопоставления сил друг другу. Власть, которая проистекает из~всего массива, затем приписывается {\itshape последней} силе, захваченной всеми остальными. 
	\begin{itemize}
	\item 
	Причина, по~которой я~говорил о~силе с~самого начала должна быть теперь ясна. Это была не~попытка {\itshape распространить} технические метафоры на~философию. Напротив, сила машин и~автоматизмов лишь редко и~локально достигает своей цели. Только тогда, когда мы~игнорируем все остальные силы, благодаря которым они оказываются {\itshape последними в~цепи}, мы~можем говорить о~<<технике>>. Мотор, урчащий под кожухом это лишь одна из~возможных форм, принимаемых заговором сил. Дизель надеялся оптимизировать производительность социальных организаций так же, как он~это сделал для двигателя внутреннего сгорания. Должен был быть такой же~мотор, такое же~исследование, такая же~оптимизация: давление, смесь, регенерация, производительность.
	\end{itemize}

\paragraph{3.3.3.2}\hypertarget{par:3.3.3.2}{} В~этих махинациях нет ничего особенного, не~считая этого макиавеллевского предписания: собрать как можно большее количество союзников {\itshape внутри} и~оставить тех, в~ком мы~сомневаемся {\itshape снаружи}. Благодаря этому мы~имеет новое разделение между жестким и~мягким.
	\begin{itemize}
	\item 
	Те, кто захвачен этим разделение говорит о~<<техническом>> и~<<социальном>>, не~осознавая того, что <<социальное>> может быть оставлено в~стороне как стружка из-под рубанка плотника. Каждый проект может быть прочитан как еще один <<Государь>>: скажи мне, каковы твои допустимые отклонения, исходные пункты, твои калибровки, патенты, от~которых ты~уклонился и~уравнения, которые ты~выбрал, и~я скажу тебе, кого ты~боишься, на~чью поддержку ты~надеешься, кого ты~решил избегать или игнорировать, и~над кем ты~желаешь господствовать\footnote{{\itshape Mick{\`e}s Coutouzis}. Soci{\'e}t{\'e}s et~techniques en~voie de~d{\'e}placement: Ie~transfert d'un village solaire des Etats Unis en~Cr\^ete. (Paris: Universite Dauphine, 1983).}.
	\end{itemize}

\paragraph{3.3.4}\hypertarget{par:3.3.4}{} И~все же~вы не~можете удержать силы от~того, чтобы играть друг против друга. Нет заговора, колдовства, логики, аргумента или машины, которая бы~уберегла мобилизованных актантов от~того, чтобы кипеть и~пениться, когда находятся в~поисках других целей и~альянсов. Самая безличная машина переполнена в~большей степени, чем пруд с~рыбой.
	\begin{itemize}
	\item 
	Вопреки Лейбницу, в~ходе часов также есть пруды полные рыбы и~рыбы полные прудов. Конечно, всегда возможно найти людей, которые скажут, что машины холодны, безличны, бесчеловечны или стерильны. Но~посмотрите на~чистейший сплав: его также предают повсюду, как и~все остальные наши альянсы. Запад всегда верил в~то, что моторы <<чисты>> также как аргументы <<логичны>>, а~слова <<буквальны>>. Вот что капитан сказал Крузо прямо перед самым кораблекрушением: <<Остерегайся чистоты. Это язва (vitriol) души>> (Интерлюдия IV).
	\end{itemize}

\paragraph{3.3.5}\hypertarget{par:3.3.5}{} Для того, чтобы существовать самому, актант должен программировать других актантов таким образом, чтобы они не~смогли его предать (\hyperlink{par:3.3.3}{3.3.3}), несмотря на~тот факт, что они близки к~этому (\hyperlink{par:3.3.4}{3.3.4}). Есть только один способ разрешить это затруднительное положение: так как ни~одна индивидуальная связь не~является надежной, то~актанты должны поддерживать друг друга; в~тот момент, когда множество связей выстроено ярусами, они становятся реальностью.
	\begin{itemize}
	\item 
	Так как не~существует ничего кроме слабостей, то~власть это всегда впечатление. Однако данное впечатление это все что требуется дл~того, чтобы изменить форму вещей, ин{\itshape{формируя}} или в{\itshape{печатляя}} их. Это то~самое чудо, которое следует объяснить.
	\end{itemize}

\paragraph{3.3.6}\hypertarget{par:3.3.6}{} {\itshape Мы~всегда неправильно истолковываем силу сильного.} Хотя люди приписывают ее~безупречности актанта, она существует неизменно благодаря иерархизированному массиву слабостей.

%\subparagraph Интерлюдия V: Где мы~с большим удивлением узнаем, что не~существует такой вещи как современный мир.

\paragraph{3.4.1}\hypertarget{par:3.4.1}{} Как нам следует говорить обо всем том, что сохраняет единство? Должны ли~мы говорить об~экономике, законе, механизмах, языковых играх, обществе, природе, психологии или о~системе, которая охватывает их~всех?
	\begin{itemize}
	\item 
 В~фильмах про Джеймса Бонда всегда есть одна черная кнопка, которая может свести на~нет все махинации злого гения, кнопка, которую герою, переодетому техником удается нажать в~конце. Замаскированный и~переодетый в~белый халат, философ достигает точки, где совпадают абсолютное могущество и~абсолютная хрупкость.
	\end{itemize}

\paragraph{3.4.2}\hypertarget{par:3.4.2}{} Это не~предмет {\itshape экономики}. Она использует эквиваленты, не~зная того, кто установил эквиваленты, и~бухгалтерию, не~зная того, кто измеряет и~считает. Экономика ступает в~дело {\itshape после} того, как инструменты измерения были расставлены по~местам~--- инструменты, которые делают возможным определять цены и~вступать в~обмен. Будучи далекой от~того, чтобы освещать испытания сил, экономика маскирует и~подавляет их. В~лучшем случае она является способом регистрировать эти испытания, как только они стабилизируются.
	\begin{itemize}
	\item 
	Стоит установить инструмент измерения, и~мы~можем вести экономику и~считать, экономить и~копить. Другими словами, мы~можем убеждать и~обогащаться. Но~экономисты не~говорят о~том, каким образом эти инструменты были установлены в~первую очередь.
	\end{itemize}

\paragraph{3.4.2.1}\hypertarget{par:3.4.2.1}{} Всеобщая экономия~--- подсчет удовольствия, генов или прибыли~--- невозможна. Необходимо бы~было раскрыть тех, кто ведет переговоры, тех, кто платит, тех, кто выиграл и~проиграл, во~что обошлись вознаграждения и~когда следует закрыть счет. 

\paragraph{3.4.3}\hypertarget{par:3.4.3}{} Это не~предмет {\itshape права}. Оно представляет собой храповый механизм, который, как и~любой другой (\hyperlink{par:1.1.10}{1.1.10}), позволяет актанту сделать временное занятие позиции необратимым. Закон делают сильным не~только тексты, но~также паралич тех, кто не~осмеливается пересечь то, что по~их убеждению <<потенциально>> заложено в~его скрижалях (священных писаниях); то~есть разрыв между законом и~силой или законом и~фактом. Если мы~обладает такой властью, то~мы можем устрашить других и~распространить себя в~новые места, невзирая на~противодействие. Сила закона проистекает не~из него самого, но~от нищей презренной толпы, дающей ему силу факта: нравов, слов, дубинок полицейских, надежд, администраций, стен, телексов, файлов, финансов, зол.

\paragraph{3.4.4}\hypertarget{par:3.4.4}{} Это не~вопрос {\itshape машин} или {\itshape механизмов}. Они никогда не~существовали без механики, изобретателей, финансистов и~машинистов. Машины это тайные объекты желания актантов, которые настолько эффективно приручили силы, что они более не~выглядят как
силы. Результат состоит в~том, что актанты подчиняются даже тогда, когда сил там нет (\hyperlink{par:3.3.3}{3.3.3}).
Многие люди грезили о~таких машинах, которые можно было бы~распространить на~все виды отношений, но~этот сон всегда преследует ночной кошмар: восстание саботирующих актантов, которые расставляют ловушки для самый гладко работающих машин. Сила машин черпается из~других сил, которые становятся их~частью~--- сил, которых другие презирают и~подавляют; сил, которые представляют собой разобщенную плебейскую толпу из~низших классов.

\paragraph{3.4.5}\hypertarget{par:3.4.5}{} это не~вопрос {\itshape языка} или языковых игр (\hyperlink{par:2.3.0}{2.3.0}, \hyperlink{par:2.4.3}{2.4.3}, \hyperlink{par:2.4.4}{2.4.4}). Слова не~обладают властью, но~заимствуют свою силу у~компромиссов, которые далеки от~<<изящной словесности>> (художественной литературы).

\paragraph{3.4.6}\hypertarget{par:3.4.6}{} Это не~предмет {\itshape науки}. Если бы~аргументы были суверенными, они бы~обладали всем могуществом (potency) подагрического монарха, заточенного в~разрушающемся замке. Если наука и~развивается, то~это потому, что ей~удалось убедить множество актантов сомнительного происхождения одолжить ей~их силу: крыс, бактерий, промышленников, мифов, газ, червей, специальных сплавов, страстей, учебников, мастерских~\ldots толпу невежд, чья помощь отрицается даже в~то~время, когда она используется. 
	\begin{itemize}
	\item 
	Высшая школа фактов нередко также является и~школой высокомерия (самонадеянности). Просвещение ведет к~наиболее грубой форме обскурантизма.
	\end{itemize}

\paragraph{3.4.7}\hypertarget{par:3.4.7}{} Это не~вопрос {\itshape общества}. Значение понятия <<социальное>> постоянно сжимается~--- теперь оно сведено до~уровня <<социальных>> проблем. Это то, что остается, когда остальное было поделено властьимущими (powerful); все, что не~является ни~экономическим, ни~техническим, ни~правовым, ни~каким-либо еще оставляется социальному. Неужели мы~действительно надеемся связать все вместе с~помощью такой обедненной версии социального? Как майонез, который никто не~берет, оно грозит прокиснуть. <<Социальное>>~--- его акторы, группы, стратегии~--- слишком тесно идентифицируется с~человеческими существами для того, чтобы обратить внимание на~беспомощную нечистоту и~аморальность альянсов.
	\begin{itemize}
	\item 
	Если бы~{\itshape социология} (как подсказывает ее~имя) была наукой об~ассоциациях, а~не~наукой о~социальном, к~которой она была редуцирована в~девятнадцатом веке, тогда, возможно, мы~были бы~счастливы назвать себя <<социологами>>.
	\end{itemize}

\paragraph{3.4.8}\hypertarget{par:3.4.8}{} Это не~вопрос {\itshape интерсубъективных отношений}. Только в~наши дни мы~можем надеяться найти людей убогих настолько, чтобы пытаться объяснить ядерные реакторы, национальные государства или биржевые сделки на~основе <<интеракций>>. Психология и~ее~брат психоанализ, полагают себя богатыми в~их~безграничной нищете. Об~этой точке зрения не~следует говорить ничего, кроме того, что ее~не следует придерживаться. Уклончивый (who shrinks) психоаналитик (shrink) не~может распространяться по~поводу того, чтобы объяснить остальных (\hyperlink{par:2.5.6.2}{2.5.6.2}).
	\begin{itemize}
	\item 
 В~глубинке всегда находились пристанища для тех, кто хочет строить соборы из~спичек или шариковых ручек.
	\end{itemize}

\paragraph{3.4.9}\hypertarget{par:3.4.9}{} Это не~проблема {\itshape природы} (\hyperlink{par:3.2.5}{3.2.5}). Попробуйте осмыслить эти серии: пятна на~солнце, тальвеги\footnote{Тальвег (нем. {\itshape Talweg}, от~{\itshape Tal} <<долина>> и~{\itshape Weg} <<дорога>>)~--- линия, соединяющая наиболее пониженные участки дна реки, долины, балки, оврага и~др. вытянутых форм рельефа. Тальвег в~плане обычно представляет собой относительно прямую или извилистую линию. В~более широком смысле тальвег~--- дно долины~--- {\itshape Прим. перев.}}, антитела, спектральный анализ углерода; рыба, подстриженные живые изгороди, пустынные пейзажи; <<le petitshape pan de~mur jaune>>, горные ландшафты нарисованные китайской тушью, лес трансептов\footnote{Трансепт (от позднелатинского {\itshape transeptum} из~лат. {\itshape trans} <<за>> и~лат. {\itshape septum} <<ограда>>)~--- поперечный неф в~базиликальных и~крестообразных в~плане храмах, пересекающий под прямым углом основной (продольный) неф.~--- {\itshape Прим. перев.}}; львы, которые ночью превращаются в~людей, богини матери из~слоновой кости, тотемы из~черного дерева.

Видите? Мы~не можем редуцировать количество или гетерогенность альянсов таким способом. {\itshape Природы} смешиваются друг с~другом и~с <<нами>> так основательно, что мы~не можем надеяться отделить их~и открыть чистые, уникальные истоки их~власти (Интерлюдия IV).

\paragraph{3.4.10}\hypertarget{par:3.4.10}{} Это не~проблема {\itshape систем} (\hyperlink{par:3.2.3}{3.2.3}). Поскольку люди знают, что исток власти не~покоится в~чистоте сил, они помещают его в~<<систему>> чистых сил. Это мечта постоянно возрождается. Право привязывается (is attached) к~экономике, биологии, языку, обществу, кибернетике\ldots Нарисованы красивые ящички, дополненные красиво расставленными стрелочками. К~несчастью для тех, кто делает системы, акторы остаются неподвижными достаточно долго, чтобы сделать групповое фото; ящики переполняются; стрелки изгибаются и~разрываются; право просачивается в~биологию, которая растворяется в~обществе. Нет, альянсы выковываются не~в отношениях {\itshape между} приятными и~абстрактными сторонами, а~в беспорядочном и~разнородном конфликте, который ужасен для тех, кто почитает чистоту.

\paragraph{3.5.1}\hypertarget{par:3.5.1}{} Мы~всегда неправильно истолковываем действенность сил: мы~приписываем им~то, что им~лишь одолжили (\hyperlink{par:1.5.1}{1.5.1}). Мы~считаем их~чистыми, тогда как они были бы~совершенно немощны, если бы~это случилось. Когда мы~смотрим на~то, как они работают, мы~хлам, который не~может быть подытожен (приведен к~единому знаменателю). Всякая сеть разрежена, пуста, хрупка и~гетерогенна. Она становится сильной только, если она расширяется и~выстраивает в~боевой порядок слабых союзников. 
	\begin{itemize}
	\item 
 С~чем мы~можем сравнить слабости, которые создают силу? С~макраме. Есть ли~такой узел, который связывает человека с~человеком, нейроны с~нейронами, лист железа с~листом железа? Нет. Веревка для этого Гордиева узла еще не~была свита. Но~каждый день мы~видим собственными глазами макраме из~нитей разных цветов, материалов, происхождений, длин, к~которым прикреплены наиболее ценимые нами блага.
	\end{itemize}

\paragraph{3.5.2}\hypertarget{par:3.5.2}{} Можем ли~мы все сети описать одним и~тем же~способом? Да, потому что <<мира Нового Времени>> нет.
Годами этнографы говорили, что невозможно изучать <<примитивные>> или древние народы, если мы~разделяем право, экономику, религию, и~все остальное. Напротив они доказывали, что эти слабо связанные смешения могут быть поняты, только если мы~пристально посмотрим на~места, семьи, обстоятельства и~сети. Но~когда они говорят о~своих собственных странах, он~прибегают к~разделению сфер и~уровней.
 

\paragraph{3.5.3}\hypertarget{par:3.5.3}{} <<Мир Нового Времени>> это наклейка на~кнопке, которая объединяет в~себе абсолютное могущество и~абсолютную немощь (\hyperlink{par:3.4.1}{3.4.1}). Гетерогенное и~локальное применение слабостей становится системой властей с~такими престижными именами как природа, экономика, право и~техника.
	\begin{itemize}
	\item 
	Как те, кто ненавидит мир Нового Времени, так и~его фанатичные приверженцы, изобрели для его описания больше терминов, чем благочестивец нашел для прославления имя Господа. На~каждый из~этих призывов они отвечают либо <<Изыди, сатана>>, либо <<Услышь мои молитвы>>:

		\begin{verse}
		мир Нового Времени \\
		секуляризация  \\
		рационализация \\
		анонимность \\
		разочарование \\
		меркантилизм \\
		оптимизация \\
		дегуманизация \\
		механизация \\
		вестернизация \\
		капитализм \\
		индустриализация \\
		постиндустриализация \\
		технологизация \\
		интеллектуализация \\
		стерилизация \\
		объективизация \\
		американизация \\
		сциентизация \\
		общество потребления \\
		одномерное общество \\
		бездушное общество \\
		современное безумие \\
		современность \\
		прогресс
		\end{verse}

	<<Услышь мою молитву>>. <<Изыди, Сатана>>. Каждый из~этих лозунгов скрывают работу, проводимую силами, и~делают невозможным антропологию того, что существует здесь и~сейчас. Хотя это действительно очень просто: не~существует модерного мира, а~если и~существует, то~это просто стиль, как когда мы~говорим <<стиль модерн>>.
	\end{itemize}


%\subparagraph{Интерлюдия VI: В~которой автор, теряя самообладание, утверждает, что те, кто осуществляют редукцию~--- предатели.}

\paragraph{3.5.4}\hypertarget{par:3.5.4}{} К~счастью мир больше не~расколдован как это было раньше, машины больше не~полируются, мышление не~является больше точным, а~обмены не~организованы лучше.
Как мы~можем говорить о~<<модерном мире>>, когда его действенность зависит от~идолов: денег, закона, разума, природы, машин, организаций или лингвистических структур? Мы~уже использовали слово <<магия>> (\hyperlink{par:2.1.11}{2.1.11}). Поскольку истоки власти <<модерного мира>>истолковываются неправильно, и~действенность приписывается вещам, которые ни~двигаются, ни~говорят, мы~еще раз можем сказать о~магии (\hyperlink{par:4.1.0}{4.1.0}).

\paragraph{3.5.5}\hypertarget{par:3.5.5}{} То, что мы~с удовольствием называем <<другими культурами>> имеет множество секретов; наша же~возможно имеет только один. Вот почему <<другие культуры>> представляются нам таинственными и~стоят того, чтобы их~познавал, в~то~время как наша~--- выглядит равным образом непознаваемой и~лишенной загадки. Секрет, который является тем единственным, что отличает нашу культуру от~остальных, состоит в~том: что она и~только она одна не~является одной культурой из~многих (рядовой культурой, такой же~культурой, как и~многие другие). Наша вера в~модерный мир возникает из~этого отрицания. Для того чтобы избавиться от~нее, мы~все должны соединить вместе то, что мы~обычно разделяем, когда говорим о~сами себе. Мы~должны стать антропологами нашего собственного мира.

\paragraph{3.6.1}\hypertarget{par:3.6.1}{} О~чем все это? Каково положение дел? Некто говорит от~имени тех, кто не~говорит ничего, и~отказывает моим просьбам, определяя меня к~немым. Если его отказ убедит меня, я~больше не~смогу разобраться почему [он отказал], поскольку он~заставил слишком
много помощников поддержать его.

\paragraph{3.6.2}\hypertarget{par:3.6.2}{} Все происходит так, как будто нет никаких испытаний сил, но~только странная фантазия: <<человек>> <<открывающий>> <<природу>>!

\paragraph{3.6.3}\hypertarget{par:3.6.3}{} Только в~политике люди готовы говорить об~<<испытаниях сил>>. Политики~--- это козлы отпущения, жертвенные агнцы. Мы~высмеиваем, презираем и~ненавидим их. Мы~соревнуемся в~суждении их~продажности и~некомпетентности, узколобости, их~схем и~компромиссов, их~ошибок, их~прагматизма или достатка реализма, их~демагогии. Считается, что только в~политике испытания сил определяют форму вещей (\hyperlink{par:1.1.4}{1.1.4}). Только политиков считают бесчестными, блуждающими в~темноте (who are held to~grope in~the dark). 
	\begin{itemize}
	\item 
	Потребуется нечто подобное отваге для того, чтобы предположить, что мы~{\itshape никогда не~сможем быть лучше}, чем политик (\hyperlink{par:1.2.1}{1.2.1}). Мы~противопоставляем его некомпетентность компетенции того, кто хорошо осведомлен, строгости ученого, ясновидению провидца, озарению гения, незаинтересованности профессионала, мастерству ремесленника, вкусу артиста, прочному здравому смыслу простого человека с~улицы, чутью индейца, проворству ковбоя, который стреляет быстрее собственной тени, виду и~самообладанию самодовольного интеллектуала. И, тем не~менее, никто не~может быть лучше, чем политик. Тем другим просто есть, где спрятаться, когда они делают свои ошибки. Они могут возвратиться и~попробовать снова. Только политик ограничен одним выстрелом и~обязан стрелять на~публике. Я~сомневаюсь, что кто-либо способен на~то, чтобы думать более точно или видеть хоть сколько-нибудь дальше, чем самый близорукий конгрессмен (\hyperlink{par:2.1.0}{2.1.0}, \hyperlink{par:4.2.0}{4.2.0}).
	\end{itemize}

\paragraph{3.6.3.1}\hypertarget{par:3.6.3.1}{} То, что мы~презираем как политическую <<посредственность>> есть набор компромиссов, к~которым мы~от нашего имени принуждаем прийти политиков.
	\begin{itemize}
	\item 
	Если мы~презираем политиков, мы~должны презирать самих себя. Пеги ошибался. Он~должен был сказать: <<Все начинается политикой, но, увы, вырождается в~мистику>>.
	\end{itemize}

\paragraph{3.6.4}\hypertarget{par:3.6.4}{} Некто, задыхаясь, говорит другим, которые понимают только то, что хотят услышать. Это история о~тех, кто обнаруживает себя посредством загадок и~симптомов. Время от~времени те~о ком говорят, вмешиваются в~разговор, взбешенные из-за того, что их~предали. Иногда те, кто ведут разговор, останавливаются, будучи злыми из-за того, что они не~понимают или их~не поняли. Колеблясь, участники разговора ощупью продвигаются от~полумеры к~компромиссу. Они собирают силы, которые проверяют методом проб и~ошибок и~объединяют их~во временные союзы. Когда результат и~по~нраву, они связывают свою судьбу с~более прочными материалами. Мало по~мало силы
растут, от~объединений к~соглашению, от~одного недопонимания к~другому, до~того момента, когда другие более многочисленные и~умелые не~сокрушат их.
	\begin{itemize}
	\item 
	Макиавелли и~Спиноза, которых обвиняют в~политическом <<цинизме>>, были самыми великодушными из~людей. У~тех, кто убежден, что может достичь чего-то лучшего, чем плохо переведенный компромисс между слабо связанными силами, всегда получается еще {\itshape хуже}.
	\end{itemize}

\paragraph{3.6.5}\hypertarget{par:3.6.5}{} Хотя это может прозвучать странно, наша связь с~большинством сил, от~лица которых мы~говорим, является, вероятно, не~более тесной, чем связь профсоюзного деятеля с~рабочими, которых он~представляет, или управляющего директора с~его акционерами. Я~говорю здесь о~наших грезах, в~той же~мере в~какой он~наших крысах,
желудках или машинах. 
	\begin{itemize}
	\item 
 В~конечном счете политика является приемлемой моделью до~тех пор, пока она распространяется на~политику вещей-в-себе (\hyperlink{par:4.5.0}{4.5.0}).
	\end{itemize}

\paragraph{3.6.6}\hypertarget{par:3.6.6}{} Миры, вероятно, больше походят на~Рим, нежели на~компьютер. Или скорее так, превосходно задуманный компьютер следует коллажом из~переставленных и~заново использованных руин, великолепным Римским беспорядком\footnote{{\itshape Tracy Kidder}. The Soul of~a New Machine (London: Allen Lane, 1981).}. Каждая энтелехия подобна двору Пармы.
	\begin{itemize}
	\item 
	Бальзак сказал о~<<Пармской обители>> Стендаля, что она была <<Государем>> девятнадцатого века. Ни~секреты сердца, ни~секреты двора не~являются грандиозными~--- ни~грандиозными, ни~жалкими, но~нередуцируемыми, смещаемыми и~предаваемыми.	
	\end{itemize}	
\chapter{Нередуцируемость <<Наук>>}


\paragraph{4.1.1}\hypertarget{par:4.1.1}{} Вы~можете стать сильным только посредством ассоциации. Но~так это всегда достигается посредством перевода (\hyperlink{par:1.3.2}{1.3.2}), сила (\hyperlink{par:1.5.1}{1.5.1}, \hyperlink{par:2.5.2}{2.5.2}) приписывается могуществу, а~не~союзникам, благодаря которым вещи сохраняют устойчивость (holding things together) (\hyperlink{par:3.3.6}{3.3.6}). <<Магия>> --- это предложение могущества немощным (powerless). <<У них есть глаза, но~они не~видят, есть уши, но~они не~слышат{\ldots}>>
	\begin{itemize}
	\item 
 Я~уже говорил о~<<магии>>. Сначала я~использовал это слово для того, чтобы опровергнуть тех, кто убеждены, что они думают (\hyperlink{par:2.5.3}{2.5.3}), а~затем, чтобы одинаковым образом рассмотреть все виды логики (\hyperlink{par:2.1.11}{2.1.11}). Я~еще раз использовал его для того, чтобы создать эффект симметрии между <<примитивными культурами>> и~<<модерным миром>> (\hyperlink{par:3.5.4}{3.5.4}). Сейчас я~хочу использовать его для описания {\itshape всех} ошибок относительно истоков силы, {\itshape всякого} могущества.
	\end{itemize}	


\paragraph{4.1.2}\hypertarget{par:4.1.2}{} Не~верьте тем, кто занимается анализом магии. Они, как правило, маги, ищущие отмщения.
	\begin{itemize}
	\item 
 Я~имею ввиду Марка Оже, который всерьез с~<<удвоенной силой>>принялся критиковать колдунов с~Берега Слоновой Кости\footnote{{\itshape Marc Aug\'{e}.} Th\'{e}orie des pouvoirs et~id\'{e}ologie (Paris: Hermann, 1975).}. Это очень помогло мне не~принимать всерьез <<двойную>> критику со~стороны ученых. Когда все виды магии приводятся единому основанию, мы~получает новую форму скептицизма\footnote{{\itshape David Bloor}. Knowledge and Social Imagery (London: Routledge, 1976).}.
	\end{itemize}	

\paragraph{4.1.3}\hypertarget{par:4.1.3}{} Напротив, как только начинают понимать, что сила заключается в~альянсе слабостей, могущество исчезает. Конечно, силы все еще там, но~иллюзия могущества уничтожена. Нечто устраняет магическое воздействие могущества и~возвращает [силу] в~сеть, где она оформляется в~то, что я~называют <<ирредукция>>. 

Быть сильным, возможно, но~никогда могущественным. Убей меня, но~жди того, что я~возжелаю смерти и~поду на~колени перед властью. К~силе я~не~прибавлю {\itshape ничего}.

	\begin{itemize}
	\item 
 В~Интерлюдии III я~сказал, что мы~должны <<редуцировать тех, кто осуществляет редукцию>>. В~прежние времена борьба с~магией называлась <<Просвещением>>, но~это представление привело к~непредвиденным последствиям\footnote{В оригинале <<this image has backfired>>. --- {\itshape Прим. перев.}}. С~тех пор Просвещение стало эпохой (ир)радиации. Голова отважного исследователя, который пытался осветить тени обскурантизма, с~тех пор стала боеголовкой ракеты, которая ослепит нас светом. (Может быть, уже слишком поздно. Может быть, ракеты уже запущены. В~таком случае, давайте приготовимся к~последствиям очередной войны).
	\end{itemize}	


\paragraph{4.1.4}\hypertarget{par:4.1.4}{} Когда сеть скрывает основание ее~ассоциации, я~говорю, что она проявляет <<могущество>>. Когда составляющий ее~массив слабостей видим, я~говорю, что она проявляет <<силу>>.


\paragraph{4.1.5}\hypertarget{par:4.1.5}{} Мы~страдает не~от недостатка, но~от избытка духа. {\itshape Дух}, увы, никогда не~живет в~соответствии с~{\itshape буквой}. Дух~--- это всего лишь несколько слов среди многих других, которым несправедливо приписывается смысл всех остальных слов. Дух, таким образом, становится могущественной иллюзией. Истинно, говорю я~вам, дух слаб, но~буква.

	\begin{itemize}
	\item 
 На~словах верующие ставят телегу впереди лошади. Однако на~практике они поступают совершенно иначе. Они утверждают, что фрески, витражи, молитвы и~коленопреклонение это просто способы приблизиться к~Богу, его удаленному отсвету. Тем не~менее, они не~прекращают строить церкви и~выстраивать тела (arranging bodies) с~тем, чтобы создать средоточие могущества божественного. Мистикам хорошо известно, что если отбросить все элементы, которые, как говорят, указывают [на божественное], то~все что останется это ужасная ночь Нады (\hyperlink{par:1.4.6.1}{1.4.6.1}). Чисто духовная религия избавит нас от~религиозного. Убить букву (письмо, послание), значит зарезать курицу несущую золотые яйца.
	\end{itemize}	


\paragraph{4.1.6}\hypertarget{par:4.1.6}{} То, что мы~называем <<наукой>> состоит из~обширного множества элементов, власть которого мы~приписываем немногим. 
	\begin{itemize}
	\item 
	<<Наука>> существует не~в большей степени, чем <<язык>> или <<современный мир>> (\hyperlink{par:3.5.2}{3.5.2})
	\end{itemize}	


\paragraph{4.1.7}\hypertarget{par:4.1.7}{} То, что мы~называем наукой довольно случайным образом выбрано из~пестрой толпы актантов. Несмотря на~то, что она представляет (репрезентирует) других, она отрицает этот факт (\hyperlink{par:3.4.6}{3.4.6}). 
	\begin{itemize}
	\item 
	Те, кто называет себя <<учеными>> на~словах всегда ставят телегу впереди лошади, несмотря на~то, что на~практике они делают все в~точности наоборот. Они утверждают что лаборатории, библиотеки, встречи, полевые заметки, приборы и~тексты есть всего лишь {\itshape способы и~средства} явить на~свет истину. Но~они не~прекращают строить лаборатории, библиотеки, приборы с~тем, чтобы создать средоточие могущества истины. Рационалисты прекрасно знают, что если задушить эту второстепенную материальную жизнь, то~они будут вынуждены замолчать. Чисто научная наука избавит нас ученых. По~этой причине они так пекутся о~том, чтобы не~зарезать курицу, несущую золотые яйца.
	\end{itemize}	


\paragraph{4.1.8}\hypertarget{par:4.1.8}{} Они являются скептиками и~неверующими в~отношении ведьм и~священников, но~когда дело доходит до~науки, они легковерны. Без тени сомнения они заявляют, что ее~действенность происходит от~ее <<метода>>, <<логики>>, <<строгости>> или <<объективности>> (\hyperlink{par:2.1.0}{2.1.0}). Однако, в~отношении <<науки>> они совершают ту~же ошибку, которую делает шаман, когда приписывает могущество своим заклинаниям. Вера в~существование <<науки>> имеет своих реформаторов, но~у нее нет скептиков, и~еще меньше агностиков.


\paragraph{4.1.9}\hypertarget{par:4.1.9}{} Так ничто само по~себе не~является редуцируемым или нередуцируемым к~чему-либо еще (\hyperlink{par:1.1.1}{1.1.1}), невозможно, чтобы с~одной стороны были испытания и~слабости, {\itshape а~с другой что-то еще} (\hyperlink{par:1.1.2}{1.1.2}, \hyperlink{par:1.1.5.2}{1.1.5.2}, \hyperlink{par:2.3.4}{2.3.4}, \hyperlink{par:2.4.3}{2.4.3}, \hyperlink{par:2.5.1}{2.5.1}). Однако, хитрость <<науки>> (\hyperlink{par:4.1.7}{4.1.7}) делит силы, которые делают так, чтобы некоторые выглядели сильными, в~то~время как другие <<истинными>> или <<разумными>>.

\paragraph{4.1.10}\hypertarget{par:4.1.10}{}Если бы~люди перестали верить в~<<науку>>, то~не было ничего кроме испытаний сил. Но~даже <<в науке>> существуют только испытания сил. Это означает, что ирредукция <<науки>> одновременно необходима и~трудна. Необходима потому, что он~стала {\itshape единственным препятствием} на~пути нашего избавления от~магии; трудна потому, что это наша последняя иллюзия и~когда мы~защищаем ее, мы~верим, что защищает наше самое священное наследие. 
	\begin{itemize}
	\item 
	Если бы~это было не~так, я~бы~не посвятил целую главу критике <<науки>>, поскольку в~ней нет ничего особенного. 
	\end{itemize}	 


% \subparagraph{Интерлюдия VII: В~которой мы~узнаем, почему в~этом коротком фрагменте (конспекте, трактате) не~высказывается никакой симпатии в~адрес эпистемологии.}


\paragraph{4.2.1}\hypertarget{par:4.2.1}{} <<Наука>>~--- в~кавычках~--- не~существует. Это имя, которое приклеилось определенным сегментам определенных сетей, ассоциациям столь разреженным хрупким, которые бы~и вовсе остались без внимания, если бы~им не~приписывали все. 
	\begin{itemize}
	\item 
	Два или три процента ВНП нескольких промышленных государств (nations), две трети которого расходуется на~промышленность и~военные нужды~--- это не~так много. Малая доля того, что остается, имеет ценность лишь для нескольких тысяч людей, которые общаются еще с~несколькими тысячами и~популяризована на~благо миллионов прекрасных душ, которые с~трудом вообще это понимают. Для миллиардов других все эти сети {\itshape невидимы}.
	\end{itemize}	

\paragraph{4.2.2}\hypertarget{par:4.2.2}{} <<Наука>> не~самостоятельна. Она обретает форму, лишь отрицая то, что привело ее~власти и, приписывая собственную солидность не~тому, что удерживает, а~тому, что удерживается вместе (\hyperlink{par:2.4.7}{2.4.7}). В~этом отрицании <<она>> игнорирует саму себя.
	\begin{itemize}
	\item 
	Если бы~<<физику>> лишили племен ублюдков, которые выполняют грязную работу, то~ее литературные произведения невозможно было бы~отличить от~литературных произведений алхимиков или психоаналитиков: было ли~это возможно в~прошлом, когда племен было не~так много.
	\end{itemize}	


\paragraph{4.2.3}\hypertarget{par:4.2.3}{} <<Наука>> это искусственная сущность {\itshape несправедливо} отделенная от~гетерогенных сетей. Есть две мерки: одна для ученых, другая для всех остальных. 
	\begin{itemize}
	\item 
	Если капиталист продает неприбыльную фабрику, его обвиняют в~жадности. Однако если известный ученый отказывается от~дискредитировавшей себя гипотезы, тогда напротив считают, что он~проявляет незаинтересованность. Если незадачливая ведьма приписывает успех в~битве магическому ритуалу, то~ее высмеивают за~ее легковерие. Но~если прославленный исследователь приписывает успех собственной лаборатории революционной идеи, никто не~смеется, хотя каждый был бы~должен. Мысль о~том, чтобы сделать революцию с~помощью идей! Если потребители разрезают бифштексы на~маленькие кусочки, чтобы их~было легко прожевать, то~это никто не~комментирует. Но~если известный философ из~Амстердама утверждает, что мы~должны <<делить каждую трудность на~как можно большее число частей>>, то~невозможно выразить большего восхищения <<методом верно направляющим разум и~отыскивающим истину в~науках>>. Если самый безвестный фанатик-попперианец говорит о~<<фальсификации>>, люди готовы увидеть глубокую тайну. Но~если мойщик окон наклоняет голову, чтобы увидеть находится ли~пятно, которое он~хочет смыть внутри или снаружи, никто не~удивляется. Если молодая пара сменила часть мебели в~их~гостиной и, мало помалу, заключает, что обстановка не~выглядит должным образом, и~что необходимо поменять всю мебель, чтобы все опять подходило друг другу, кто находит это достойным упоминания? Но~когда вместе столов меняют <<теории>>, тогда люди говорят о~куновском <<парадигмальном сдвиге>>. Я~груб, но~это необходимо в~сфере, где несправедливость так глубоко [укоренена]. Они смеются над теми, кто верит в~левитацию, но~утверждают, не~встречая возражений, что теории могут создать мир.
	\end{itemize}	


\paragraph{4.2.4}\hypertarget{par:4.2.4}{} <<Наука>> производит впечатление существующей, только превращая свое существование в~{\itshape перманентное чудо}. Будучи неспособной принять своих настоящих союзников, она вынуждена объяснять чудо другим, а~другое третьим. Это продолжается до~тех пор, пока не~становится похожим на~сказку. 
	\begin{itemize}
	\item 
	Кто-то называет чудом то, что <<математика применима к~физической реальности>>. Другие говорят, что <<самая непостижимая вещь о~вселенной состоит в~том, что она вообще постижима>>. Остальные по-прежнему выражают изумление по~поводу того, что законы физики <<универсально применимы>>, что Ньютон открыл их, и~что Эйнштейн их~революционизировал. <<Наука>> становится по~истине цирковой интермедией с~гениями, революциями и~dei ex~machine. Но~никто не~говорит о~комнате страха позади всего этого. Когда становимся агностиками, мы~должны признать, что большая часть мест научного паломничества выглядят так же, как Лурд, но~все же~они более легковерны (наивны), поскольку высмеивают Лурд!
	\end{itemize}	

\paragraph{4.2.5}\hypertarget{par:4.2.5}{} <<Наука>> является санктуарием только до~тех пор, пока мы~будет рассматривать победителей и~побежденных ассиметрично. 
	\begin{itemize}
	\item 
	Никто не~может отделить <<внутреннюю>> историю науки, от~<<внешней>> истории ее~союзников. Первая вообще не~может считаться историей. В~лучшем случае это историография двора, в~худшем~--- Жития Святых. Последняя это не~история <<науки>>, это история.
	\end{itemize}


\paragraph{4.2.6}\hypertarget{par:4.2.6}{} Вера в~существование <<науки>> это следствие преувеличения, несправедливости, асимметрии, невежества, легковерия и~отрицания. Если <<наука>> отличается от~всего остального, то~это конечный результат длинной череды силовых переворотов (coups de~force).


\paragraph{4.3.1}\hypertarget{par:4.3.1}{} <<Наука>> чересчур ветха, чтобы говорить о~ней. Вместо этого мы~должны говорить {\itshape союзниках}, которых определенные сети используют для того, чтобы стать сильнее других (\hyperlink{par:1.3.1}{1.3.1}, \hyperlink{par:2.4.1}{2.4.1}, \hyperlink{par:3.3.1}{3.3.1}). В~таком случае вместо могущества мы~увидим силу (\hyperlink{par:4.1.5}{4.1.5}).


\paragraph{4.3.2}\hypertarget{par:4.3.2}{} Знание не~существует~--- как это (what would itshape be) (\hyperlink{par:1.4.3}{1.4.3})? Есть только ноу-хау. Другими словами, существуют ремесла и~профессии. Несмотря на~все утверждения об~обратном, ключ от~знаний в~руках у~ремесел. Они создают возможность возвращения <<науки>> в~сети, из~которых она произошла (Введение).


\paragraph{4.3.3}\hypertarget{par:4.3.3}{} Мы~не думаем. У~нас нет идей (\hyperlink{par:2.5.4}{2.5.4}). Скорее есть акт {\itshape письма}, акт который предполагает работу с~добытыми {\itshape записями} ({\itshape inscriptions}), акт который осуществляется посредством {\itshape разговоров} с~другими людьми, которые также пишут, записывают, говорят живут в~таких же~необычных местах; акт, который убеждает или не~способен {\itshape убедить} помощью записей (inscriptions), которых заставляют говорить, писать или быть прочитанными (\hyperlink{par:3.1.0}{3.1.0}, \hyperlink{par:3.1.9}{3.1.9}). 
	\begin{itemize}
	\item 
	Когда мы~говорит о~<<мысли>>, даже самые большие скептики теряют свои критические способности. Как вульгарные колдуны, он~позволяют <<мысли>> подобно магии путешествовать с~высокой скоростью на~большие расстояния. Я~не~знаю никого, кто не~был бы~легковерен, когда дело доходит до~идей. Тем не~менее <<мысль>> действительно совершенно проста, поскольку, когда мы~пишем о~других записях (inscriptions), мы~действительно покрываем большие расстояния за~несколько сантиметров. Карты, диаграммы, столбцы, фотографии, спектрографы~--- вот те~материалы, о~которых забывают, материалы, которые используются для того, чтобы сделать <<мысль>> неуловимой.
	\end{itemize}


\paragraph{4.3.4}\hypertarget{par:4.3.4}{} Несмотря на~все впечатления об~обратном, защищать то, что написано только лишь на~бумаге, рискованное ремесло. Однако это ремесло не~является более чудотворным, чем ремесло художника, моряка, канатоходца или банкира.
	\begin{itemize}
	\item 
	Интересно видеть грека, который склонился над слепящей поверхностью пергамента и~страстно следует за~надрезами стилуса, даже тогда, когда они приводят к~софизмам. Увлекательно видеть Отцов Церкви, распространяющих разные версии одного и~того же~текста и~обучающихся ремеслу экзегетики, матери {\itshape всех} научных дисциплин. Волнительно следить за~итальянцем, переписывающим заново в~своих Диалогах книгу природы математической форме\footnote{{\itshape Elizabeth Eisenstein}. The Printing Press as~an Agent of~Change (Cam­bridge: Cambridge University Press, 1979).}. Увлекательно изучать, как это делал я~в течение двух лет, иголки, царапающие цилиндры физиографов; видеть, как расставляют ловушки, для того чтобы заставить то, о~чем говорят, писать (\hyperlink{par:3.1.5}{3.1.5}) и~говорить напрямую с~теми, кого хотят убедить. Эти странные тексты, которые являются не~священными писаниями, но~записями, произведенными внутренностями крысы или открытыми сердцами собак, странным образом очаровательны. Все они очень красивы, я~согласен. Они говорят об~огромной работе и~большой сноровке, но~они не~чудотворны. Нет ничего нематериального в~том, как бесконечно рассыпаются переплеты, щелкают ручки, стрекочут принтеры и~скрипят стилусы. Нет ничего нематериального в~этой одержимости письмом, записыванием, диаграммами и~спектрограммами.
	\end{itemize}	

\paragraph{4.3.5}\hypertarget{par:4.3.5}{} Обращены ли~они к~природе? Что бы~это могло значить? Посмотрите на~них! Они опирают на~то, что они написали и~на~разговоры друг с~другом внутри своих лабораторий. Посмотрите на~них! Их~единственным принципом реальности является тот, который они установили сами (\hyperlink{par:1.2.7}{1.2.7}). Посмотрите на~них! <<Внешние>> референты, которые они создали, существуют, только внутри их~мира (\hyperlink{par:1.2.7.1}{1.2.7.1}).


\paragraph{4.4.1}\hypertarget{par:4.4.1}{} Все что локально всегда таковым и~остается. Ни~один тип работы не~является {\itshape более} локальным, чем какой-либо другой, пока его не~завоюют (\hyperlink{par:1.2.4}{1.2.4}) и~не~заставят оставить след. После над ним могут работать {\itshape в~его отсутствии}.
	\begin{itemize}
	\item 
	Африканского охотника, который покрывает десятки квадратных миль, и~который научился распознавать {\itshape сотни} {\itshape тысяч} знаков и~меток, называют <<локальным>>. Но~про картографа, который научился распознавать {\itshape несколько сотен} знаков и~индексов, склонившись над несколькими ярдами карт и~аэрофотоснимков, говорят, что он~более универсален, чем охотник и~обладает глобальным видением. Кто из~них скорее потеряется на~территории другого? До~тех пор, пока мы~не проследим длинную историю превращения охотника в~раба, а~картографа в~господина, у~нас не~будет ответа на~этот вопрос. Между глобальным и~локальным нет тропинки, поскольку глобального не~{\itshape существует}. Вместо этого у~нас есть географы, самолеты, карты и~Ежегодные международные съезды геодезистов (International Geodesic Years).
	\end{itemize}	


\paragraph{4.4.2}\hypertarget{par:4.4.2}{} <<Общие идеи>> могут быть выстроены, но~сделать это не~более и~не~менее трудно, чем построит сеть железных дорог. Мы~должны заплатить за~<<общие идеи>>. Мы~не можем переехать с~одного стола на~другой с~помощью понятия <<стол>>. Для того чтобы переехать нам потребуется сеть, поддержание которой стоит так же~дорого, как и~поддержание железнодорожной сети с~ее~стрелочниками, замечательными железнодорожниками, бухгалтерами и~сигналами. 
	\begin{itemize}
	\item 
	Ученые очень хорошо понимают принцип <<приватизации прибыли и~национализации убытков>>. Они заставляют нас поверить, что они думают и~что идеи ничего не~стоят, но~они просят нас заплатить за~их лаборатории, лектории и~библиотеки (\hyperlink{par:4.1.9}{4.1.9}).
	\end{itemize}	


\paragraph{4.4.3}\hypertarget{par:4.4.3}{} Когда овладевают серией дислокаций и~соединяют их~в сеть, становится возможным перемещаться из~одного места в~другое, не~замечая той работы, которая связала их~вместе. {\itshape Одна} <<дислокация>> как кажется <<потенциально>> содержит в~себе все остальные. Я~рад назвать жаргон, который используется для того, чтобы проникнуть в~эти сети, <<теорией>> до~тех пор, пока подразумевается, что он~является подобием указателей и~меток, которые мы~используем для того, чтобы найти {\itshape обратный путь}.
	\begin{itemize}
	\item 
	Есть жаргон торговцев финикийского побережья, портовых грузчиков, финансистов, людей в~белых халатах, которые читают в~световых годах и~взвешивают в~пикограммах. Как все они могут понять друг друга? У~них нет общих целей. Они не~движутся вдоль одних и~тех же~силовых линий и~не~манипулируют одними и~теми же~следами. То, что мы~называем <<теорией>> реально не~более и~не~менее чем карта метро в~метро (\hyperlink{par:2.1.7}{2.1.7}).
	\end{itemize}	


\paragraph{4.4.4}\hypertarget{par:4.4.4}{} <<Универсальность>> так же~локальна, как и~все остальное. Универсальность существует только <<in potentia>>. Другими словами она не~существует до~тех пор, пока мы~не будем готовы заплатить высокую цену за~постройку и~поддержание дорогостоящих опасных связей. 
	\begin{itemize}
	\item 
	Если все случается локально и~только один раз (\hyperlink{par:1.2.1}{1.2.1}) и~если ни~одно место не~может быть редуцировано к~другому, тогда каким образом одно место может содержать в~себе другое? Не~обвиняйте меня в~номинализме. Все части армии {\itshape могут} быть соединены со~штабом. Офицеры ВВС {\itshape могут} работать с~картой мира размером три на~четыре метра. Все часы мире {\itshape могут} быть синхронизированы, если установлено универсальное время. Я~просто хочу, чтобы стоимость создания этих универсалий и~те~узкие каналы, по~которым они циркулируют, были включены в~счет.
	\end{itemize}	


\paragraph{4.4.5}\hypertarget{par:4.4.5}{} Так вы~верите, что применение математики к~физическому миру это чудо? Если так, тогда я~предлагаю вам полюбоваться другим чудом; я~могу путешествовать по~всему мир. с~моей кредиткой Америкэн Экспресс. Вы~скажите о~втором: <<Это всего лишь сеть. Если вы~выйдете за~ее пределы хотя бы~на дюйм, то~ваша кредитка не~будет иметь никакой ценности>>. Именно так. Это то, что я~говорю о~математике и~науке, {\itshape не~больше и~не~меньше}. 
	\begin{itemize}
	\item 
	Уравнение второй степени имеет область распространения, которую можно нанести на~карту, как и~все остальное. Его изобретение, перевод и~инкорпорацию в~другие практики можно проследить тем же~самым образом, каким мы~регистрируем распространение сбруи, кормового руля, галстука-бабочки, регулятора хода часов или IQ~тестов. Но~мы не~можем удержаться от~того, чтобы делить профессии на~две кучи. Одни жестко встроены свои контексты, в~то~время как другие летают как духи вне контекста. Я~хочу похоронить этих духов на~дне их~сетей, чтобы не~дать им~вновь вернуться после наступления темноты и~преследовать нас.
	\end{itemize}	

\paragraph{4.4.5.1}\hypertarget{par:4.4.5.1}{} <<Универсальное>> более не~может поглотить партикулярное так же, как исторические полотна могут заменить собой натюрморты. Теории не~могут быть абстрактными, если они таковы, то~это название указывает на~стиль, как абстрактная живопись. 
	\begin{itemize}
	\item 
	Когда кто-либо говорит мне об~универсальном, я~всегда спрашиваю какого оно размера, кто проецирует его и~на~какой экран. Я~также спрашиваю сколько людей обслуживают его и~сколько стоит оплата их~труда. Я~знаю, что это дурной тон, но~король голый и~выглядит одетым только благодаря тому, что мы~верим в~универсальное.
	\end{itemize}	

\paragraph{4.4.6}\hypertarget{par:4.4.6}{} Каким образом достигаются <<абстрактность>>, <<формализм>>, <<точность>> <<чистота>>? Как сыр~--- посредством фильтрования, сепарирования, формовки и~выдержки. Или как бензин~--- посредством очистки, крекинга, дистилляции. Нам нужны сыроварни и~очистительные заводы. Все это дорогие процессы и~грязные, вонючие профессии.

\paragraph{4.4.6.1}\hypertarget{par:4.4.6.1}{} {\itshape Работа} абстракции не~более абстрактна, чем работа могильщика; {\itshape ремесло} формализатора не~менее формально, чем ремесло мясника; работа по~очищению не~более чиста, чем работа санитарного инспектора. Сказать, что какие-либо процедуры являются чистыми, формальными или абстрактными, значит путать глагол с~прилагательным. Мы~можем также сказать, что дубление выдублено, фильтрование отфильтровано или логика логична.


\paragraph{4.4.7}\hypertarget{par:4.4.7}{} Быть абстрактным более не~в нашей власти, мы~можем лишь говорить надлежащим образом~(\hyperlink{par:2.2.1}{2.2.1}).


\paragraph{4.4.8}\hypertarget{par:4.4.8}{} Сети тонки, хрупки и~разрежены. Мы~читаем и~пишем внутри них. Мы~способны убедить, только расширяя сеть, иными словами редуцируя масштаб всего того, что было абсорбировано. В~результате несколько людей сидящих за~столом в~одной комнате, могут видеть (survey) все. {\itshape Что может быть проще? }Здесь не~из-за чего поднимать шум. 

%\subparagraph{Интерлюдия VIII: В~которой маленький пример повседневной социологии показывает, что такое меры.}

\paragraph{4.5.1}\hypertarget{par:4.5.1}{} В~научных профессиях, также как и~во~всех других, мы~учимся тому, как увеличить нашу силу локально (\hyperlink{chap1}{Глава 1}).


\paragraph{4.5.2}\hypertarget{par:4.5.2}{} Прибавление силы, достигнутое в~лаборатории, проистекает из~того факта, что большим количеством маленьких объектов манипулировали множество раз, что эти микрособытия могли быть записаны, что они при желании могли быть прочитаны вновь весь процесс в~целом может быть записан так, чтобы люди его прочитали. Для этого, требуются умения и~много денег, но~колдовство здесь не~при чем. 
	\begin{itemize}
	\item 
 Не~важно туманности ли~это, кораллы, лазеры, микробы, Валовые Национальные Продукты или результаты I.\,Q. теста. Не~важно являются ли~они <<бесконечно большими>> или <<бесконечно малыми>>. О~них {\itshape с~уверенностью} можно говорить только тогда, когда они будут перенесены в~маленькое место, где ими может управлять небольшое количество людей, и~их~заставят показать знаки~--- кривые, фигуры, точки, лучи или полосы~--- которые настолько просты, что согласие является возможным. Мы~можем только запинаться относительно всего остального.
	\end{itemize}	

\paragraph{4.5.2.1}\hypertarget{par:4.5.2.1}{} Правило достаточно просто: если мы~хотим увеличить нашу силу, пойдем тысячей на~одного в~тех вопросах, которые окупятся в~одном шансе из~ста. 
	\begin{itemize}
	\item 
	Представьте себе бациллу антракса, которая в~течение миллионов лет жила, будучи спрятанной в~толпе своих сородичей. В~один прекрасный день она обнаруживает себя одиночестве вместе со~своими детьми под слепящем светом микроскопа, над которым возвышается огромная борода Пастера. Кроме мочи ей~больше негде жить (\hyperlink{chap1}{Глава 1}). Это хороший пример переворота в~балансе сил. Разве точность всегда не~вырастает из~таких переворотов? В~действительности требуется слепота веры, чтобы игнорировать испытания сил, которые имеют место в~пыточных камерах науки~--- биотестерах, тензиметрах, линейных акселераторах, прессах, иглах, стилусах, вакуумных насосах, калориметрах. Оставаться слепым перед лицом этих испытаний равносильно <<отчаянному сопротивлению этому {\itshape вопросу}>>! Те, кто верит в~<<науку>> вместо этого являются настоящими мучениками.
	\end{itemize}	

\paragraph{4.5.3}\hypertarget{par:4.5.3}{} Итак, они более уверены в~себе чем другие? Конечно! Они множество раз испробовали свои аргументы на~маленьких моделях и~сделали все возможные ошибки. Очевидно, что они более уверены, чем те, у~кого есть только одна попытка.
	\begin{itemize}
	\item 
	Уважаемый эксперт ничем не~отличается от~презираемого всеми политика. Эксперт делает большое количество скрытых маленьких ошибок и~уверенно появляется из~укрытия {\itshape в~самый последний момент}. Политик совершает по-настоящему большие ошибки вынужден делать это перед всеми. Здесь решения принимаются {\itshape до} ошибок (\hyperlink{par:3.6.3}{3.6.3}). Все люди одинаковы~--- в~равной степени правдивы, и~в равной степени лживы. Иначе как бы~они могли существовать.
	\end{itemize}	 

\paragraph{4.5.4}\hypertarget{par:4.5.4}{} Единственный способ быть сильным вновь, это воспроизвести отношения сил, которые некогда были выгодны. {\itshape Нет такой вещи как предсказание.} Предсказание это повторение чего-то, что уже имело место, в~увеличенном или уменьшенном масштабе. Только волшебники верят в~то, что они могут предсказывать будущее. 
	\begin{itemize}
	\item 
	Если мы~находим чудодейственным тот факт, что невакцинированная овца умирает Пуйи-ле-Фор или что Вояджер II~прошел между кольцами Сатурна в~установленный момент времени, тогда нам следовало бы~считать смерть Гамлета в~последнем акте столь же~удивительной. Ни~одно предсказание не~является чем-то большим, чем менеджмент сцены, который учит тому, как повторять генеральную репетицию~--- хотя он~и не~избавляет от~страха сцены и~тревожного ожидания. До~тех пор пока дело касается прогнозов Пастер, Шекспир и~НАСА оказываются неотличимыми. Если бы~им нужно было импровизировать или предсказывать, они бы~бессвязно бормотали как Пифия, так же~как это делаем мы, когда покидаем убежища наших профессий. И~Шекспир, вероятно, был бы~менее непоследовательным, чем любой другой. В~театре доказательства, или обычном театре все режиссеры одинаковы, в~равной степени лживы и~правдивы. Каким образом они могут различаться?
	\end{itemize}	

\paragraph{4.5.5}\hypertarget{par:4.5.5}{} Узнать это можно только посредством испытания сил. <<Знание>> это положение на~данном участке фронта. Оно не~простирается дальше. Как это возможно? (\hyperlink{par:1.1.0}{1.1.0}).
	\begin{itemize}
	\item 
	Ученые говорят, что приходят к~выводам в~лаборатории, <<при прочих равных условия>>, но~потом они забывают [об этом], предпочитая путешествовать магическим образом другие места, и~продолжая заниматься законотворчеством, как если бы~они все еще были дома.
	\end{itemize}	


\paragraph{4.5.6}\hypertarget{par:4.5.6}{} Ничего нельзя узнать за~пределами сетей, организованных и~управляемых ноу-хау (\hyperlink{par:1.3.7}{1.3.7}), но~эти сети могут быть расширены.


\paragraph{4.5.7}\hypertarget{par:4.5.7}{} Нет такой вещи как <<знание>> (\hyperlink{par:4.3.2}{4.3.2}), но~есть возможность осознавать (realize), то~есть делать реальным, понимать. 
	\begin{itemize}
	\item 
	Тайна {\itshape adequatio rei et~intellectus} есть просто расширение лаборатории. Если мы~не верим магию, это расширение становится видимым, но~если мы~обращаем массив слабостей чудодейственную власть, это расширение скрывается. У~<<науки>> нет внешнего (\hyperlink{par:4.3.5}{4.3.5}), но~только узкие галереи, которые позволяют лабораториям расширяться и~пробираться места, которые могут быть далеко. 
	\end{itemize}	

\paragraph{4.5.7.1}\hypertarget{par:4.5.7.1}{} Ничто не~покидает [пределы] сетей, и~в меньшей степени [это может сделать] ноу-хау, но~кто сомневается в~том, что сеть, которая оплачивает издержки, может расшириться? <<Докажи мне, что это вещество, которое так хорошо действует в~Париже столь же~хорош. в~предместьях Тимбукту>>.

<<С какой стати? Есть универсальный закон>>.

<<Я не~хочу просто {\itshape поверить} в~это. Я~хочу {\itshape увидеть} это>>.

<<Просто подожди пока я~построю лабораторию и~я докажу это тебе\ldots>>

Спустя несколько лет и~миллионов долларов я~своими собственными глазами увидел доказательство того, о~чем я~просил в~новехонькой лаборатории. Я~выхожу, проезжаю несколько миль, и~ставлю вопрос снова:

<<Докажи мне, что\ldots>> 

	\begin{itemize}
	\item 
	Когда люди говорят о~том, что знание является <<универсально истинным>>, то~мы должны понимать, что это так же, как с~железными дорогами, которые существуют по~всему миру, но~только имеют ограниченную протяженность. Перейти к~утверждению того, что локомотивы могут двигаться за~пределами своих узких и~дорогостоящих путей это совсем другое дело. Тем не~менее, маги пытаются ослепить нас с~помощью <<универсальных законов>>, которые, как они утверждают, сохраняют свою силу даже в~разрывах между сетями!
	\end{itemize}

\paragraph{4.5.7.2}\hypertarget{par:4.5.7.2}{} Каким образом можно распространить ноу-хау? Так же~как радиоприемники, которые делаются в~Гонконге, или как таблицы умножения! Должны быть покупатели и~продавцы, учителя и~коммерческие каналы, представители и~книги, которые считаются заслуживающими доверия. 
	\begin{itemize}
	\item 
 Мы~говорим, что законы Ньютона могут быть обнаружены в~Габоне и~что это достаточно удивительно, так как это далеко от~Англии. Но~я видел камамбер в~супермаркетах Калифорнии. Это тоже достаточно удивительно, так как Лизьё это далеко из~Лос-Анджелеса. Либо есть два чуда, каждым из~которых следует восхищаться одинаково, либо нет ни~одного.
	\end{itemize}	

\paragraph{4.5.7.3}\hypertarget{par:4.5.7.3}{} Люди обычно говорят о~<<научной истине>> шепотом. Однако всегда существовало лишь три способа почитать ее: согласованность~--- <<это логично>>; репрезентация~--- <<это соответствует>>; эффективность~--- <<это работает>>. Эти три выражения служат для того, чтобы просто указывать на~степень распространения сети. 
	\begin{itemize}
	\item 
 На~кухонной латыни мы~бы сказали {\itshape adequatio laboratorii et~laboratorii, adequatio laboratorii et~alius laboratoris, adequatio laboratorii et~vulgi percoris}.
	\end{itemize}


\paragraph{4.5.8}\hypertarget{par:4.5.8}{} Одна форма ноу-хау не~более <<истинна>> чем другая. Она не~более и~не~менее истинна, чем кофейник, дерево или лицо ребенка. Вот она, на~мгновение установившаяся линия сил (\hyperlink{par:1.1.6}{1.1.6}). Слово <<истинный>> это приложение, добавляемое к~конкретным испытаниям силы, чтобы ослепить тех, кто все еще может в~них сомневаться. 
	\begin{itemize}
	\item 
	Рационалисты смеются над ордалиями, которые делают победителя в~схватке правым. Однако, они ежедневно коронуют победителей научных споров, утверждая, что у~них более чистые сердца и~более рациональные умы! Есть две меры, два стандарта (\hyperlink{par:4.2.3}{4.2.3}).
	\end{itemize}	


\paragraph{4.5.9}\hypertarget{par:4.5.9}{} Мы~можем сказать, что все, что сопротивляется реально! (\hyperlink{par:1.1.5}{1.1.5}) Слово <<истина>> делает только лишь маленькое дополнение к~испытанию сил. Это не~много, но~это производит впечатление могущества (\hyperlink{par:2.5.2}{2.5.2}), которое спасает от~испытания то, что не~выдержит [проверки]. 
	\begin{itemize}
	\item 
	Релятивисты и~идеалисты никогда не~были в~состоянии удерживать свои позиции достаточно долго (\hyperlink{par:1.3.6}{1.3.6}), поскольку утверждения выходящие из~лабораторий противостоят, сопротивляются и~потому являются реальными (\hyperlink{par:2.4.7}{2.4.7}). Но~они правы: это не~основание для того, чтобы верить сказкам.
	\end{itemize}	

\paragraph{4.5.10}\hypertarget{par:4.5.10}{}Если нечто сопротивляется, то~оно создает у~тех, кто испытывает его оптическую иллюзию относительно того, что есть некий видимый и~описываемый объект, который вызывает это сопротивление. Но~объект это следствие, а~не~причина. Иллюзия исчезает, когда фронт борьбы смещается и~осторожно появляется вновь, когда фронт опять стабилизируется. 
	\begin{itemize}
	\item 
	<<Реальные миры там>>~--- это следствия стабильных силовых линий, а~не~причина их~стабилизации.
	\end{itemize}	

\paragraph{4.5.11}\hypertarget{par:4.5.11}{}Мы можем перформировать, трансформировать, деформировать и~таким образом формировать и~информировать самих себя, но~мы не~можем {\itshape описывать} что-либо. Другими словами не~существует репрезентации, кроме как в~театральном и~политическом смыслах этого термина. 
	\begin{itemize}
	\item 
	Трудность в~отношении <<наук>>, возможно, возникает из-за того факта, что работа, осуществляемая руками, приносит записи, которые читаются глазами. Возможно, эпистемология~--- это смешение чувств. Мы~следуем за~ослепленным взглядом, но~забываем о~руках, которые пишут, комбинируют, монтируют. Однако не~существует <<теории>>, <<созерцания>>, <<умозрения>>, <<предвидения>>, <<видения>> и~<<знания>>. Солнце Платона никогда не~припекало и~не~обращалось в~небе. Но~внутри сетей есть электроны, лампы накаливания, и~проекторы, которые потребляют электричество и~есть объекты так же, как и~кое-что еще. Такие лампы не~окружены ореолом тайны. Они включены в~свои розетки реальными руками.
	\end{itemize}	


\paragraph{4.6.1}\hypertarget{par:4.6.1}{} Почему нас должно удивлять, что те, кто собрали избыток сил и~присоединились со~своим влиянием к~конфликту, где никто не~имел преимущества, должны победить?


\paragraph{4.6.2}\hypertarget{par:4.6.2}{} Когда мы~не может победит только благодаря своим собственным силам, мы~говорим о~тех, кем мы~повелеваем как <<власти>>, а~о соотношении сил как о~<<знании>>. Наши оппоненты могут быть в~состоянии сопротивляться сложению <<сил>>, но~не превосходству <<знания>> над <<властью>>.
	\begin{itemize}
	\item 
	Вот как можно объяснить разделение сил, с~которым мы~столкнулись с~самого начала (\hyperlink{par:1.1.5.2}{1.1.5.2}, \hyperlink{par:4.1.9}{4.1.9}). Это различение обозначает не~нечто явное, а~стратагему, которая умножает силы десятикратно, описывая некоторые из~них как <<науку>>. <<Наука>> сродни мечу Бренна, брошенному на~весы. Да, {\itshape vae victis}, поскольку их~объявят <<нелогичными>>, <<плохими>> и~<<неразумными>>. <<У неимеющего отнимется и~то, что имеет>>\footnote{<<Ибо всякому имеющему дастся и~приумножится, а~у неимеющего отнимется и~то, что имеет>> (Матфей 25:29).}. 
	\end{itemize}	

\paragraph{4.6.2.1}\hypertarget{par:4.6.2.1}{} Если бы~мы могли объяснить <<науку>> в~терминах <<политики>>, то~тогда бы~не было наук, так как их~развивают именно для того, что находить союзников, новые ресурсы и~пополнение войск. 
	\begin{itemize}
	\item 
	Вот почему социология науки так прирожденно слаба. Огюст Конт, отец сциентизма социологии, изобрел забавную систему двойной бухгалтерии. Наука~--- это не~политика. Это политика, осуществляемая {\itshape другими средствами}. Но~люди возражают <<наука не~редуцируется к~власти>>. Совершенно верно. Она не~редуцируется к~власти. Она предлагает другие средства. Но~на это опять возражают, что <<эти средства по~природе своей не~могут быть предвидены>>. Совершенно верно. Если бы~их можно было предвидеть, то~они бы~уже были использованы противостоящей силой (power). Что может быть лучше новой формы власти, которую никто не~знает, как использовать? Призывайте резервы! Мое почтение Шейпину и~Шафферу\footnote{{\itshape Steve Shapin, Simon Schaffer}. Leviathan and the Air-Pump (Prince­ton: Princeton University Press, 1985).}.
	\end{itemize}	

\paragraph{4.6.3}\hypertarget{par:4.6.3}{} Теперь, когда нас больше нельзя одурачить с~помощью этих маневров, мы~видим делегатов (\hyperlink{par:3.1.3}{3.1.3}), кем бы~они ни~были, говорящих от~имени других акторов, какими бы~они не~были. Мы~видим, как они бросают в~битву одно за~другим соединения своих союзников, отчасти колеблющихся, отчасти воинственных. 

	\begin{itemize}
	\item 
	Первый делает успехи, следуя за~своими микробами; второй~--- за~разгневанными рабочими; третий~--- за~своими китами, о~нуждах и~численности которых он~знает, которых он~хочет спасти; четвертый~--- за~своими батальонами; пятый~--- за~своим Кораном и~нефтедолларами; шестой~--- за~важными деловыми кругами, которые он~представляет; седьмой~--- за~своим бульдозером; еще один~--- за~своими овцами и~собакой. Все они выстроены в~боевые порядки и~проранжированы в~соответствии с~номерами, под которыми они были зачислены. Все они устанавливают, что является реальным на~линии фронта их~испытаний. Если мы~пытаемся разделить эту толпу на~людей и~не-человеков или на~<<политическое>> и~<<научное>>, то~тогда мы~совершаем ошибку~--- мы, я~настаиваю, мы~совершаем предательство (\hyperlink{par:4.7.0}{4.7.0}).
	\end{itemize}	

\paragraph{4.6.4}\hypertarget{par:4.6.4}{} Но~что бы~эти энтелехии, которые вступили в~наши конфликты, сказали, если бы~они могли говорить {сами за~себя}? <<То же~самое>>, поскольку их~заставляют говорить. Какие могут быть сомнения, когда блестящие демонстрации принуждают нас признавать это ежедневно? 
	\begin{itemize}
	\item 
	Иногда люди говорят о~<<природе>>, ссылаясь на~толпу рабов и~покоренных актантов, которых заставили замолчать, или когда говорят о~приказаниях, отдаваемых классом исследователей, которые в~свою очередь пляшут по~дудку горстки <<великих мыслителей>>. Однако крайне маловероятно, что силы ведут себя таким образом. В~конце концов, только два или три процента ВНП нескольких стран циркулирует внутри разреженных и~хрупких сетях <<науки>>. Мы~с таким же~успехом могли бы~редуцировать все путешествия к~сети авиалиний (\hyperlink{par:2.1.8.1}{2.1.8.1}). Актор должен достичь гегемонии, чтобы говорить в~единственном числе о~<<природе>> или <<реальном мире там>>. Гегемония является причиной, а~не~следствием <<мира>> в~единственном числе.
	\end{itemize}

\paragraph{4.6.5}\hypertarget{par:4.6.5}{} Но~что бы~могли сказать бесчисленные актанты зачисленные в~наши конфликты и~наши блестящие демонстрации, если бы~они могли говорить сами за~себя? Мы~не имеем понятия. Не~потому что они непознаваемы (\hyperlink{par:1.2.12}{1.2.12}), и~не~потому что они неописуемы (\hyperlink{par:2.2.3}{2.2.3}), но~потому что никто не~пытался, или скорее потому что те, кто пытался, стали слабее чем были до~этого.


\paragraph{4.6.6}\hypertarget{par:4.6.6}{} Мы~все еще мало знаем <<объективно>>. Мы~знает нечто только потому, что одни силы растут за~счет других. Мы~не имеет ни~малейшего представления о~том, что связывает силы вместе до~тех пор, пока они не~начнут действовать как пробы и~факты в~наших лабораторных конфликтах (\hyperlink{par:1.3.1}{1.3.1}).


\paragraph{4.6.7}\hypertarget{par:4.6.7}{} Как только мы~редуцируем редукцию <<наук>>, мы~вынуждены признать, что <<знание>> может существовать только на~уровне следов~--- во~всех смыслах этого термина. 
	\begin{itemize}
	\item 
 Мы~часто различаем между знанием прошлого и~современного мира (\hyperlink{par:3.2.5}{3.2.5}, \hyperlink{par:3.3.0}{3.3.0}). Это Великий Разлом, который не~позволяет нам видеть, что все эти знания обладают одним и~тем же~двигателем и~одной и~той же~общей формой: они не~заинтересованы в~вещах самих по~себе, в~том, чтобы проследить {\itshape их} пути; они касаются только человека и~видоизменений, которым человек может быть принудительно подвергнут. Как мы~обычно говорим, они <<социальны, слишком социальны>>. Образно говоря, мы~могли бы~сказать, что древние кривды и~современные правды относятся друг к~другу как две революции одной спирали. Несомненно, первая меньше, чем вторая, но~обе они обращаются за~помощью к~обществу. 

	Однако они различаются, явно различаются. Эти различия не~имеют никакого отношения ни~к критической строгости, с~которой получаются [эти знания], ни~к наличию данных. Различие состоит просто в~их~размере. В~прошлом поддерживались только малые коллективы. К~<<вещами>> проявляли интерес только с~той целью, чтобы усмирить их. Это знание теперь называют ложным, потому что оно слишком маленькое. Со~строительством больших Левиафанов стало необходимым интересоваться большим количеством вещей в~течение более длительного времени, чтобы быть более точными, более педантичными и~проникнуть в~гущу еще большего количества сил с~помощью еще большего количества лабораторий. Но~цель осталась той же: это все еще был человек, который должен был быть реформирован, деформирован, трансформирован и~информирован. Да, то~знание, которое мы~считаем новым, является столь же~антропо{\itshape{морфным}}, как и~его предшественники. Нет, оно является таковым даже в~большей степени! Поскольку стало необходимым завоевывать большие количества людей, стало важным наносить удары еще сильнее. Итак, мы~восхищаемся объективностью доводов, которые мы~создали? Но~в чем мы~хотим быть правыми для того, чтобы наносить удары столь сильно и~жестоко? Нужны ли~тому, кто не~желает никого убивать факты столь же~твердые как биты? 

	Как насчет того, чтобы совершить такую странность и~последовать за~вещами туда, куда они нас приведут? Кто может сказать по~совести, что сейчас больше людей, которые бы~были заинтересованы в~том, чтобы фланировать вдоль {\itshape их} пути, чем это было в~прошлом? Сделать такое означает, что мы~слабы, а~не~сильны. Это означает отъезд без мысли возвращении. Или, если мы~все-таки вернемся, это значит, что мы~придем с~пустыми руками; без добычи, трофеев, коллекций, статей или диссертаций. Можем ли~мы по~совести сказать, что мы~видели больше людей, которые ведут себя таким образом? 

	Идеалисты были правы: мы~можем знать, только в~той мере, в~которой мы~приближаем вещи к~нам самим. Но~они забыли добавить, что вещи необходимо собрать вместе, чтобы свергнуть нас. Крылатые ракеты вращаются по~орбите Левиафанов и~рано или поздно падают, чтобы произвести впечатляющий побочный эффект. Коперниковская революция была совершена посредством игнорирования всего остального, и~потеряно было практически все. Нам осталась лишь магия~--- наука и~колдовство, будущие войны определенное количество восхитительного знания добытого, вопреки нам всем, на~пересечении антропоморфизма и~объективности.

 Я~не~говорю этого, потому что я~хочу потопить нашу единственную спасательную шлюпку. Я~говорю это, потому что я~хочу предотвратить кораблекрушение, или, если уже поздно, сделать возможным выжить после кораблекрушения.
	\end{itemize}	



\paragraph{4.7.1}\hypertarget{par:4.7.1}{} Поскольку существуют только узы слабости, нет двух способов обучения~--- один их~которых академический, человеческий, рациональный или нововременной, а~другой народный, естественный, дезорганизованный или древний. Есть только один способ. Мы~всегда учимся одним и~тем же~способом, без того чтобы идти напрямик, предвидеть или когда-либо покидать сети, которые мы~выстроили. Мы~совершаем каждую ошибку столько раз, сколько необходимо, чтобы продвинуться от~одного пункта к~другому. Мы~никогда не~сможем делать что-то лучше (\hyperlink{par:1.2.1}{1.2.1}). 
 Мы~никогда не~сможем идти быстрее. Мы~никогда не~будет видеть яснее. 

	\begin{itemize}
	\item 
	Науки всегда критиковали от~имени более высоких форм знания, которые являются более интуитивными, непосредственными, человеческими, глобальными, теплыми, развитыми, духовными или искусными. Мы~всегда хотели критиковать науку, утверждая, что некая альтернатива лучше, добавляя апелляционный суд к~суду первой инстанции, обращаясь 

	Богу богов за~тем, чтобы он~уязвил гордость ученых и~приберег тайну вещей для скромных и~смиренных. Но~нет знания более высокого, чем знание наук, потому что нет шкалы знания и, в~конечном счете, вообще знания. Нам следует растворить все споры об~<<уровнях знания>> в~низшей форме знания, единственной форме имеющейся у~нас. Не~метафизика, а~инфрафизика. Как мы~уже сказали, нам никогда не~удастся возвыситься над необузданным политиканством (\hyperlink{par:3.6.0}{3.6.0}).
	\end{itemize}	


\paragraph{4.7.2}\hypertarget{par:4.7.2}{} Нет такой вещи как высшее и~низшее знание. Если мы~вообще хотим сохранить эти термины, мы~должны сказать, что некоторые формы знания <<выше>> чем другие, потому что обладатели высшего знания вознесли себя при молчаливом попустительстве обладателей низшего знания (\hyperlink{par:4.4.0}{4.4.0}).


\paragraph{4.7.3}\hypertarget{par:4.7.3}{} Холодны ли~<<науки>>? Строги? Бесчеловечны? Объективны? Скучны? Аполитичны? Нововременны? Эти недостижимые качества просто приписываются им~их врагами, которые таким образом надеются их~заклеймить (Интерлюдия IV). Горячие? Беспорядочные? Неистовые? Антропоморфные? Антропоцентричные? Пристрастные? Дикие? Мифичные? Нет, эти термины не~описывают их~также. Разреженные и~хрупкие и~в первую очередь разреженные. Каков их~специфический знак? Отсутствие особых примет.
	\begin{itemize}
	\item 
 Я~упрекаю эти плохо изученные агрегаты, которые мы~называем науками, не~за то, что они слишком рациональны, но~скорее за~не понимание природы своих природ. Давайте редуцируем их~к тем сферам, которые они занимают, и~наконец-то избавимся от~магии. С~самого начала эпистемология следовала за~науками по~пятам, пытаясь быть: ПЕРИ-, МЕТА-, ПАРА-, ИНФРА-, СУПРА-научной. Но~это не~понимание сути предмета. Политика, несомненно, все еще лучшая модель испытаний слабостей, и~никогда еще не~была столь подходящей как, когда мы~обнаружили, что исследователь ведет себя как делегат молчаливых толп атомов, микробов или звезд. В~этом случае мы~видим исполнителя, законодателя и~судью, который слишком долго уклонялся даже от~самых элементарных форм демократии.
	\end{itemize}	

\paragraph{4.7.3.1}\hypertarget{par:4.7.3.1}{} Те~из нас, кто хочет конвоировать <<науки>> обратно к~их~должному месту обитания, больше рационалисты, чем большинство ученых людей, которые хотят распространить их~<<вдвойне>>. Мы, по~крайней мере, знаем о~стоимости работы, включенной в~умножение тех мест обитания.
	\begin{itemize}
	\item 
	Гностики должны понять правильно: я~не~пытаюсь сделать их~жизнь легче.
	\end{itemize}	

\paragraph{4.7.4}\hypertarget{par:4.7.4}{} Коль скоро не~существует другого мира, совершенство обитает в~этом. Абсолютное знание обнаруживается в~этом мире, коль скоро не~существует более уровней знания. Те~же самые люди, которые устанавливают уровни знания, оказываются потом теми, кто впадает в~отчаяние от~невозможности достичь вершины: те~же самые редукционисты, которые попеременно то~опьянены властью, то~ослаблены немощью, оказываются то~высокомерными, то~скромными. Все испытания силы являются цельными, законченными и~точными именно в~той степени, в~которой это возможно. {\itshape Они не~приблизительны}. Не~являются они и~смутными, конвенциональными или субъективными. До~тех пор пока не~будут установлены новые отношения силы, они не~хороши и~не~плохи. Совсем не~утратив определенности, мы, в~конце концов, обнаружили, что привело к~иллюзии знания без неопределенности.


\paragraph{4.7.5}\hypertarget{par:4.7.5}{} Поскольку существует не~два способа познания, а~только один, существуют с~одной стороны те, кто склоняется перед силой аргумента, а~с другой~--- те, кто понимает только насилие. Демонстрации~--- это всегда демонстрации силы (\hyperlink{par:3.1.8}{3.1.8}), а~линии силы это всегда мера реальности, ее~единственная мера (\hyperlink{par:1.1.4}{1.1.4}). Мы~никогда не~преклоняемся перед разумом, но~скорее перед силой.


\paragraph{4.7.6}\hypertarget{par:4.7.6}{} Веря в~обратное, мы~позволяем определенным линиям силы и~определенным аргументам господствовать в~сетях, к~которым они собственно принадлежат. Мы~создаем могущество (\hyperlink{par:1.5.1}{1.5.1}), и~таким образом ослабляем других. 
	\begin{itemize}
	\item 
	Существуют, говорят они, благоразумные люди, которые уступают только силе аргумента, и~остальные, которые неблагоразумны, и~которые слепо подчиняются силе без понимания. Мне никогда не~приходилось встречать кого-либо, кто не~презирал бы~неблагоразумных людей, и~кто не~верил бы~в то, что это презрение воплощает благодетель.
	\end{itemize}

\paragraph{4.7.7}\hypertarget{par:4.7.7}{} Коль скоро <<правота>> отделяется от~<<могущества>>, или <<разум>> от~<<силы>>, правота и~разум оказываются ослабленными, потому что мы~больше не~понимаем их~слабости, мы~постепенно овладеваем единственным способом стать справедливым благоразумным, который доступен для презираемых. Эти две утраты освобождают поле для нечестивцев. Я~называю это преступлением, единственным преступлением, которое нам необходимо в~этом эссе.
	\begin{itemize}
	\item 
	Человек, который уступает твердости маленького аргумента только после сотен испытаний и~тестов, ошибок и~починок в~его лаборатории, тем не~менее, утверждает, что те, кого тестировали и~испытывали, ничего не~понимают и~размышляют как болваны. Несмотря на~то, что он~не умеет говорить ясно, в~тот момент, когда он~выходит из~двери своей лаборатории, он~в возмущении обнаруживает, что <<не каждый понимает такой простой аргумент>>. Его возмущение подпитывает презрение. Поскольку он~презирает дураков вокруг него, он~забывает об~одной вещи, которая заставляет его уступать силе аргумента: его лаборатории, месте, где он~сам подвергается испытаниям. Это порочный круг. Чем более глупы другие, то~больше он~верит в~то, что он~может <<думать>> и~тем меньше он~состоянии увидеть то, как он~научился. Чем больше он~расширяет могущество разума за~пределы силы, тем более ослабленным оказывается разум.
	\end{itemize}


\paragraph{4.7.8}\hypertarget{par:4.7.8}{} Противопоставление правоты и~могущества преступно, потому что оно расчищает поле для нечестивцев, делая вид в~это время, что защищает его при помощи силы справедливого. Но~то, что является правым, лишено силы кроме как <<в принципе>>. Не~будучи в~состоянии гарантировать, что то, что является правым, является сильным, люди вели себя так, будто бы~то, что было сильным, было порочным. Сильные просто заняли место, которое освободили те, кто без задней мысли презирали их.
	\begin{itemize}
	\item 
 В~результате понятной перестановки Макиавелли и~Спинозу стали считать аморальными, даже несмотря на~то, что они были правы в~том, что отказывались разделять могущество правоту. Однако настоящее изложение отличается от~<<Богословско-политического трактата>> Спинозы. Времена изменились. Экзегетика религиозных текстов теперь заменена экзегетикой <<научных>> записей. По~этой причине я~рассматриваю это эссе как <<Научно-политический трактат>>. И~все же, объект один и~тот же. Мы~все еще в~самом начале экзегезиса, и~связь между наукой и~демократией истончилась в~ходе <<научных войн>>. Подобно Спинозе мы~выглядим жестокими с~тем, чтобы быть честными.
	\end{itemize}	

\paragraph{4.7.9}\hypertarget{par:4.7.9}{} Мы~страдаем не~от нехватки души, разума, науки или справедливости, а~от~излишка всех этих дополнений, которые присоединяются к~отношениям силы, чтобы замедлить рост могущества и~сделать слабого немощным. Если бы~у слабого против них был бы~только набор слабостей, который я~описал, то~они бы~замарали руки и~трансформировали бы~его по~своему нраву. 
	\begin{itemize}
	\item 
	<<Noli me~tangere>>\footnote{Не прикасайся ко~мне {\itshape(лат.).}}~--- это слова магов, которые хотят в~одно и~тоже время быть мертвыми, и~живыми; и~там и~здесь; и~сильными и~рациональными; и~сильными хорошими.
	\end{itemize}	

\paragraph{4.7.10}\hypertarget{par:4.7.10}{}Поскольку есть только один способ познания, а~не~два~--- испытание отношений между силами~--- мы~никак не~можем избежать одной ошибки, нелепости или преступления. Мы~не можем избежать одного эксперимента или одной попытки сократить путь. Даже {\itshape думать} об~обратном~--- значит сбить себя с~толку преступными иллюзиями. 
	\begin{itemize}
	\item 
 Во~скольких ядерных войнах мы~должны повоевать прежде, чем мы~уступим силе аргумента, что это никоим образом не~может быть способом ведения наших дел? Послушайте, это очень просто. Мы~никогда не~станем лучше, чем те, кто просто убедили себя в~пустяках, имея под рукой все, что им~нужно, имея надлежащее питание, будучи хорошо одетыми и~соответствующим образом обученными. Как много ошибок они 10~Не прикасайся ко~мне  совершат прежде, чем они начнут расставаться с~мельчайшими предрассудками? Десятки, сотни, тысячи? Как много войн потребуется для того, чтобы убедить пять миллиардов мужчин и~женщин? Десять? Сто? Иначе, до~тех пор массы не~смогут думать быстрее яснее, чем те, кто находится в~лаборатории.
	\end{itemize}	

\paragraph{4.7.11}\hypertarget{par:4.7.11}{}Те, кто думают, что они могут делать что-либо лучше и~работать быстрее всегда будут делать хуже, потому что они забудут поделиться своими выдающимися средствами познания и~испытания. Они будут верить в~то, что они достаточно сделали, когда они <<распространили>> причины, коды, результаты. В~действительности, все это погибнет, как только будет извлечено из~презираемых сетей, которые позволяют им~быть сильными.
	\begin{itemize}
	\item 
	Когда Вольтер хотел высмеять религию, он~обычно подписывал свои письма <<ecrelinf>> <<истребить бесчестных>>. Религия пережила свои худшие времена, более чем худшие. Сегодня мы~обнаруживаем себя в~такой же~позиции. Мы~никогда не~смогли бы~вообразить такой источник чудес, энтузиазма, теплоты и~откровения, который бы~сравнился с~тем, что мы~вульгарно называем <<науками>>. И~тем не~менее, пока тысячелетие не~подошло концу, мы~должны подписывать наши письма словом <<ecrelinf>>. Для того, чтобы обладать знанием в~следующем тысячелетии, чтобы мочь говорить о~точности, без того, чтобы на~не обозвали облученными, мы~должны спасти знание от~<<наук>>, точно так же, как божественное было спасено от~пустой оболочки религии. При помощи божественной любви мы~должны были истребить все, что было религиозного в~нас. При помощи любви знания мы~должны выпутаться из~<<наук>>. Мы~не можем сопоставить Galileo и~крылатые ракеты, таким же~образом, каким Нагорная Проповедь столь долго противопоставлялась Инквизиции. Меня не~интересует апологетика. В~<<науке>> так же, как и~в <<религии>> более чем достаточно протестантов, мистиков, интегристов, анабаптистов, фундаменталистов светских Иезуитов. Никто из~них меня не~интересует, потому что все они хотят реформировать или обновить те~плохо сконструированные целостности, <<науки>>. Все они ищут пути примирить непримиримое, и, занимаясь этим, они делают для меня непостижимым то~единственное, что я~хочу понять. Если крылатые ракеты настигают меня в~винограднике, то~я не~хочу преклоняться перед <<разумом>>, <<ошибающейся физикой>>, <<людской глупостью>>, <<жестокостью Бога>> или <<Realpolitik>>. Я~не~хочу обращаться к~путанным объяснениям, говорящим о~могуществе, когда причина моей смерти кроется в~силе фактов. В~те~несколько секунд, которые разделяют иллюминацию от~радиации, я~хочу быть агностиком настолько, насколько это возможно для человека, который присутствует при смерти первого Просвещения, агностиком настолько, насколько это возможно для человека, который достаточно уверен в~божественном и~знании, что осмеливается верить в~рождение нового Просвещения. Я~не~уступлю им; я~не~буду верить <<наукам>> наперед; и~после я~не~потеряю веру в~знание, когда одно из~отношений силы, в~которое внесли свой вклад лаборатории, разорвется над Францией. Не~вера, и~не~отчаяние. Я~буду настолько агностиком и~настолько честным, насколько это будет возможно.
	\end{itemize}

\end{document}
