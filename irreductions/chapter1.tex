\chapter{От слабости к~могуществу (потенции)}\hypertarget{chap1}{}

\paragraph{1.1.1}\hypertarget{par:1.1.1}{} Ничто само по себе не~является редуцируемым или нередуцируемым к~чему-либо еще. 
	\begin{itemize}
	\item {Я назову это <<принципом нередуцируемости>>, но это государь, который не~правит, так как это было бы~противоречием самому себе (\hyperlink{par:2.6.1}{2.6.1})}.
	\end{itemize}

\paragraph{1.1.2}\hypertarget{par:1.1.2}{} Существуют только испытания сил, слабостей. Или проще~--- существуют только испытания. Вот пункт моего отправления: глагол, <<испытывать>>.

\paragraph{1.1.3}\hypertarget{par:1.1.3}{} Поскольку ничто само по себе не~является редуцируемым или нередуцируемым к~чему-либо еще, постольку существуют только испытания (сил, слабостей). То, что не~является ни~редуцируемым и~ни нередуцируемым должно быть проверено, сосчитано и~измерено. Другого пути нет.

\paragraph{1.1.4}\hypertarget{par:1.1.4}{} Все может быть сделано мерой всего остального.

\paragraph{1.1.5}\hypertarget{par:1.1.5}{} Реально то, что сопротивляется испытаниям.
	\begin{itemize}
	\item Глагол <<сопротивляться>> не~является привилегированным словом. Он используется для обозначения всей совокупности глаголов и~прилагательных, орудий и~приборов, которые вместе определяют способы быть реальным. Мы равным образом можем сказать <<свертываться>>, <<сгибаться>>, <<загораживать>>, <<обострять>>, <<скользить>>. Существует множество альтернатив.
	\end{itemize}

\paragraph{1.1.5.1}\hypertarget{par:1.1.5.1}{} Реальное не~вещь среди других вещей, но скорее градиент (уровень, шкала) сопротивления.

\paragraph{1.1.5.2}\hypertarget{par:1.1.5.2}{} Не существует различия между <<реальным>> и~<<нереальным>>, <<реальным>> и~<<возможным>>, <<реальным>> и~<<воображаемым>>. Скорее все это различия между теми, кто сопротивляется длительное время и~теми, кто этого не~делает, между теми, кто отважно сопротивляется и~теми, кто нет, между теми, кто знает, как вступить в~союз или изолировать себя, и~теми, кто этого не~знает.

\paragraph{1.1.5.3}\hypertarget{par:1.1.5.3}{} Ни одна сила не~может, как это часто говорится, <<познавать реальность>> кроме как через различия, которые она создает в~сопротивлении другим силам.

\paragraph{1.1.5.4}\hypertarget{par:1.1.5.4}{} Ничто не~может быть познано~--- только сделано реальным (realized).

\paragraph{1.1.6}\hypertarget{par:1.1.6}{} Форма это линия фронта (передовая) испытания сил, которая де-формирует, транс-формирует, ин-формирует и~пер-формирует его. Конечно, стоит форме стабилизироваться, она уже более не~выглядит как испытание сил.

\paragraph{1.1.7}\hypertarget{par:1.1.7}{} Что такое сила? Кто это? На что она способна? Это субъект, текст, объект, энергия или вещь? Как много сил существует? Кто силен, а~кто слаб? Это битва? Это игра?
Это рынок? Все эти вопросы определяются и~деформируются в~последующих
испытаниях.
	\begin{itemize}
	\item Вместо <<силы>> можно говорить о~<<слабостях>>, <<энтелехиях>>, <<монадах>>, или проще~--- <<актантах>>.
	\end{itemize}

\paragraph{1.1.8}\hypertarget{par:1.1.8}{} Ни один актант не~является настолько слабым, что не~может завербовать другого. Тогда двое объединяются вместе и~становятся одним для третьего актанта, которого поэтому им легче привести в~движение. Образовался небольшой вихрь, и~он растет, пополняясь многими другими.
\begin{itemize}
	\item Является ли актант сущностью (essense) или отношением? Мы не~можем сказать без испытания (\hyperlink{par:1.1.5.2}{1.1.5.2}). Для того, чтобы избежать уничтожения, сущности могут связать себя отношениями со многими союзниками, а~отношения [связать себя] со многими сущностями.
	\end{itemize}

\paragraph{1.1.9}\hypertarget{par:1.1.9}{} Актант может обрести силу только путем ассоциации с другими. Таким образом он говорит от их имени. Почему другие не~говорят сами за себя? Потому что они немы, потому что их заставили замолчать, потому что, говоря в~одно и~тоже время, они стали неслышны. Поэтому некто интерпретирует их и~говорит вместо них. Но кто? Кто говорит? Они или он? {\itshape Traditore~--- traduttore}. Один равняется нескольким. Это невозможно определить. Если лояльность актанта ставится под сомнение, он может продемонстрировать, что он только лишь повторяет то, что хотели сказать другие. Он предлагает экзегетику расстановки сил, которая не~может быть оспорена даже временно без другого альянса.

\paragraph{1.1.10}\hypertarget{par:1.1.10}{} Действуй так, как тебе этого хочется до тех пор, пока нельзя будет легко повернуть вспять. В результате работы актантов, определенные вещи не~возвращаются в~их исходное состояние. Форма сложена, как сгиб. Это можно назвать ловушкой, храповником, необратимостью, демоном Максвелла, реификацией. Конкретное слово не~имеет значения до тех пор, пока оно обозначает асимметрию. Тогда вы
уже не~можете действовать так, как Вы этого хотите. Существуют победители и~проигравшие, существуют приказы, и~некоторые из них обладают большей силой, чем другие.

\paragraph{1.1.11}\hypertarget{par:1.1.11}{} Все пока еще поставлено на~карту. Однако, так как много игроков пытаются сделать игру необратимой и~делают все, что они могут для того, чтобы гарантировать, что все не~является одинаково возможным, игра окончена.
	\begin{itemize}
	\item Мое почтение Мастерам Го\footnote{{\itshape Yasunari Kawabata}. The Master of Go (New York: Alfred Knopf, 1972).}.
	\end{itemize}

\paragraph{1.1.12}\hypertarget{par:1.1.12}{} Для того, что создать асимметрию актанту необходимо опереться на~силу чуть более прочную, чем он сам. Даже если это различие невелико, его достаточно для того, чтобы создать градиент сопротивления, который сделает их более реальными
для другого актанта~(\hyperlink{par:1.1.5}{1.1.5}).

\paragraph{1.1.13}\hypertarget{par:1.1.13}{} Мы не~можем сказать, что актант следует правилам, законам или структурам, но мы также не~можем сказать, что он действует без всего этого. Извлекая уроки из того, что делают другие актанты, он постепенно разрабатывает правила, законы и~структуры. Затем он стремится заставить других играть по этим правилам, которые
как он утверждает, он выучил, соблюдал или принял. Если он побеждает, то тогда он верифицирует эти правила и~таким образом применяет их. 
	\begin{itemize}
	\item Является ли какой-либо порядок конвенцией, социальной конструкцией, законом природы или творением человеческого духа? Мы не~можем сказать. Но в~любви, как и~на~войне, все средства хороши в~попытке приписать правила чему-то более прочному, чем породившему их моменту.

	\end{itemize}

\paragraph{1.1.14}\hypertarget{par:1.1.14}{} Ничто само по себе не~является упорядоченным или не~упорядоченным, единственным или множественным, гомогенным или гетерогенным, текучим или инертным, человеческим или нечеловеческим, полезным или бесполезным. Никогда само по себе, но всегда посредством других.
	\begin{itemize}
	\item Спиноза сказал давным-давно: до тех пор, пока {\itshape формы} имеют значение, давайте не~будем антропо{\itshape{морфичными}}. Каждая слабость распределяет полный набор ролей. В зависимости от того, чего она ожидает от других, она отделяет стабильное и~упорядоченное от бесформенного и~движущегося. Но поскольку все другие также распределяют роли, получается красивая путаница. Тем не~менее, понятно почему энтелехии могут принять за бесформенную материю тех, кого они сломали, разорвали на~части или соблазнили.

	\end{itemize}

\paragraph{1.1.14.1} Порядок возникает не~из беспорядка, а~из приказов.
	\begin{itemize}
	\item Мы всегда совершаем одну и~ту же~ошибку. Мы проводим различия между варварским и~цивилизованным, сконструированным и~разобранным (dissolved), упорядоченным и~беспорядочным. Мы всегда оплакиваем упадок и~разложение морали. Скатертью дорога! Аттила говорит на~греческом и~латыни, панки одеваются с такой же~тщательностью, как и~Коко Шанель; у бактерий чумы такие же~искусные стратегии, как и~у IBM; представители племени Азандэ фальсифицируют свои верования с попперовским смаком. Не важно как далеко мы заберемся, там всегда уже будут формы; в~каждой рыбе пруды полные рыбы. Некоторые верят в~то, что они лекало, в~то время как другие сырой материал, но это форма элитизма. Для того чтобы завербовать силу мы должны сговориться с ней. Она не~может быть выкована как листовое железо или отлита как гипсовый слепок (по образцу).

	\end{itemize}

\paragraph{1.1.15}\hypertarget{par:1.1.15}{} Утверждения <<все детерминировано>> и~<<все контингентно>> означают одно и~то же~--- то есть ничего. Слова <<детерминированный>> и~<<контингентный>> приобретают
значение, только когда используются в~разгар описания градиентов сопротивления~--- то есть, реальности.
	\begin{itemize}
	\item Длина носа Клеопатры не~является ни~значительной, ни~незначительной. Обстоятельства определяют, на~время, относительную важность того, что их составляет. Роли случая и~необходимости не~могут быть определены заранее.

	\end{itemize}

\paragraph{1.1.16}\hypertarget{par:1.1.16}{} Что одинаково и~что различно? Что с кем? Что противопоставлено или соединено или близко? Что продолжается, останавливается, отвергается, торопится или привязывается? Это общие вопросы, да, общие для всех испытаний прислуживаемся ли мы, пробуем на~вкус, распутываем, плетем, присоединяемся, стираем или же~обращаемся.

\paragraph{1.2.1}\hypertarget{par:1.2.1}{} Ничто само по себе не~является таким же~или отличным от чего-либо еще. То есть, не~существует эквивалентов, только переводы.

Другими словами все происходит только один раз и~в~одном месте.

Если между актантами существуют тождества, то это потому, что они были сконструированы за высокую плату. Если эквивалентности существуют, то это потому, что они были с большим трудом выстроены из разрозненных элементов, и~поддерживаются с помощью силы. Если существуют обмены, то они всегда неравные и~стоят целого состояния для того, чтобы устраивать равно, как и~поддерживать их.
	\begin{itemize}
	\item Я называю это <<принципом относительности>>. Также как для одного наблюдателя невозможно общаться с другим быстрее скорости света, лучшее, что может быть сделано с актантами~--- это перевод одного актанта в~другой. Между несоизмеримыми и~нередуцируемыми силами нет ничего: ни~эфира, ни~непосредственности. Верно то, что данный принцип относительности имеет своей целью восстановить неэквивалентность актантов, в~то время как другой принцип был создан для того, чтобы возродить эквивалентность всех наблюдателей. В обоих из них, однако, мы должны привыкнуть к~тому, чтобы дышать в~отсутствии эфира. Вещи, о~которых я говорю, редки, разбросаны и~преимущественно пусты. Скопления, проявления насыщенности и~полноты редки и~разбросаны, как большие города на~карте страны.

	\end{itemize}

%\subparagraph {Интерлюдия 1: Объяснить в~псевдоавтобиографическом стиле цели автора.}

\paragraph{1.2.2}\hypertarget{par:1.2.2}{} Энтелехии ни~о чем не~договариваются и~могут договориться обо всем, поскольку ничто по своей природе и~без связи с чем-либо еще не~является ни~соизмеримым, ни~несоизмеримым. Каким бы~ни было согласие, всегда будет нечто, что может дать пищу для разногласий. Какой бы~ни была дистанция, всегда будет нечто, на~чем может быть основано согласие. Иными словами, все может стать предметом переговоров.
	\begin{itemize}
	\item <<Переговоры>> не~плохое слово пока подразумевается, что все может стать предметом переговоров, а~не~только форма стола и~имена делегатов. Также необходимо принять решения о~том, что является предметом переговоров, когда можно будет сказать, что они начались или завершились, на~каком языке они будут проводиться, и~каким образом будет определяться то, были ли поняты или нет. Была ли это битва, церемония, дискуссия или игра? Это тоже предмет спора, спора, который продолжается до тех пор, пока все энтелехии не~будут определены и~не~определят других сами. Для изображения этих переговоров мне~нужно <<Поле золотой парчи>>.
	\end{itemize}

\paragraph{1.2.3}\hypertarget{par:1.2.3}{} Сколько существует актантов? Это невозможно определить до тех пор, пока они не~будут измерены друг относительно друга.
	\begin{itemize}
	\item Я еще не~сказал как много нас: 50 миллионов французов, одна экосистема, 20 миллиардов нейронов, три ли четыре типа характеров, один <<me, I, me, I>>. Мы не~можем сосчитать число сил, решить, что существует единственная субстанция, два социальных класса, три хариты, четыре стихии, семь смертных грехов или двенадцать апостолов. Мы не~можем подсчитать итог. В этой своеобразной арифметике никто никогда не~вычитает. Мы прибавляем столько промежуточных сумм, сколько существует счетоводов.
	\end{itemize}

\paragraph{1.2.3.1}\hypertarget{par:1.2.3.1}{} Не существует ни~целых, ни~частей. Также как не~существует ни~гармонии, ни~структуры, ни~интеграции, ни~системы (\hyperlink{par:1.1.14}{1.1.14}). То, каким образом нечто сохраняется
(hold together) определяется на~поле битвы, поскольку никто не~согласен относительно того, кто будет командовать, а~кто подчиняться, кто будет частью, а~кто целым. 
	\begin{itemize}
	\item Предустановленной гармонии, несмотря на~Лейбница, не~существует, гармония устанавливается после локально и~посредством экспериментальной починки.
	\end{itemize}

\paragraph{1.2.4}\hypertarget{par:1.2.4}{} Мы не~знаем, где может быть найден какой-либо актант. Определение его положения это изначальная борьба, в~ходе которой многое может быть потеряно. Мы можем сказать, что некто/нечто определяет местоположение, а~другим определяют место.

\paragraph{1.2.4.1}\hypertarget{par:1.2.4.1}{} Хотя места удалены, нередуцируемы и~несовместимы, их, тем не~менее, связывают
вместе, объединяют, складывают, выравнивают и~подвергают всевозможным
операциям. Если бы~не~эти операции, ни~одно место не~вело бы~к другому.

\paragraph{1.2.5}\hypertarget{par:1.2.5}{} Силы, которые вступают в~союз в~ходе испытания, считаются надежными. Каждая энтелехия создает времена для других, вступая с ними в~союз или предавая их. <<Время>> возникает в~конце этой игры, игры в~которой большинство проиграло то, что поставило.
Мгновение до или мгновение после? Является ли нечто/некто одержимым, пророческим, устаревшим, упадочным, современным, временным или вечным? Это невозможно определить заранее. Об этом необходимо договориться.

\paragraph{1.2.5.1}\hypertarget{par:1.2.5.1}{} Время это отдаленные последствия стремлений каждого из акторов создать необратимые события в~собственных интересах (\hyperlink{par:1.1.10}{1.1.10}). Таким образом проходит время.

\paragraph{1.2.5.2}\hypertarget{par:1.2.5.2}{} Время не~проходит. Времена являются ставкой в~борьбе между силами. Конечно, одна сила может овладеть другими, но это может быть только локально и~временно, потому что постоянство стоит слишком дорого и~требует слишком много союзников.

\paragraph{1.2.5.3}\hypertarget{par:1.2.5.3}{} Во Франции часто говорится, что <<существуют>> революции, но они лишь акторы, которые выиграли время и~историю у других акторов и~таким образом обошли других, и~сделали их устаревшими. Конечно, побежденные иногда добиваются отмщения и, таким образом, опрокидывают ход времен еще раз.
	\begin{itemize}
	\item Кто тогда более современен~--- Шах; Хомейни, Мусульманин из другой эпохи; или Бани-Садр, Президент, который нашел убежище в~Париже? Никто не~знает, и~именно поэтому они так много борются за то, чтобы прийти в~свое время (to make their time).
	\end{itemize}

\paragraph{1.2.5.4}\hypertarget{par:1.2.5.4}{} Свободнейшая из всех демократий властвует между мгновениями. Ни одно мгновение не~может короновать, калечить, оправдать, заменить или ограничить какое-либо другое. Не существует последнего мига, чтобы осудить все те, что были до этого.
	\begin{itemize}
	\item Времена нередуцируемы и~поэтому <<смерть>> всегда оказывалась побежденной. Цель не~оправдывает средства. Также как и~смерть не~отрицает (condemn) жизнь.
	\end{itemize}

\paragraph{1.2.6}\hypertarget{par:1.2.6}{} Пространство и~время не~создают (frame) энтелехии. Они лишь становятся системой координат (framework) описания тех актантов, которые подчинились, локально и~временно, гегемонии другого. 
	\begin{itemize}
	\item Следовательно, существует время времен и~пространство пространств и~так далее до тех пор, пока все не~будет оговорено (has been negotiated). Мое почтение <<Клио>> Шарля Пеги\footnote{{\itshape Charles P{\'e}guy}. Clio: dialogues de l'{\^a}me pa{\"\i}enne et l'{\^a}me charnelle (Paris: Ple{\"\i}ade, 1914).}.
	\end{itemize}

\paragraph{1.2.7}\hypertarget{par:1.2.7}{} Каждая энтелехия определяет: что находится внутри, а~что снаружи; каким другим акторам она поверит, когда она решит, что принадлежит ей, а~что нет; и~какие испытания она будет использовать для того, чтобы решить верить или нет этим третейским судьям. 
	\begin{itemize}
	\item Лейбниц был прав, сказав, что у монад нет ни~дверей, ни~окон, поскольку они никогда не~выходят за пределы самих себя. Однако, они болтуны, поскольку они без конца ведут переговоры о~своих границах, о~том, кто будет вести переговоры, и~что они должны делать. В конечном итоге они как химеры, неспособные определить, где дверь, а~где окно, где право, а~где лево.
	\end{itemize}

\paragraph{1.2.7.1}\hypertarget{par:1.2.7.1}{} Не существует внешнего референта. Все референты всегда внутренние по отношению к~силам, которые используют их в~качестве критериев.

\paragraph{1.2.7.2}\hypertarget{par:1.2.7.2}{} Принцип реальности~--- это другие (люди).
	\begin{itemize}
	\item Интерпретация реального не~может быть отличена от реального как такового, потому что реальное это градиенты сопротивления (\hyperlink{par:1.1.5}{1.1.5}). Актант следовательно не~прекращает вести переговоры о~числе, градиентах и~природе этих различий; о~числе, авторитете и~весе тех, кто ведет переговоры; числе, качестве и~надежности критериев, которые они используют для того, чтобы оценить степень доверия судьям.
	\end{itemize}

\paragraph{1.2.8}\hypertarget{par:1.2.8}{} Каждая энтелехия создает для себя целый мир. Она определяет место себе и~всем остальным; она решает, из каких сил она состоит; она создает свое собственное время; она указывает того, кто будет первопричиной реальности этого мира. Она переводит все остальные силы в~своих интересах, и~стремится заставить их принять текст о~ней самой, который она бы~хотела, чтобы они перевели.
	\begin{itemize}
	\item Ницше называл это <<оценкой>>, а~Лейбниц <<выражением>>.
	\end{itemize}

\paragraph{1.2.9}\hypertarget{par:1.2.9}{} Это сила, о~которой мы говорим? Это говорит сила? Это актор, которого заставил говорить другой? Это интерпретация или сам объект? Это текст или мир? Мы не~можем сказать, потому что это то, за что мы боремся, построение целого слова (мира? word).
	\begin{itemize}
	\item То, что те, кто используют герменевтику, экзегетику или семиотику говорят о~текстах, может быть сказано обо всех слабостях. Долгое время было принято считать, что отношение между одним текстом и~другим всегда является вопросом интерпретации. Почему бы~не~признать, что это верно также в~отношениях между так называемыми текстами и~так называемыми объектами и~даже между самими так называемыми объектами?
	\end{itemize}

\paragraph{1.2.10}\hypertarget{par:1.2.10}{} Ничто не~может избежать примордиальных испытаний. До переговоров у нас никакого представления о~том, каковы будут испытания~--- можно ли помыслить как конфликт, игру, любовь, историю, экономику или жизнь. Мы также не~знаем примордиальны ли они или вторичны до того, как мы прибудем на~место действия. В конце концов, мы не~можем сказать до окончания [переговоров], являются ли они результатом переговоров, унаследованы от рождения, или выскочили на~коже сами по себе.

\paragraph{1.2.11}\hypertarget{par:1.2.11}{} Мы не~должны верить заранее, что мы знаем, говорим ли мы о~субъектах или объектах, людях или богах, животных, атомах или текстах. Я еще не~сказал, поскольку это именно то, что поставлено на~карту между силами: кто говорит и~о чем?
	\begin{itemize}
	\item Нам не~следует спешить отделять <<природу>> от <<культуры>>. Морские гребешки также считают, что природа суровый патрон~--- враждебный, питательный, расточительный~--- потому, что у рыбы, рыбаков, камней к~которым они прикрепляются цели иные, чем у морских гребешков.
	\end{itemize}

\paragraph{1.2.12}\hypertarget{par:1.2.12}{} Ничто само по себе не~является познаваемым или непознаваемым, высказываемым или невысказываемым, далеким или близким. Все переводится. Что может быть проще?

\paragraph{1.2.13}\hypertarget{par:1.2.13}{} Если все о~чем мы должны написать подлежит обсуждению и~переводу, тогда нам нужна, как говорил Декарт, временная мораль. Когда мы говорим об испытаниях сил, мы должны избегать использовать термины, которые закрепляют отношения в~пользу одной или другой стороны. Если это невозможно нам, по крайней мере, следует попытаться написать текст, который не~забирает время и~пространство, но вместо этого предоставляет их.

\paragraph{1.3.1}\hypertarget{par:1.3.1}{} Все энтелехии могут измерять и~быть мерой всех других энтелехий (\hyperlink{par:1.1.14}{1.1.14}). Однако некоторые силы постоянно пытаются измерять, нежели быть измеряемыми и~переводить, нежели быть переведенными. Они хотят действовать, нежели испытывать действие. Они хотят быть сильнее, чем другие.
	\begin{itemize}
	\item Я сказал <<некоторые>>, а~не~<<все>> как в~ницшевском воинственном мифе. Большинство актантов слишком разрозненны или безразличны для того, чтобы бросить вызов, слишком недисциплинированны или сбивчивы (devious), чтобы долгое время следить за теми, кто говорит от их имени; и~слишком счастливы или горды для того, чтобы командовать другими. В этой работе я говорю только о~тех слабостях, которые хотят увеличить свою силу. Другим нередуцируемым нужны скорее поэты, чем философы.
	\end{itemize}

\paragraph{1.3.2}\hypertarget{par:1.3.2}{} Учитывая, что актанты несоизмеримы, и~каждый создает мир столь же~обширный и~завершенный, как и~любой другой, как получается, что они становится больше, чем другой? Посредством заявления о~том, что он равняется нескольким (to be several), посредством ассоциирования (\hyperlink{par:1.1.9}{1.1.9}).

\paragraph{1.3.3}\hypertarget{par:1.3.3}{} Поскольку ничто по своей природе и~безотносительно к~чему-либо еще не~является эквивалентным или неэквивалентным (\hyperlink{par:1.2.1}{1.2.1}), две силы не~могут вступать в~ассоциацию без недопонимания. 
	\begin{itemize}
	\item Договор, соглашение, компромисс, переговоры, схема, комбинация~--- все эти термины могут быть использованы. Те, кто находит их унизительными и~верит в~то, что они вступают в~противоречие с более совершенными формами ассоциаций, не~могут понять, что невозможно сделать лучше, как потому, что не~существует эквивалентностей (\hyperlink{par:2.2.1}{2.2.1}), так и~потому что, ничто само по себе не~является редуцируемым или нередуцируемым к~чему-либо еще (\hyperlink{par:1.1.1}{1.1.1}).
	\end{itemize}

\paragraph{1.3.4}\hypertarget{par:1.3.4}{} Хотя энтелехии <<одинаково>> активны, они могут представать в~двух состояниях: доминирующими или доминируемыми, действующими или претерпевающими действие. Для того чтобы энтелехию можно было назвать пассивной, ей достаточно (необходимо) лишь не~отвечать.
	\begin{itemize}
	\item Я не~утверждаю, что существуют активные силы и~силы пассивные, но только то, что одна сила может действовать так, как если бы~другая была пассивна и~покорна (\hyperlink{par:1.1.14}{1.1.14}). Точка зрения пассивной силы, конечно же, совершенно другая. Существует тысяча причин для того, чтобы симулировать покорность, десять тысяч~--- для того, чтобы желать над собой власти и~сто тысяч~--- для того, чтобы оставаться безмолвным. Причины, о~которых никогда не~подозревают те, кто верит в~то, что им служат.
	\end{itemize}

\paragraph{1.3.5}\hypertarget{par:1.3.5}{} Так как какой-либо актант может стать больше, чем другой, только будучи одним [состоящим] из нескольких, и~поскольку данная ассоциация всегда является непониманием, тот, кто определяет природу ассоциации, не~встречая возражений, берет управление в~свои руки. 
	\begin{itemize}
	\item Когда две силы провозглашают собственное единство, говорит только одна; когда две силы осуществляют обмен, который они полагают равным, она всегда решает, кто определяет вещи для обмена, каким образом измеряется качество, и~где должен состояться обмен.
	\end{itemize}

\paragraph{1.3.6}\hypertarget{par:1.3.6}{} Поскольку ничто не~является эквивалентным, быть сильным означает сделать эквивалентным то, что таковым не~было. Таким образом несколько действуют как один.
	\begin{itemize}
	\item <<Дозволено не~все>>. Дискурсы и~ассоциации не~эквивалентны, потому что союзники и~аргументы завербованы в~точности так, чтобы одна ассоциация оказалась сильнее другой. Если все дискурсы представляются эквивалентными, если все выглядит так, что существуют <<языковые игры>> и~ничего больше, тогда кто-то был неубедительным. Это слабое место релятивистов. Они говорят только о~силах, которые не~в состоянии объединиться с другими для того, чтобы убедить и~победить. Повторяя <<все дозволено>>, они не~замечают работы, которая создает неэквивалентности и~асимметрию (\hyperlink{par:1.1.11}{1.1.11}).
	\end{itemize}

\paragraph{1.3.7}\hypertarget{par:1.3.7}{} Поскольку ничто не~является соизмеримым или несоизмеримым (\hyperlink{par:1.1.4}{1.1.4}), более активен тот, кому удается определять механизмы измерения.
	\begin{itemize}
	\item Существуют {\itshape акты} дифференциации и~идентификации, а~не~различия и~тождества~(\hyperlink{par:1.1.16}{1.1.16}). Слова <<тот же~самый>> и~<<другой>> являются следствием испытания сил, поражений и~побед. Они не~могут сами по себе описывать эти связи.
	\end{itemize}

%\subparagraph{Интерлюдия II: Показывающая какое это облегчение перестать редуцировать вещи.}

\paragraph{1.4.1}\hypertarget{par:1.4.1}{} Некоторые актанты испытывают свою силу против других, объявляют их пассивными и~заключают с ними союз, который они сами же~определяют. Навязывая эквивалентности, которыми они управляют, они распространяются шаг за шагом от пассивного актора к~пассивному актору.
	\begin{itemize}
	\item Мы слишком часто склонны начинать с <<обменов>>, <<равенств>>, <<трансферов>> эквивалентов. Но мы никогда не~говорим о~предварительной работе, в~ходе которой эти эквиваленты были изобретены. Это как если бы~мы говорили дорожных сетях, но никогда о~гражданском строительстве. Однако, разница между эквивалентом и~тем, чтобы сделать нечто эквивалентным такая же, как между вождением автомобиля и~строительством шоссе.
	\end{itemize}

\paragraph{1.4.2}\hypertarget{par:1.4.2}{} Когда одна слабость вербует другие, она формирует сеть до тех пор пока она в~состоянии удерживать привилегию определять их ассоциацию. 
	\begin{itemize}
	\item В сети некоторые очень отдаленные точки могут обнаружить себя соединенными, в~то время как другие, которые были соседями, удалены далеко друг от друга. Хотя каждый актор локален, он может перемещаться от места к~месту, по крайней мере, до тех пор, пока он способен договариваться об эквивалентностях, которые делают одно место таким же, как другое. Сеть, таким образом, может быть <<в действительности повсеместной>> не~становясь при этом <<универсальной>>. Какой бы~разреженной и~витиеватой сеть ни~была, она, несмотря на~это, остается локальной и~ограниченной, тонкой и~хрупкой, с вкраплениями пустоты (interspersed by space). Мы должны представить волокнистые энтелехии, растянутые и~переплетенные друг с другом (\hyperlink{par:1.2.7}{1.2.7}), которые не~могут достичь гармонии, потому что каждая определяет размер, темп и~оркестровку этой гармонии.
	\end{itemize}

\paragraph{1.4.3}\hypertarget{par:1.4.3}{} Между одной сетью и~другой так же, как между одним актантом и~другим (\hyperlink{par:1.2.7}{1.2.7}), ничто само по себе не~является соизмеримым или несоизмеримым. Таким образом, мы никогда не~покидаем сеть независимо от того, как далеко она простирается.
	\begin{itemize}
	\item Именно по этой причине кто-то может быть Комендантом Аушвица, оливковым деревом на~Корфу, водопроводчиком в~Рочестере, чайкой на~островах Сцилли, физиком в~Стэнфорде, гейсом в~Минас Жераис, китом в~водах Земли Адели, одной из бацилл Коха в~Дамиетте и~так далее. Каждая сеть создает для себя целый мир, мир, внутренности которого являются не~чем иным как внутренними выделениями тех, кто его соорудил. Ничто не~может проникнуть в~галереи такой сети, не~будучи вывернутым внешней стороной внутрь. Если бы~мы подумали, что термиты были лучшими философами, чем Лейбниц, мы могли бы~сравнить сеть с термитником~--- до тех пор, пока мы понимаем, что снаружи нет никакого солнца, которое бы~напротив ослепило (затемнило) галереи сети. Никогда невозможно будет видеть более ясно, никогда невозможно будет забраться дальше <<наружу>>, чем термит, а~наиболее широко принимаемая эквивалентность может в~ходе испытания предстать не~более крепкой, чем глиняная стена.
	\end{itemize}

\paragraph{1.4.4}\hypertarget{par:1.4.4}{} Сила прокладывает путь, делая другие силы пассивными. Затем она может двигаться в~другие места, которые ей не~принадлежат и~относиться к~ним так, как если бы~они были ее собственные.
	\begin{itemize}
	\item Я готов разговаривать (talk) о~<<логике>> (\hyperlink{par:2.0.0}{2.0.0}), но только, если она рассматривается как отрасль общественных работ или гражданского строительства. Говорить (speak) таким образом более правильно, чем болтать (talk), как Ульрих, о~Генеральном Секретариате Точности и~Души\footnote{{\itshape Роберт Мусил}. Человек без свойств. Гл. 116}.
	\end{itemize}

\paragraph{1.4.5}\hypertarget{par:1.4.5}{} Энтелехия желающая быть сильнее можно сказать создает {\itshape линии сил}. Они держать других в~строю. Они делают их более предсказуемыми. 
	\begin{itemize}
	\item Термин <<линия силы>> еще более смутный чем <<сеть>>, <<путь>>, <<галерея>> или <<логика>>, но он превосходен. Читатель не~должен быть в~состоянии определить, говорю ли я социальных существах, напечатанных схемах, причинах, машинах, театрах или привычках. Эта неясность именно тот эффект, к~которому я стремлюсь, поскольку, возможно, мы никогда не~повстречаем объекты, классифицированные таким образом.
	\end{itemize}

\paragraph{1.4.6}\hypertarget{par:1.4.6}{} Как только какому-либо актанту удается убедить других встать в~строй (fall into line), он тем самым увеличивает свою силу и~становится сильнее тех, кого он выстроил в~ряд и~убедил (\hyperlink{par:1.5.1}{1.5.1}). Этот рост может быть измерен несколькими способами. Можно сказать, что А {\itshape соединено} с другими. Хотя в~принципе любое соединение равным образом возможно, теперь связать В с А стало легче, чем А с С. А, можно также сказать, {\itshape командует} другими. Хотя в~принципе эти другие отдали свою силу А (ссудили А своей силой), они позволяют себе быть управляемыми А. Также можно сказать, что А {\itshape переводит} желания других. Хотя другие, возможно, хотят сказать что-то еще, они согласны с тем, что то, что говорит А это то, что они хотели сказать, но не~смогли выразить это в~словах. Силу А можно измерить, сказав, что оно может покупать других. Хотя в~принципе другие не~стоят одинаково (\hyperlink{par:1.2.1}{1.2.1}), E или F согласны быть эквивалентными тому, что А готово заплатить. Наконец, можно сказать, что А {\itshape объясняет} других. Хотя другие не~редуцируют себя к~А, они согласны быть его следствиями, предикатами или приложениями (\hyperlink{par:2.0.0}{2.0.0}).

В конечном счете, работа по установлению цены и~установлению эквивалентности означает, что А сильнее чем другие, несмотря на~их несоизмеримость. Оно переводит, объясняет, понимает, управляет, покупает, решает, убеждает и~заставляет их работать. 
	\begin{itemize}
	\item Иногда это накопление эквивалентов или знаков (tokens) называется <<капиталом>>, но капитал был не~первым шагом. Сначала было необходимо создать эквивалентности (\hyperlink{par:1.3.7}{1.3.7}), покорить силы и~удержать их в~одном месте достаточно долго для того, чтобы определить их масштаб и~измерить. Только потом стало возможным подсчитать прибыль (\hyperlink{par:1.3.5}{1.3.5}). Рынок это только следствие установления сетей; он не~объясняет их формирование.
	\end{itemize}

\paragraph{1.4.6.1}\hypertarget{par:1.4.6.1}{} Абсолютная сила это сила, которая была бы~способна все объяснить, все перевести, все произвести, все купить и~возместить и~быть причиной действия всего. Как универсальный эквивалент, способный заменить собой все, и~универсальное провидение способное дать жизнь всему, она могла бы~быть перводвигателем и~первопричины, из которой может быть выведено все остальное. 
	\begin{itemize}
	\item Некоторые люди говорят о~Боге, когда думают о~силе способной спасти мир с помощью своего Сына, об объяснении происхождения и~сотворения, о~переводе на~Его язык (слово) того, что каждое создание, одушевленное и~неодушевленное, желает в~глубине~своего сердца, о~том, вести нас по окольным путям Провидения к~тому, чего мы все желает. Поскольку ничто само по себе не~является редуцируемым или не~редуцируемым (\hyperlink{par:1.1.1}{1.1.1}), эта абсолютная сила также является абсолютно чистым выражением ничтожества. Именно из-за своей чистоты она всегда очаровывала мистиков, военачальников, капитанов индустрии и~ученых ищущих первопричины. <<О!>>,~--- говорят все оно сами себе,~--- <<овладейте единственной силой (городом, потир (чашу), аксиомой, банком) и~все остальное будет отдано нам>>. Чтобы избежать паники редукции, мы всегда должны говорить: <<То, что оставлено, так это все (Интерлюдии I--II). Великий Пан мертв>>.
	\end{itemize}

\paragraph{1.4.6.2}\hypertarget{par:1.4.6.2}{} Актор расширяется, пока он может убедить других, что он их включает в~себя, защищает, спасает, понимает. Он расширяется быстрее и~дальше, если он может заполучить акторов, которые уже сделали себя эквивалентными многим другим.
	\begin{itemize}
	\item Часто говорилось, что <<капитализм>> был радикальным новшеством, неслыханным прорывом, <<детерриториализацией>> доведенной до предела. Как всегда, различие это мистификация. Как и~Бог, капитализм не~существует. Не существует эквивалентов (\hyperlink{par:1.2.1}{1.2.1}), они должны быть сделаны, и~они дорогостоящи, не~ведут далеко и~их не~хватает надолго. Мы, в~лучшем случае, может создавать протяженные сети (\hyperlink{par:1.4.2}{1.4.2}). Капитализм все еще остается маргинальным даже сегодня. Скоро люди осознают, что он универсален лишь в~воображении его врагом и~адвокатов (Интерлюдия IV). Также как римские католики верят в~универсальность своей религии, даже если она течет только по римским каналам, противники и~сторонники капитализма верят, возможно, в~одну из чистейших мистических грез: в~то, что абсолютная эквивалентность достигнута. Даже Соединенные Штаты, страна подлинного капитализма, не~могут жить согласно его идеалам. Несмотря на~усилия профсоюзов и~ассоциаций работодателей, роение сил не~может стать эквивалентным без работы (\hyperlink{par:3.0.0}{3.0.0}). Мое почтение Фердинанду Броделю\footnote{{\itshape Fernand Braudel}. The Perspective of the World, Fifteenth to Eighteenth Century (New York: Harper and Row, 1985).}, который не~скрывает этот факт, и~показывает, как можно достичь удаленного контроля посредством разреженных сетей.
	\end{itemize}

\paragraph{1.5.1}\hypertarget{par:1.5.1}{} Силе не~могут быть приписаны (cannot be given) те силы, которые она выстраивает в~боевой порядок и~убеждает. По определению она может только заручиться их поддержкой (\hyperlink{par:1.3.4}{1.3.4}). Несмотря на~это, она будет претендовать на~то, что ей не~принадлежит и~добавит их силы к~своим в~новой форме: так рождается потенция.
	\begin{itemize}
	\item Когда энтелехия содержит в~себе энтелехии, которые в~ней не~содержатся, мы говорим, что она содержит их в~себе <<потенциально>>. Исток потенции лежит в~следующей путанице: теперь уже не~возможно отличить актора от союзников, которые сделали его сильным. С этого момента мы начинаем говорить, что аксиома заключает в~себе ее доказательство потенциально (in potentia); мы начинаем говорить о~государе, что он могущественный, что бытие-в-себе включает бытие для себя, тем не~менее, только <<потенциально>>. Вместе с потенцией также начинается несправедливость, потому что, не~считая немногих счастливых~--- государей, принципов, истоков, банкиров и~директоров~--- другие энтелехии, то есть все оставшиеся, становятся деталями, последствиями, приложениями, последователями, слугами, агентами~--- короче говоря, рядовыми. Монады рождаются свободными (\hyperlink{par:1.2.8}{1.2.8}), но повсюду они остаются в~цепях.
	\end{itemize}

\paragraph{1.5.1.1}\hypertarget{par:1.5.1.1}{} Разговор о~возможностях (possibilities) это иллюзия акторов, которые путешествуют, позабыв о~стоимости транспорта.
	\begin{itemize}
	\item Производство возможностей так же~дорогостояще, локально, практично, как и~производство специальных сплавов или лазеров. Возможности покупаются и~продаются, как и~все остальное. Они не~отличны по своей природе. Они, например, не~являются <<нереальными>>. Нет такой вещи как бесплатная возможность. Картотеки консультантов дорогостоящи~--- спросите тех, кто стал банкротом, потому что они произвели слишком много возможностей, но не~продали их в~достаточном количестве.
	\end{itemize}

\paragraph{1.5.2}\hypertarget{par:1.5.2}{} Если актор содержит в~себе многих других потенциально (in potentia), это впечатляет, потому что, даже находясь в~одиночестве, он предстает толпой. Поэтому ему легче завербовать других акторов и~заручиться их поддержкой.
	\begin{itemize}
	\item Хотя это начинается как блеф с утверждения о~владении тем, что было лишь одолжено, это становится реальным. Так как реально то, что сопротивляется (\hyperlink{par:1.1.5}{1.1.5}), кто способен сопротивляться энтелехии превратившейся в~толпу? Власти, троны и~господства раздают чин за чином, хотя они ни~выросли, ни~переместились и~столь же~слабы, как и~те, кто позволяет им действовать.
	\end{itemize}

\paragraph{1.5.3}\hypertarget{par:1.5.3}{} Властью никогда не~владеют. Она либо есть у нас потенциально, но тогда мы не~обладаем ей; либо она у нас есть <<действительно>> (in actu), но тогда наши союзники единственные, кто переходят к~действиям.
	\begin{itemize}
	\item Философы и~социологи власти льстят господствующим, которых, как утверждается, они критикуют. Они объясняют действия господствующих в~терминах мощи власти, однако эта власть является эффективной только как результат соучастий, потворств, компромиссов и~смешиваний (\hyperlink{par:3.4.0}{3.4.0}), которые не~объясняются властью. Понятие <<власти>> это усыпляющее действие мака, которое вызывает в~критиках дремоту как раз в~тот момент, когда безвластные государи вступают в~союз с другие, которые равным образом слабы для того, чтобы стать сильными.
	\end{itemize}

\paragraph{1.5.4}\hypertarget{par:1.5.4}{} Хотя они не~могут ни~сосчитать, ни~суммировать других, все меньшее и~меньшее количество сил, не~имеющих ничего своего, приписывают потенцию всех других мощностей (powers) себе. Это {\itshape reductio ad absurdum} целого к~ничто. Государи, которые практически ничего собой не~представляют, действуют так, как если бы~остальные, которые являются всем, уже ничего собой не~представляли.