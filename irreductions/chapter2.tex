\chapter{Социологики}

\paragraph{2.1.1}\hypertarget{par:2.1.1}{} Всякая аргументация имеет одну и ту же форму: одно предложение следует за другим. Затем третье заявляет, что они идентичны, несмотря на то, что они не совпадают друг с другом. Впредь второе употребляется вместо первого, а пятое утверждает, что второе и четвертое идентичны, несмотря на то, что{\ldots} и так далее, до тех пор, пока одно предложение не замещают, делая вид, что оно осталось на месте и не переводят, делая вид, что осталось верным исходному значению (faithful).

\paragraph{2.1.2}\hypertarget{par:2.1.2}{} Никогда не существовало такой вещи как дедукция. Одно предложение {\itshape следует} за другим, а затем третье утверждает, что второе неявно или потенциально уже было внутри первого (\hyperlink{par:1.5.1}{1.5.1}).


\paragraph{2.1.3}\hypertarget{par:2.1.3}{} Когда устанавливается эквивалентность многих предложений, они все сворачивают в первое, о котором говориться, что оно <<заключает их всех в себе>>. Эта единственная фраза затем обсуждается, и утверждается, что все остальные могут быть выведены из нее <<посредством простой (pure) дедукции>>.

\paragraph{2.1.3.1}\hypertarget{par:2.1.3.1}{} Те, кто осуществляют доказательство в присутствии других и утверждают, что выводят (extract) одну фразу из {\itshape другой} в лучшем случае фокусники, а в худшем мошенники. Долгие годы они проделывали свои фокусы, используя кроликов и шляпы, взятые у зрителей.

\paragraph{2.1.3.2}\hypertarget{par:2.1.3.2}{} Только учителя могут заявлять, что они способны вывести одно предложение из в другого средствами <<простой, формальной дедукции>>. Они заранее знают вывод из аргументации, которую, как утверждается, они разворачивают. Систематичные доводы, выученные {\itshape медленно} и {\itshape беспорядочно} развертываются ими на высокой скорости, один за другим, скрывая то, что происходит за кулисами позади классной доски, беспорядочную историю, которая ведет к тому, что эта пропозиция должна быть соединена с той. Они
предлагают то, что потенциально (in potentia) содержит в себе все следствия ради поклонения своих учеников, которые горячо верят в то, что они дедуцировали одно из другого.
	\begin{itemize}
	\item Без школы никто бы не верил в эту религию дедукции. Мы также можем сказать, что пропозиции спинозовой <<Этики>> содержаться <<все в>> первой пропозиции, или что десерт содержится в антреме\footnote{Блюдо подаваемое между рыбой и жарким.~--- {\itshape Прим. перев.}}. Но школяры всегда были зачарованы абсолютной (совершенной) шпаргалкой предложенной принципом Лапласа: держать все знание в ладони нашей руки, достав его из каблука нашей обуви.
	\end{itemize}

\paragraph{2.1.4}\hypertarget{par:2.1.4}{} Аргументы образуют систему или структуру, только если мы забыли проверить их. Что? Если бы я атаковал один элемент, столпились бы все остальные вокруг меня, не мешкая ни секунды? Вряд ли! Всякое собрание актантов включает в себя ленивых, трусливых, двойных агентов, мечтателей, равнодушных и диссидентов. Да, я согласен с вами, что страх от того, что они увидели как А, В, или Е идут на помощь может настолько впечатлить людей, что они сдадутся. Но если они выстоят, вероятнее всего, что В будет
диссоциирован, потому что С шел слишком медленно, Е подавлен, F~--- предатель, а G не мог помочь потому, что пытался предотвратить предательство F. 
	\begin{itemize}
	\item 
	Как хорошо известно, альянс между логиками и армией привел генерала Штумма к тому, чтобы проверить надежность структур в библиотеке в Вене\footnote{{\itshape Роберт Мусил}. Человек без свойств. Гл.~85.}. Он был очень разочарован. В Париже все еще верим в структуры, потому что мы не заботимся о том, чтобы проверять их лояльность.
	\end{itemize}

\paragraph{2.1.5}\hypertarget{par:2.1.5}{} Комментарии никогда не бывают преданными. Либо это повторение, которое не является комментарием, либо это комментарий, который произносится {\itshape по-иному}. Другими словами это перевод и предательство. Несмотря на это, экзегеты не устают приписывать толкования текстам. Текст набухает толкованиями, которые он потенциально (in potentia) должен содержать в себя для того, чтобы оправдаться все эти прочтения. 
	\begin{itemize}
	\item 
	Тексты никогда не бывают преданными друг другу, но они всегда держат друг друга на некоторой дистанции.
	\end{itemize}

\paragraph{2.1.6}\hypertarget{par:2.1.6}{} Мы говорим <<кто управляет причиной, управляет следствием>>, как если бы следствие потенциально содержалось внутри причины. Однако, ни одно слово не может стать причиной другого. Слова {\itshape следуют} друг за другом в рассказе. И только дальше в рассказе один герой становится <<причиной>>, а другой <<последствием>>. Единственное следствие, которое следует принимать во внимание~--- это эффект производимый на публику тем или иным альянсом слов: <<Нет, он преувеличивает>>, или <<это написано хорошо>>, или еще <<очень понятно>>, <<очень убедительно>>, <<как самовлюбленно>>, <<какая скука>>.

\paragraph{2.1.7}\hypertarget{par:2.1.7}{} Теорий нет. Есть тексты, которым мы, как ленивые властелины, почтительно приписываем те вещи, которых они не совершали, подразумевали, предвидели или вызывали. Теории никогда не бывают одни, также как на открытой местности не бывает дорожных развязок без шоссе, которые соединяют и перенаправляют.

\paragraph{2.1.7.1}\hypertarget{par:2.1.7.1}{} В теории, теории существуют. На практике, нет. 
	\begin{itemize}
	\item 
	Никто никогда не выводил всю геометрию их аксиом и постулатов Евклида. Но <<в теории>>, говорят они, <<всякий и везде может>> вывести <<в любое время>> <<всю>> геометрию из аксиом <<одного>> Евклида. На практике, {\itshape такого не случалось ни с кем}. Но никому не требовалось приходить к такому заключению, поскольку <<в теории>> обратное возможно. А колдунов презирают, потому что, как говорят, они не способны принять факты, даже когда факты противоречат им каждый день на протяжении столетий!
	\end{itemize}

\paragraph{2.1.7.2}\hypertarget{par:2.1.7.2}{} Не существует метаязыков, только инфраязыки. Иными словами существуют только языки. Мы не можем более редуцировать один язык к другому, иначе как построив Вавилонскую башню. 
	\begin{itemize}
	\item 
	Те, кто говорит о метаязыке, я думаю, должно быть имеют в виду диалект (пиджин) специалистов, который слишком скуден даже для того, чтобы перевести то, что говорится на кухне.
	\end{itemize}

\paragraph{2.1.7.3}\hypertarget{par:2.1.7.3}{} Ежедневная практика не нуждается в теоретике, который бы обнаружил <<лежащую в ее основании структуру>>. <<Сознание>> не лежит в основании практики, но оно является чем-то еще, где-нибудь еще в другой сети. Практика ни в чем не испытывает недостатка. 
	\begin{itemize}
	\item 
	{\itshape Где} находятся бессознательные структуры примитивных мифов? В Африке? В Бразилии? Нет! Они среди архивных карточек кабинета Леви-Стросса. Если они выходят за пределы Коллеж де Франс на рю де Эколь, то это благодаря его книгам и ученикам. Если их находят в Баия\footnote{Штат на северо-востоке Бразилии.~--- {\itshape Прим. перев.}}, то это потому что они там преподаются.
	\end{itemize}

\paragraph{2.1.8}\hypertarget{par:2.1.8}{} До тех пор, пока форма имеет значение (\hyperlink{par:2.1.1}{2.1.1}), все аргументы одинаково хороши.
Все что нам нужно это серия предложений, а затем мы скажем, что одни из них одинаковы, а другие различны (\hyperlink{par:2.1.2}{2.1.2}). Предложения в таком случае сплетаются в косы, локоны, гирлянды, венки и паутины. Это можно сделать {\itshape всегда}, не так ли? В результате {\itshape некоторые} перемещения становятся легче, а некоторые труднее. 
	\begin{itemize}
	\item 
	Никто не может классифицировать аргументы с точки зрения их {\itshape формальных} качеств. Если вы настаиваете, мы можем выстроить их в ряд с точки зрения их {\itshape материальных} качеств.
	\end{itemize}

\paragraph{2.1.8.1}\hypertarget{par:2.1.8.1}{} Ничто само по себе не является логичным или нелогичным. Тропа всегда куда-либо ведет. Все что нам нужно знать это куда она ведет, и какой транспорт (груз) по ней должен проходить. Кто бы был столь глуп, что назвал бы шоссе <<логичными>>, дороги <<нелогичными>>, а козьи тропы <<абсурдными>>?

\paragraph{2.1.8.2}\hypertarget{par:2.1.8.2}{} Ни один набор предложений не является сам по себе последовательным или непоследовательным (\hyperlink{par:1.1.14}{1.1.14}); все, что нам необходимо знать~--- это кто проверяет его с какими союзниками и как долго. Последовательность чувствуется (\hyperlink{par:1.1.2}{1.1.2}); это не диплом, не медаль или торговая марка.

\paragraph{2.1.8.3}\hypertarget{par:2.1.8.3}{} Нить {\itshape доказательства} никогда не бывает прямой. Те, кто говорят о <<логике>>, никогда не видели как что-либо прядут, заплетают, выстраивают в линию, ткут или дедуцируют. Бабочка летит по более прямой линии, чем рассуждающий разум. (Иногда, конечно, вытканные орнаменты могут представлять собой, радующую глаз, прямую
линию.)

\paragraph{2.1.8.4}\hypertarget{par:2.1.8.4}{} <<Разум>> применяется для работы (\hyperlink{par:2.5.4}{2.5.4}) по распределению согласия и несогласия между словами. Это дело вкуса и чувства, ноу-хау и понимания, класса и статуса. Мы оскорбляем, хмуримся, дуемся, сжимаем кулаки, приходим в восторг, плюем, вздыхаем и мечтаем. Кто рассуждает?
	\begin{itemize}
	\item 
	Антрополог, изучающий язык тела может нарисовать эскиз мышления преподавателя из Кембриджа или банкира с Уолл Стрит.
	\end{itemize}

\paragraph{2.1.9}\hypertarget{par:2.1.9}{} Так как {\itshape количество} тождеств и различий, которое мы вынуждены {\itshape делить вместе} (share) остается постоянным (\hyperlink{par:2.1.8}{2.1.8}), это не в нашей власти быть нелогичными и иррациональными (\hyperlink{par:2.1.8.1}{2.1.8.1}). Тем не менее, существует множество способов расставить всевозможные <<вследствие>>, <<из-за>>, <<в противоречии с>>, <<несмотря на>>. Никто так невнимателен к <<non sequiturs>> как логики, волшебники и режиссеры. Когда придумываются эффекты, мы с особой тщательностью должны выбрать, что за чем последует. Мы должны определить, когда станет известно имя предателя или аксиомы и приготовиться к выходу актеров на сцену, который произведет наибольшее впечатление на аудиторию. Мы должны определит единицы измерения времени и пространства, причины и принципы. По мере того, как мы со вкусом отобрали теоремы и реплики в сторону, нам необходимо выбрать писать ли <<more geometrico>> или <<more populo>>. Короче, убеждение зависит от жанра, который мы избрали.
	\begin{itemize}
	\item 
	Мы забываем, что среди Азанде столько же скептиков, любителей силлогизмов, попперианцев и рационалистов, сколько их среди коперинков и сцилардов\footnote{Лео Сцилард (1898---1964)~--- американский физик. В 1934 обнаружил (совместно с Т.\,Чалмерсом) эффект разрушения химической связи под действием нейтронов. В 1939 наряду с другими показал возможность осуществления цепной ядерной реакции при делении ядер урана. Вместе с Э.\,Ферми определил критическую массу урана-235 и принял участие в создании первого ядерного реактора (1942). --- {\itshape Прим. перев.}}. Так как количество согласий и несогласий постоянно, мы не можем {\itshape чисто} отделить мифические фикции от научных исследований. Это может быть сделано только грязным способом и потому это настоящая бойня. Художник, который предпочитает лишь оттенки серого, является художником в не меньшей степени, чем тот, кто использует кричащие краски. Существуют доказательства такие же суровые как зима и эластичные (springlike) доказательства, но все они, тем не менее, доказательства.
	\end{itemize}

\paragraph{}\hypertarget{par:}{} Так как ничто ничему не присуще, то диалектика это небылица. Противоречия являются предметом переговоров как и все остальное. Они не даны, а выстраиваются.

\paragraph{}\hypertarget{par:}{} Если магия это форма практики, которая дает некоторым словам возможность воздействовать на <<вещи>>, тогда мир логики, дедукции и теории должен быть назван <<магическим>>: но это {\itshape наша} магия. 
	\begin{itemize}
	\item 
	Так же как греки называли высокие языки парфян, абиссинцев или сарматов <<варварскими>>, мы называем превосходные аргументы (\hyperlink{par:2.1.8}{2.1.8}) тех, кто верит в другие возможности дедукции, <<нелогичными>>.
	\end{itemize}

\paragraph{2.2.1}\hypertarget{par:2.2.1}{} Сказать что-либо, значит сказать это другими словами. Другими словами, это значит перевести.
	\begin{itemize}
	\item 
	Слово произносится вместо другого, с которым оно не совпадает. Третье слово говорит, что оно одинаковы (\hyperlink{par:2.1.1}{2.1.1}). А это не А, а В и С. Рим больше не в Риме, но на Крите и у Саксов. Это называется <<утверждением>> (предикацией). {\itshape То есть}, мы не можем говорить правильно, переходя от {\itshape одного} (same) {\itshape к тому же самому} (same), но только небрежно двигаясь от одного {\itshape к другому}.
	\end{itemize}

\paragraph{2.2.2}\hypertarget{par:2.2.2}{} Поскольку ничто не является редуцируемым или нередуцируемым к чему-либо еще (\hyperlink{par:1.1.1}{1.1.1}) и не существует эквивалентностей (\hyperlink{par:1.2.1}{1.2.1}), о каждой паре слов можно сказать, что они тождественны или не имеют ничего общего. Таким образом, нет очевидного способа отделить буквальный смысл от {\itshape переносного}\footnote{{\itshape Mary Hesse}. The Structure of Scientific Inference (London: Macmillan, 1974).}. Каждая группа слов может быть скабрезной, точной, метафоричной, аллегоричной, технической, правильной, или надуманной.

\paragraph{2.2.3}\hypertarget{par:2.2.3}{} Ничто само себе не является <<высказываемым>> или <<невысказываемым>>. Все переводится (\hyperlink{par:1.2.12}{1.2.12}). С того момента как одно слово всегда одалживает свой смысл другому, от которого оно, однако, отличается, не в нашей власти более говорить о том, что верно или ошибочно, кроме как уберечь маленькую мельницу сказки от вымучивания сути.

\paragraph{2.2.4}\hypertarget{par:2.2.4}{} Либо сказано то же самое и тогда ничего не сказано, либо нечто сказано, но это нечто другое. Выбор должен быть сделан. Все зависит от расстояния, которое мы готовы пройти и сил, которые мы готовы уговорить, в попытке сделать эквивалентными слова,
которые бесконечно далеки друг от друга. 

\paragraph{2.2.5}\hypertarget{par:2.2.5}{} Нас могут понимать, то есть окружать, отвлекать, предавать, замещать, передавать, но нас никогда не понимают {\itshape надлежащим} образом. Если сообщение транспортируется, то оно трансформируется. Нет сообщения, которое было бы просто прочитано.

\paragraph{2.3.1}\hypertarget{par:2.3.1}{} Мы никогда не заговариваем словами, которые свободно ассоциируются, но скорее на своем родном языке (\hyperlink{par:2.2.2}{2.2.2}). 
	\begin{itemize}
	\item 
	Другие уже поиграли словами, когда мы начали говорить (1.1.10). Год за годом, век за веком другие делали определенные ассоциации звуков, слогов, фраз и аргументов возможными или невозможными, правильными или варварскими, приличными или вульгарными, ложными или элегантными, точными или бессмысленными. Хотя ни одно из этих образований не является таким прочным, как утверждается (\hyperlink{par:2.1.4}{2.1.4}), если мы пожелаем их разрушить и переделать, мы станем объектом, который получает удары, плохие оценки, ласки, артиллерийский огонь или аплодисменты.
	\end{itemize}

\paragraph{2.3.2}\hypertarget{par:2.3.2}{} Хотя не существует точного или переносного смысла, возможно присвоить слово, редуцировать его значения и альянсы и жестко связать его службой другому [слову]. 
	\begin{itemize}
	\item 
	Даже все ароматы Аравии не сдобрят эту маленькую метафору, чтобы сделать ее фигуральной (\hyperlink{par:2.2.2}{2.2.2}). 
	\end{itemize}

\paragraph{2.3.3}\hypertarget{par:2.3.3}{} Все ассоциации звуков, слов, и предложений эквивалентны (\hyperlink{par:2.1.8}{2.1.8}), но так как они вступают в ассоциации в точности так, что они {\itshape более не} эквивалентны друг другу (\hyperlink{par:1.3.6}{1.3.6}), в конечном счете есть победители и побежденные, сильные и слабые, смысл и бессмыслица, и выражения, которые буквальны и метафоричны.

\paragraph{2.3.4}\hypertarget{par:2.3.4}{} Ничто само по себе не является логичным или нелогичным (\hyperlink{par:1.2.8}{1.2.8}), но не все равным образом является убедительным. Есть только одно правило: <<Дозволено все>>; говори все что угодно до тех пор, пока те, к кому ты обращался, не будут убеждены. Ты сказал, что для того, чтобы попасть из В в С необходимо пройти через D и E? Если другие не выразили несогласие и не предложили другие пути, то вы были убедительны. Они идут из В в С по предложенному пути, даже если никто не хочет покидать В ради и есть много других маршрутов, которые можно было бы избрать. Те, кого вы стремились убедить, уступили. Для них больше нет <<Дозволено все>>. Это придется сделать, поскольку {\itshape невозможно сделать что-либо лучше} (\hyperlink{par:1.2.1}{1.2.1}).

\paragraph{2.3.5}\hypertarget{par:2.3.5}{} Мы можем говорить все, что пожелаем, и все же не можем. Как только мы произнесли и вновь собрали (rallied) слова, образование других союзов стало более легким или более трудным. Асимметрия растет вместе с потоком слов; в потоке смысла пороги и плато скоро разрушаются. На поле брани между словами заключаются альянсы. Нам верят, нас ненавидят, нам помогают, нас предают. Мы больше не управляем игрой. Новые значения предлагаются, в то время как другие отбрасываются; нас комментируют, дедуцируют, понимают или игнорируют. Именно так: мы более не можем говорить то, что пожелаем.

\paragraph{2.4.1}\hypertarget{par:2.4.1}{} Каким образом одна серия предложений становится настолько <<сильнее>> чем другая, что последняя становится <<нелогичной>>, <<абсурдной>>, <<противоречивой>>, <<вымышленной>> или <<несерьезной>>? Как и сила (\hyperlink{par:1.3.2}{1.3.2}) аргумент становится сильным, лишь используя все, что оказывается под рукой. Таким способом мы можем заставить какой-либо актант признать, что это предложение является <<противоречивым>> или <<абсурдным>>, до тех пор, пока не найдется никого, кто бы и далее считал этот аргумент нелогичным.
	\begin{itemize}
	\item 
	Риторика не может рассматриваться как сила согласования предложений, потому что если нечто названо <<риторикой>>, тогда оно слабо и уже проиграло (\hyperlink{par:1.3.6}{1.3.6}). Логика не может рассматриваться как сила, так как она приписывает победу, которая является результатом нескольких предложений, <<формальным>> качествам общим для всех аргументов (\hyperlink{par:2.1.0}{2.1.0}). В таком случае семиотика вновь остается неадекватной, потому что упорно продолжает принимать во внимание только тексты или символы вместо того, чтобы также иметь дело с <<вещами в себе>>.
	\end{itemize}

\paragraph{2.4.2}\hypertarget{par:2.4.2}{} Слова никогда не бывают одни и они не окружены только словами; они были бы неслышимы.
	\begin{itemize}
	\item 
	Актант может сделать союзником все что угодно, так ничто само по себе не является редуцируемым ил нередуцируемым (\hyperlink{par:1.1.1}{1.1.1}) и поскольку нет эквивалентностей без работы по установлению эквивалентов (\hyperlink{par:1.4.0}{1.4.0}). Слово, таким образом, может в ступить в партнерство со смыслом, с последовательностью слов, утверждением, нейроном, жестом, стеной, машиной, лицом\ldots со всем, до тех пор, пока различия в сопротивлении позволяет одной силе быть более устойчивой, чем другая. Где написано, что слова могут вступать в ассоциации только с другими словами? Всякий раз, когда цепочка слов проверяется на прочность, мы измеряем {\itshape преданность} стен, нейронов, чувств, жестов, сердец, умов, бумажников~--- то есть гетерогенного множества союзников, наемников, друзей и куртизанок. Но мы не выносим эту нечистоту и беспорядочность.
	\end{itemize}

\paragraph{2.4.3}\hypertarget{par:2.4.3}{} Мы не можем различить те моменты, когда мы сильны и когда мы правы. 
	\begin{itemize}
	\item 
	Испытания сил лишь иногда принимают форму демонстрации насилия (\hyperlink{par:1.1.2}{1.1.2}); они также предстают во многих других обличиях. На одном полюсе актанты действуют столь мирно, что они смешиваются с фоном и становятся частью (flow) природы. Их деятельность настолько спокойна, что кажется, никакая сила вообще не используется (\hyperlink{par:1.1.6}{1.1.6}). На другом полюсе кровопролитие~--- тотальная война без ритуалов, цели или подготовки. Бывает ли так когда-либо? Где-то между, я полагаю, находится великая риторическая игра, где сила слова может ослабить альянсы и показать что-либо, где очень, очень редко, {\itshape при прочих равных условиях}, кто-либо говорит и убеждает. Мы всегда ограничиваем себя разговорами об этих трех случаях из учебников; я хочу поговорить обо всех других случаях также.
	\end{itemize}

\paragraph{2.4.4}\hypertarget{par:2.4.4}{} Языки ни господствуют, ни находятся в подчинении, ни существуют, ни не существуют. Они такие же энтелехии как все другие. Они ищут союзников в своих интересах и, как и другие актанты, выстраивают из них целый мир с такими же запретами и привилегиями.
	\begin{itemize}
	\item 
	Только лингвисты могут верить в то, что слова вступают в ассоциации только с другими словами с тем, чтобы образовать лингвистическую структуру. Они забывают о трудностях, возникших у них в попытке отвязать слова от их союзников, когда они изобретали свои структуры. То, что слова это силы подобные другим со своими временами и пространствами, своим <<габитусом>> и своими дружбами (дружескими отношениями), удивительно только для тех, кто верит, что <<люди>> существуют или владеют языком. Вы когда-либо боролись со словом? Разве ваш язык не уставал от разговоров? Все, что сопротивляется~--- реально (\hyperlink{par:1.1.5}{1.1.5}). Кто поверит, что существует чистая история одних лишь слов?
	\end{itemize}

\paragraph{2.4.5}\hypertarget{par:2.4.5}{} Невозможно надолго отделить тех актантов, которые собираются играть роль <<слов>> от тех, которые будут играть роль <<вещей>>. Если мы говорим только о языках и <<языковых играх>>, мы уже проиграли, поскольку мы пропустили момент перераспределения ролей и костюмов. 
	\begin{itemize}
	\item 
	Не так давно существовала тенденция наделять язык привилегией. Долгое время считалось, что он прозрачен и является единственным среди актантов, лишенным как плотности, так и насилия. Затем начали расти сомнения относительно его прозрачности. Выражалась надежда, что его прозрачность может быть восстановлена посредством очищения языка, так же как мы протираем окно. Язык был столь привилегированным, что его стала единственным достойным занятием для поколений кантов и витгенштейнов. Потом в пятидесятые пришло осознание, что язык был непрозрачным, плотным и тяжелым. Это открытие, однако, не означало, что он утратил свое привилегированное положение и был приравнен к другим силам, которые переводят и переводятся им. Напротив, была совершена попытка редуцировать все силы к означающему. Текст был превращен в <<объект>>. Это были <<поворотные шестидесятые>>, от Леви-Стросса к Лакану через Барта и Фуко. Что за суета! Все, что сказано об означающем верно, но это же должно быть сказано также о любой другой энтелехии (\hyperlink{par:1.2.9}{1.2.9}). В языке нет ничего особенного, что бы позволяло хоть сколько-нибудь еще отделять его от всех остальных. 
	\end{itemize}

\paragraph{2.4.6}\hypertarget{par:2.4.6}{} Устойчивость альянса определяется тем количеством акторов, которое необходимо собрать для того, чтобы разъединить его (\hyperlink{par:2.1.8.2}{2.1.8.2}). Следовательно, мы должны проверить его, если мы хотим знать, с чем мы имеем дело~--- если мы хотим знать {\itshape откуда} в действительности проистекает та эффективность, которую так часто приписывают отдельному слову, единичному тексту или знаку в небесах.
	\begin{itemize}
	\item 
	Они говорят, <<Вы не может перейти от В к D, не миновав С или Е>>. <<Если вы неуверенны в С, тогда вы также сомневаетесь в В и D>>. <<Если вы в В, то вы, следовательно, должны идти в D>>. Каждое из этих утверждений может с одинаковым успехом быть сделано относительно проблем геометрии, генеалогии, подземных работ, драки между мужем и женой или рисунке (varnish) на каноэ. Каждое из них может быть высказано относительно любой прочной (устойчивой) формы (\hyperlink{par:1.1.6}{1.1.6}).Вот поэтому <<логика>> является отраслью общественных работ (\hyperlink{par:1.4.4}{1.4.4}). Мы скорее не сможем больше вести машину по шоссе, чем сомневаться в законах Ньютона. {\itshape В каждом случае причины одни и те же}: удаленные точки были соединены путями, которые поначалу были узкими, а затем были расширены и надлежащим образом вымощены. К этому времени ничто кроме революции или природного катаклизма не заставит тех, кто использует эти пути, предложить путешественнику другой маршрут. Одна логика разрушается другой так же, как бульдозер сносит лачугу. В этой перестановке нет ничего сверхъестественного, хотя она может быть опасной, если экспроприированные отомстят за себя.
	\end{itemize}

\paragraph{2.4.7}\hypertarget{par:2.4.7}{} Гетерогенные альянсы, которые делают определенные цепочки слов согласованными (\hyperlink{par:2.1.8.0}{2.1.8.0}) образуют сети, которые могут быть длинными и несоизмеримыми~--- до тех пор, пока они не решать снять друг с друга мерку. <<Можешь ли ты сомневаться в связи, которая соединяет В с С?>>. Нет, не могу, до тех пор, пока я не буду готов потерять свое здоровье, репутацию или кошелек>>. <<Можешь ли ты ослабить узы, которые связывают D и Е?>> <<Да, но только с помощью золота, терпения и злобы>>. Необходимое и контингентное (\hyperlink{par:1.1.5}{1.1.5}), возможное и невозможное, твердое и мягкое (\hyperlink{par:1.1.6}{1.1.6}), реальное и нереальное (\hyperlink{par:1.1.5.2}{1.1.5.2})~--- все они увеличиваются таким образом. Для энтелехии существуют только {\itshape более сильные} или {\itshape более слабые} интеракции, с помощью которых можно построить мир.

\paragraph{2.4.8}\hypertarget{par:2.4.8}{} Предложение сохраняет свое единство (hold together) не потому, что оно истинно, но мы говорим, что оно <<истинно>>, поскольку оно сохраняет свое единство. За что оно держится? За множество вещей. Почему? Потому что оно связало свою судьбу с чем-то, что было под рукой и что было надежнее, чем она сама. В результате, никто не может поколебать его (shake itshape loose), не сотрясая все остального. 
	\begin{itemize}
	\item 
	Ничего больше, верующие; ничего меньше, релятивисты.
	\end{itemize}

\paragraph{2.5.1}\hypertarget{par:2.5.1}{} Недостаточно быть самым сильным; они еще хотят быть самыми лучшими. Недостаточно победить; они также хотят быть правыми. 
	\begin{itemize}
	\item 
	<<Самые сильные доводы всегда уступают доводам самых сильных>>. Вот это приложение в виде доброты это то, от чего я бы хотел избавиться. Аргументация самых сильных просто самая сильная. <<Этот подлунный мир>> сильно бы изменился, если бы мы избавились от этого не существующего приложения, если бы мы лишили победителей этого маленького дополнения. Для начала, он перестал бы быть подлым (base) миром.
	\end{itemize}

\paragraph{2.5.2}\hypertarget{par:2.5.2}{} Власть это пламя, которое заставляет нас путать силу с теми союзниками, которые сделали ее сильной (\hyperlink{par:1.5.1}{1.5.1}). Если бы мы надели маску сварщика, мы бы могли смотреть на место сварки, не будучи при этом ослепленными. Я не хочу больше принимать блеск щита (кадр на экране) за лицо сероглазой Афины, по крайне мере до тех пор, пока я этого не захочу.

\paragraph{2.5.3}\hypertarget{par:2.5.3}{} Мы можем избежать того, чтобы быть запуганными теми, кто присваивает слова и претендует на то, чтобы быть <<у власти>>. 
	\begin{itemize}
	\item 
	В ночь Шабаша ведьмы летают in potentia, в то время как их тела спят на соломе. Никто не верит этому сейчас, но магия продолжает существовать, магия тех, кто верит, что они могут путешествовать {\itshape дальше}, чем их тела и {\itshape за пределы} действия их силы. Шабаш магов разума проходит каждый день и эта магия не нашла своих скептиков (\hyperlink{par:4.0.0}{4.0.0}).
	\end{itemize}

\paragraph{2.5.4}\hypertarget{par:2.5.4}{} Мы не думаем и не рассуждаем. Мы скорее {\itshape работаем} с хрупкими материалами~--- текстами, записями, следами или рисунками~--- с другими людьми. Эти материалы ассоциируются и диссоциируются благодаря отваге и усилию; у них нет смысла, ценности или связности вне узкой сети, которая на какое-то время удерживает их вместе. Конечно, мы можем {\itshape расширить} эту сеть, рекрутируя других акторов, и мы можем сделать {\itshape усилить} ее, заручившись поддержкой более прочных материалов. Однако мы не можем покинуть ее даже во сне.
	\begin{itemize}
	\item 
	Торговля мясом простирается так далеко, как простирается искусство мясника, их прилавки, их холодильники, их пастбища и их бойни. Рядом с мясником~--- у бакалейщика, например~--- торговли мясом нет. Также обстоит дело с психоанализом, теоретической физикой, философией, бухгалтерским делом, социальной безопасностью, короче, со всеми ремеслами. Однако {\itshape некоторые ремесла }утверждают, что они могут потенциально или <<в теории>> распространятся за пределами сетей, в которых они практикуют. Мясник никогда бы не поддержал идею редукции теоретической физики к искусству мясника, но психоаналитик утверждает, что может редуцировать торговлю мясом к убийству отца, а эпистемолог с успехом говорит об <<основаниях физики>>. Хотя все сети одного размера, высокомерие распределено не равномерно.
	\end{itemize}

\paragraph{2.5.5}\hypertarget{par:2.5.5}{} Мы не можем освободиться от властвующих средствами <<мысли>>, но мы освободимся от власти, когда мы превратим <<мысль>> в работу. 
	\begin{itemize}
	\item 
	Разговорные выражения, которые мы используем для обозначения работы мысли (ломать голову, шевелить мозгами, переваривать идеи) являются не метафорами, но указывают на работу рук и тел общую для всех ремесел. Почему тогда это ремесло мышления в отличие от всех других не считается ручным. Потому что иначе оно должно будет лишить привилегии выходить за пределы его сетей. Стало бы более не возможно распространяться сверх простой практики ремесленников (\hyperlink{par:2.1.7.2}{2.1.7.2}). Все предпочитают выделять интеллектуалов (даже если только для того, чтобы высмеять их), чем признать что они работают. Даже если верующие не извлекают выгоды из этих бесплатных путешествий, они не хотят, чтобы другие лишались привилегии парить вне времени и пространства.
	\end{itemize}

\paragraph{2.5.6}\hypertarget{par:2.5.6}{} Нет разницы с одной стороны между теми, кто редуцирует и с другой~--- теми, кто желает дополнения в виде души. Две группы одинаковы. Когда они сводят все к ничему, они чувствуют, что все остальное ускользает от них. Поэтому они стремятся ухватиться за это с помощью символов. 
	\begin{itemize}
	\item 
	Символическое это магия тех, кто потерял мир. Это единственный способ, который они нашли, чтобы сохранить <<в дополнение>> к <<объективным вещам>> <<духовную атмосферу>>, без которой вещи были бы <<лишь>> <<естественными>>.
	\end{itemize}

\paragraph{2.5.6.1}\hypertarget{par:2.5.6.1}{} Мы можем быть уверены, что всякий раз, когда они заговаривают о символах, они пытаются путешествовать не заплатив. Они надеются двигаться, не покидая дома, надеются связать два актанта без грузовиков, газа и автострады. 
	\begin{itemize}
	\item 
	Тех, кто говорит о <<символическом>> поведении должно исследовать как волшебников. Они говорят, что волшебство постигается через слова, что не может быть достигнуто с помощью <<действенной практики>>. Но это определение должно быть приложено {\itshape к ним самим}. Не будучи способными овладеть силами посредством их испытания, они изобрели <<символы>>, которые, будучи <<добавленными к реальности>>, ничего не стоят и не потребляют.
	\end{itemize}

\paragraph{2.5.6.2}\hypertarget{par:2.5.6.2}{} Так как реально все, что сопротивляется, <<символическое>> нельзя добавить к <<реальному>>. До того, как к ним <<добавили>> символы, актанты не испытывали недостатка в чем-либо. Поэтому, если мы прекратим редуцировать их, это ненужное добавление, в свою очередь, превратиться в ничто. 
	\begin{itemize}
	\item 
	Если бы только мы освободились от символического, <<реальное>> было бы нам возвращено. Я готов согласиться с тем, что рыбы могут быть божествами, звездами или едой, что рыба может вызвать у меня болезнь и играть разные роли в мифах о происхождении. Они живут своей жизнью, а мы своей. В действительности же, наши жизни перекрывают и используют друг друга так долго, что в каждом ките по Ионе и по киту в каждом фолианте Мельвиля. Кто остановит переводы рыболовства, океанографии, дайвинга~--- всего того, что мы и рыбы используем для того, чтобы измерить друг друга? Этот человек еще не родился. (Интерлюдия IV). Те, кто хочет {\itshape отделить} <<символическую>> рыбу от ее <<реального>> двойника сами должны быть изолированы и заключены в  тюрьму (\hyperlink{par:3.0.0}{3.0.0}).
	\end{itemize}

\paragraph{2.5.6.3}\hypertarget{par:2.5.6.3}{} Мы не страдаем от нехватки духа. Мы напротив страдаем от {\itshape слишком большого количества} неприкаянных духов, которым никогда не предлагалось достойного погребения. Они бродят повсюду средь бела дня как несчастные призраки. Я хочу изгнать этих духов и убедить их оставить нас наедине с живыми.

\paragraph{2.6.1}\hypertarget{par:2.6.1}{}\hypertarget{par:2.6.1}{Всякое исследование оснований и  истоков поверхностно, так как оно надеется выявить энтелехии, которые потенциально содержат в себе другие [энтелехии].} Это невозможно. Если мы хотим проникнуть вглубь, мы должны последовать за силами в их сговорах и переводах. Мы должны следовать за ними, куда бы они ни вели, и составить список их союзников, какими бы многочисленными и широко распространенными они не
были. 
	\begin{itemize}
	\item 
	Те, кто ищут основания~--- редукционисты по определению и горды этим. Они всегда пытаются редуцировать некоторое количество сил к одной силе, от которой все остальные могут быть произведены. Чем больше их успех, тем более незначительной становится избранная [сила]. Наиболее глубокий является {\itshape также} наиболее поверхностным. Мы могли бы с таким же успехом относиться к Королеве Елизавете как к Соединенному Королевству или к первому предложению (\hyperlink{par:1.1.1}{1.1.1}) как к настоящему тексту.
	\end{itemize}

\paragraph{2.6.2}\hypertarget{par:2.6.2}{} Те, кто пытаются обладать тем, чего у них нет (\hyperlink{par:1.5.1}{1.5.1}), быть там, где их нет, и редуцировать то, что не редуцируется несчастливы, потому что они владеют потенцией только потенциально и имеют теорию только в теории.
	\begin{itemize}
	\item 
	Теперь мы можем перейти к морали более постоянного типа (\hyperlink{par:1.2.13}{1.2.13}). Мы не будет выискивать истоки, редуцировать практики к теориям, теории к языкам, языки к метаязыкам и так далее так как это было описано в Интерлюдии I. Мы будем работать внутри узких сетей, которые не могут быть редуцированы к другим, имея привилегии или ответственности не больше, чем кто-либо еще. Как и все остальные мы будет искать союзников и благоприятных возможностей, и когда-нибудь мы их найдем. <<Это не такая уж далеко идущая мораль, не так ли?>>. Именно так: она не заведет нас очень далеко. Она отказывается идти мысленно в места, где ее нет. Когда она движется, она платит по счетам. Мы больше не будем подражать Титану, и нести мир на наших плечах, подавленные бесконечной задачей понять, устроить, оправдать и объяснить все.
	\end{itemize}

\paragraph{2.6.3}\hypertarget{par:2.6.3}{} Поскольку не существует буквального или переносного смысла (\hyperlink{par:2.2.2}{2.2.2}), ни одно употребление метафоры не может доминировать над остальными употреблениями. Без правильности нет неправильности. Каждое слово точно и обозначает именно те сети, которые оно обнаруживает, выкапывает и по которым путешествует. Мы не должны бояться того, что один значение <<истинно>> а другое <<метафорично>>. Между словами тоже существует демократия. Нам нужна эта свобода, чтобы победить потенцию.

\paragraph{2.6.4}\hypertarget{par:2.6.4}{} Как мы назовем эту свободу переходить из одних владений в другие, это расширение сетей, это обследование? Имя этому ремеслу~--- философия и старейшие традиции определяли философов как тех, у кого нет особого поля, территории или области. 
Конечно, мы можем обойтись и без философии, и без философов, но тогда может и не быть пути перебраться из одной провинции в соседнюю, из одной сети в другую.

\paragraph{2.6.5}\hypertarget{par:2.6.5}{} Существует только два способа обнаружения сил. Первое, мы можем сказать, что с одной стороны силы, а с другой~--- {\itshape все прочее} ({\itshape other things}). Это равносильно отрицанию первого принципа (\hyperlink{par:1.1.1}{1.1.1}). Таким образом получаются <<реальные>> эквивалентности, <<реальные>> обмены, <<реальные>> сущности и мир упорядочен начиная с господ (государей/принцев, принципов, представителей, истоков, оснований, причин, капитала) нисходя к тем, кто находится в подчиненном положении (подразумеваемым, объясненным, дедуцированным, купленным, произведенным, оправданным, вызванным). Второе, мы можем придерживать первого принципа до конца. Если мы так поступим, то не будет больше никаких эквивалентностей, редукций, или властей (authorities) до тех пор, пока не будет выплачена надлежащая цена, а работа господства не станет публичной. 
	\begin{itemize}
	\item 
	Первый способ работы религиозный по сути, монотеистический по необходимости и гегельянский по методу. Он питает отвращение к магии, но, тем не менее, подражает ее методам. Второй способ работы оставляет локальным то, что локально и деконструирует {\itshape потенцию}. Он ведет к скептицизму относительно всех видов магии, включая нашу собственную.
	\end{itemize}


%\subparagraph{Интерлюдия III: Избавляющая от противоречия, которое по мнению автора, возможно поставило читателя в тупик.}