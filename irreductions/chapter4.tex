\chapter{Нередуцируемость <<Наук>>}


\paragraph{4.1.1}\hypertarget{par:4.1.1}{} Вы~можете стать сильным только посредством ассоциации. Но~так как это всегда достигается посредством перевода (\hyperlink{par:1.3.2}{1.3.2}), сила (\hyperlink{par:1.5.1}{1.5.1}, \hyperlink{par:2.5.2}{2.5.2}) приписывается могуществу, а~не~союзникам, благодаря которым вещи сохраняют устойчивость (holding things together) (\hyperlink{par:3.3.6}{3.3.6}). <<Магия>> --- это предложение могущества немощным (powerless). <<У них есть глаза, но~они не~видят, есть уши, но~они не~слышат{\ldots}>>
	\begin{itemize}
	\item 
 Я~уже говорил о~<<магии>>. Сначала я~использовал это слово для того, чтобы опровергнуть тех, кто убеждены, что они думают (\hyperlink{par:2.5.3}{2.5.3}), а~затем, чтобы одинаковым образом рассмотреть все виды логики (\hyperlink{par:2.1.11}{2.1.11}). Я~еще раз использовал его для того, чтобы создать эффект симметрии между <<примитивными культурами>> и~<<модерным миром>> (\hyperlink{par:3.5.4}{3.5.4}). Сейчас я~хочу использовать его для описания {\itshape всех} ошибок относительно истоков силы, {\itshape всякого} могущества.
	\end{itemize}	


\paragraph{4.1.2}\hypertarget{par:4.1.2}{} Не~верьте тем, кто занимается анализом магии. Они, как правило, маги, ищущие отмщения.
	\begin{itemize}
	\item 
 Я~имею ввиду Марка Оже, который всерьез <<en double>> принялся критиковать колдунов с~Берега Слоновой Кости\footnote{{\itshape Marc Aug\'{e}.} Th\'{e}orie des pouvoirs et~id\'{e}ologie (Paris: Hermann, 1975).}. Это очень помогло мне не~принимать всерьез критику <<en double>> со~стороны ученых. Когда все виды магии приводятся единому основанию, мы~получает новую форму скептицизма\footnote{{\itshape David Bloor}. Knowledge and Social Imagery (London: Routledge, 1976).}.
	\end{itemize}	

\paragraph{4.1.3}\hypertarget{par:4.1.3}{} Напротив, как только начинают понимать, что сила заключается в~альянсе слабостей, могущество исчезает. Конечно, силы все еще там, но~иллюзия могущества уничтожена. Нечто устраняет магическое воздействие могущества и~возвращает [силу] в~сеть, где она оформляется в~то, что я~называют <<ирредукция>>. 

Быть сильным, возможно, но~никогда могущественным. Убей меня, но~жди того, что я~возжелаю смерти и~поду на~колени перед властью. К~силе я~не~прибавлю {\itshape ничего}.

	\begin{itemize}
	\item 
 В~Интерлюдии III я~сказал, что мы~должны <<редуцировать тех, кто осуществляет редукцию>>. В~прежние времена борьба с~магией называлась <<Просвещением>>, но~это представление привело к~непредвиденным последствиям\footnote{В оригинале <<this image has backfired>>. --- {\itshape Прим. перев.}}. С~тех пор Просвещение стало эпохой (ир)радиации. Голова отважного исследователя, который пытался осветить тени обскурантизма, с~тех пор стала боеголовкой ракеты, которая ослепит нас светом. (Может быть, уже слишком поздно. Может быть, ракеты уже запущены. В~таком случае, давайте приготовимся к~последствиям очередной войны.)
	\end{itemize}	


\paragraph{4.1.4}\hypertarget{par:4.1.4}{} Когда сеть скрывает основание ее~ассоциации, я~говорю, что она проявляет <<могущество>>. Когда составляющий ее~массив слабостей видим, я~говорю, что она проявляет <<силу>>.


\paragraph{4.1.5}\hypertarget{par:4.1.5}{} Мы~страдаем не~от недостатка, но~от избытка духа. {\itshape Дух}, увы, никогда не~живет в~соответствии с~{\itshape буквой}. Дух~--- это всего лишь несколько слов среди многих других, которым несправедливо приписывается смысл всех остальных слов. Дух, таким образом, становится могущественной иллюзией. Истинно, говорю я~вам, дух немощен, но~буква сильна.

	\begin{itemize}
	\item 
 На~словах верующие ставят телегу впереди лошади. Однако на~практике они поступают совершенно иначе. Они утверждают, что фрески, витражи, молитвы и~коленопреклонение это просто способы приблизиться к~Богу, его удаленному отсвету. Тем не~менее они не~прекращают строить церкви и~выстраивать тела (arranging bodies) с~тем, чтобы создать средоточие могущества божественного. Мистикам хорошо известно, что если отбросить все элементы, которые, как говорят, указывают [на божественное], то~все что останется это ужасная ночь Нады (\hyperlink{par:1.4.6.1}{1.4.6.1}). Чисто духовная религия избавит нас от~религиозного. Убить букву~--- значит зарезать курицу несущую золотые яйца.
	\end{itemize}	


\paragraph{4.1.6}\hypertarget{par:4.1.6}{} То, что мы~называем <<наукой>> состоит из~обширного множества элементов, власть которого мы~приписываем немногим. 
	\begin{itemize}
	\item 
	<<Наука>> существует не~в большей степени, чем <<язык>> или <<современный мир>> (\hyperlink{par:3.5.2}{3.5.2})
	\end{itemize}	


\paragraph{4.1.7}\hypertarget{par:4.1.7}{} То, что мы~называем наукой довольно случайным образом выбрано из~пестрой толпы актантов. Несмотря на~то, что она представляет (репрезентирует) других, она отрицает этот факт (\hyperlink{par:3.4.6}{3.4.6}). 
	\begin{itemize}
	\item 
	Те, кто называет себя <<учеными>> на~словах всегда ставят телегу впереди лошади, несмотря на~то, что на~практике они делают все в~точности наоборот. Они утверждают что лаборатории, библиотеки, встречи, полевые заметки, приборы и~тексты есть всего лишь {\itshape способы и~средства} явить на~свет истину. Но~они не~прекращают строить лаборатории, библиотеки, приборы с~тем, чтобы создать средоточие могущества истины. Рационалисты прекрасно знают, что если задушить эту второстепенную материальную жизнь, то~они будут вынуждены замолчать. Чисто научная наука избавит нас ученых. По~этой причине они так пекутся о~том, чтобы не~зарезать курицу, несущую золотые яйца.
	\end{itemize}	


\paragraph{4.1.8}\hypertarget{par:4.1.8}{} Они являются скептиками и~неверующими в~отношении ведьм и~священников, но~когда дело доходит до~науки, они легковерны. Без тени сомнения они заявляют, что ее~действенность происходит от~ее <<метода>>, <<логики>>, <<строгости>> или <<объективности>> (\hyperlink{par:2.1.0}{2.1.0}). Однако, в~отношении <<науки>> они совершают ту~же ошибку, которую делает шаман, когда приписывает могущество своим заклинаниям. Вера в~существование <<науки>> имеет своих реформаторов, но~у нее нет скептиков, и~еще меньше агностиков.


\paragraph{4.1.9}\hypertarget{par:4.1.9}{} Так ничто само по~себе не~является редуцируемым или нередуцируемым к~чему-либо еще (\hyperlink{par:1.1.1}{1.1.1}), невозможно, чтобы с~одной стороны были испытания и~слабости, {\itshape а~с другой что-то еще} (\hyperlink{par:1.1.2}{1.1.2}, \hyperlink{par:1.1.5.2}{1.1.5.2}, \hyperlink{par:2.3.4}{2.3.4}, \hyperlink{par:2.4.3}{2.4.3}, \hyperlink{par:2.5.1}{2.5.1}). Однако, хитрость <<науки>> (\hyperlink{par:4.1.7}{4.1.7}) делит силы, которые делают так, чтобы некоторые выглядели сильными, в~то~время как другие <<истинными>> или <<разумными>>.

\paragraph{4.1.10}\hypertarget{par:4.1.10}{}Если бы~люди перестали верить в~<<науку>>, то~не было ничего кроме испытаний сил. Но~даже <<в науке>> существуют только испытания сил. Это означает, что ирредукция <<науки>> одновременно необходима и~трудна. Необходима потому, что он~стала {\itshape единственным препятствием} на~пути нашего избавления от~магии; трудна потому, что это наша последняя иллюзия и~когда мы~защищаем ее, мы~верим, что защищаем наше самое священное наследие. 
	\begin{itemize}
	\item 
	Если бы~это было не~так, я~бы~не посвятил целую главу критике <<науки>>, поскольку в~ней нет ничего особенного. 
	\end{itemize}	 


% \subparagraph{Интерлюдия VII: В~которой мы~узнаем, почему в~этом коротком фрагменте (конспекте, трактате) не~высказывается никакой симпатии в~адрес эпистемологии.}


\paragraph{4.2.1}\hypertarget{par:4.2.1}{} <<Наука>>~--- в~кавычках~--- не~существует. Это имя, которое приклеилось определенным сегментам определенных сетей, ассоциациям столь разреженным хрупким, которые бы~и вовсе остались без внимания, если бы~им не~приписывали все. 
	\begin{itemize}
	\item 
	Два или три процента ВНП нескольких промышленных государств (nations), две трети которого расходуется на~промышленность и~военные нужды~--- это не~так много. Малая доля того, что остается, имеет ценность лишь для нескольких тысяч людей, которые общаются еще с~несколькими тысячами и~популяризована на~благо миллионов прекрасных душ, которые с~трудом вообще это понимают. Для миллиардов других все эти сети {\itshape невидимы}.
	\end{itemize}	

\paragraph{4.2.2}\hypertarget{par:4.2.2}{} <<Наука>> не~самостоятельна. Она обретает форму, лишь отрицая то, что привело ее к~власти и, приписывая собственную солидность не~тому, что удерживает, а~тому, что удерживается вместе (\hyperlink{par:2.4.7}{2.4.7}). В~этом отрицании <<она>> игнорирует саму себя.
	\begin{itemize}
	\item 
	Если бы~<<физику>> лишили племен ублюдков, которые выполняют грязную работу, то~ее литературные произведения невозможно было бы~отличить от~литературных произведений алхимиков или психоаналитиков: было ли~это возможно в~прошлом, когда племен было не~так много?
	\end{itemize}	


\paragraph{4.2.3}\hypertarget{par:4.2.3}{} <<Наука>> это искусственная сущность {\itshape несправедливо} отделенная от~гетерогенных сетей. Есть две мерки: одна для ученых, другая для всех остальных. 
	\begin{itemize}
	\item 
	Если капиталист продает неприбыльную фабрику, его обвиняют в~жадности. Однако если известный ученый отказывается от~дискредитировавшей себя гипотезы, тогда напротив считают, что он~проявляет незаинтересованность. Если незадачливая ведьма приписывает успех в~битве магическому ритуалу, то~ее высмеивают за~ее легковерие. Но~если прославленный исследователь приписывает успех собственной лаборатории революционной идее, никто не~смеется, хотя каждый был бы~должен. Мысль о~том, чтобы сделать революцию с~помощью идей! Если потребители разрезают бифштексы на~маленькие кусочки, чтобы их~было легко прожевать, то~это никто не~комментирует. Но~если известный философ из~Амстердама утверждает, что мы~должны <<делить каждую трудность на~как можно большее число частей>>, то~невозможно выразить большего восхищения <<методом верно направляющим разум и~отыскивающим истину в~науках>>. Если самый безвестный фанатик-попперианец говорит о~<<фальсификации>>, люди готовы увидеть глубокую тайну. Но~если мойщик окон наклоняет голову, чтобы увидеть находится ли~пятно, которое он~хочет смыть внутри или снаружи, никто не~удивляется. Если молодая пара сменила часть мебели в~их~гостиной и, мало помалу, заключает, что обстановка не~выглядит должным образом, и~что необходимо поменять всю мебель, чтобы все опять подходило друг другу, кто находит это достойным упоминания? Но~когда вместе столов меняют <<теории>>, тогда люди говорят о~куновском <<парадигмальном сдвиге>>. Я~груб, но~это необходимо в~сфере, где несправедливость так глубоко [укоренена]. Они смеются над теми, кто верит в~левитацию, но~утверждают, не~встречая возражений, что теории могут создать мир.
	\end{itemize}	


\paragraph{4.2.4}\hypertarget{par:4.2.4}{} <<Наука>> производит впечатление существующей, только превращая свое существование в~{\itshape перманентное чудо}. Будучи неспособной принять своих настоящих союзников, она вынуждена объяснять чудо другим, а~другое третьим. Это продолжается до~тех пор, пока не~становится похожим на~сказку. 
	\begin{itemize}
	\item 
	Кто-то называет чудом то, что <<математика применима к~физической реальности>>. Другие говорят, что <<самая непостижимая вещь о~вселенной состоит в~том, что она вообще постижима>>. Остальные по-прежнему выражают изумление по~поводу того, что законы физики <<универсально применимы>>, что Ньютон открыл их, и~что Эйнштейн их~революционизировал. <<Наука>> становится по~истине цирковой интермедией с~гениями, революциями и~dei ex~machine. Но~никто не~говорит о~комнате страха позади всего этого. Когда становимся агностиками, мы~должны признать, что большая часть мест научного паломничества выглядят так же, как Лурд, но~все же~они более легковерны (наивны), поскольку высмеивают Лурд!
	\end{itemize}	

\paragraph{4.2.5}\hypertarget{par:4.2.5}{} <<Наука>> является санктуарием только до~тех пор, пока мы~будем рассматривать победителей и~побежденных ассиметрично. 
	\begin{itemize}
	\item 
	Никто не~может отделить <<внутреннюю>> историю науки, от~<<внешней>> истории ее~союзников. Первая вообще не~может считаться историей. В~лучшем случае это историография двора, в~худшем~--- Жития Святых. Последняя это не~история <<науки>>, это история.
	\end{itemize}


\paragraph{4.2.6}\hypertarget{par:4.2.6}{} Вера в~существование <<науки>> это следствие преувеличения, несправедливости, асимметрии, невежества, легковерия и~отрицания. Если <<наука>> отличается от~всего остального, то~это конечный результат длинной череды силовых переворотов (coups de~force).


\paragraph{4.3.1}\hypertarget{par:4.3.1}{} <<Наука>> чересчур ветха, чтобы говорить о~ней. Вместо этого мы~должны говорить {\itshape союзниках}, которых определенные сети используют для того, чтобы стать сильнее других (\hyperlink{par:1.3.1}{1.3.1}, \hyperlink{par:2.4.1}{2.4.1}, \hyperlink{par:3.3.1}{3.3.1}). В~таком случае вместо могущества мы~увидим силу (\hyperlink{par:4.1.5}{4.1.5}).


\paragraph{4.3.2}\hypertarget{par:4.3.2}{} Знание не~существует~--- как это (what would itshape be) (\hyperlink{par:1.4.3}{1.4.3})? Есть только ноу-хау. Другими словами, существуют ремесла и~профессии. Несмотря на~все утверждения об~обратном, ключ от~знаний в~руках у~ремесел. Они создают возможность возвращения <<науки>> в~сети, из~которых она произошла (Введение).


\paragraph{4.3.3}\hypertarget{par:4.3.3}{} Мы~не думаем. У~нас нет идей (\hyperlink{par:2.5.4}{2.5.4}). Скорее есть акт {\itshape письма}, акт который предполагает работу с~добытыми {\itshape записями} ({\itshape inscriptions}), акт который осуществляется посредством {\itshape разговоров} с~другими людьми, которые также пишут, записывают, говорят живут в~таких же~необычных местах; акт, который убеждает или не~способен {\itshape убедить} помощью записей (inscriptions), которых заставляют говорить, писать или быть прочитанными (\hyperlink{par:3.1.0}{3.1.0}, \hyperlink{par:3.1.9}{3.1.9}). 
	\begin{itemize}
	\item 
	Когда мы~говорит о~<<мысли>>, даже самые большие скептики теряют свои критические способности. Как вульгарные колдуны, они~позволяют <<мысли>> подобно магии путешествовать с~высокой скоростью на~большие расстояния. Я~не~знаю никого, кто не~был бы~легковерен, когда дело доходит до~идей. Тем не~менее <<мысль>> действительно совершенно проста, поскольку, когда мы~пишем о~других записях (inscriptions), мы~действительно покрываем большие расстояния за~несколько сантиметров. Карты, диаграммы, столбцы, фотографии, спектрографы~--- вот те~материалы, о~которых забывают, материалы, которые используются для того, чтобы сделать <<мысль>> неуловимой.
	\end{itemize}


\paragraph{4.3.4}\hypertarget{par:4.3.4}{} Несмотря на~все впечатления об~обратном, защищать то, что написано только лишь на~бумаге,~--- рискованное ремесло. Однако это ремесло не~является более чудотворным, чем ремесло художника, моряка, канатоходца или банкира.
	\begin{itemize}
	\item 
	Интересно видеть грека, который склонился над слепящей поверхностью пергамента и~страстно следует за~надрезами стилуса, даже тогда, когда они приводят к~софизмам. Увлекательно видеть Отцов Церкви, распространяющих разные версии одного и~того же~текста и~обучающихся ремеслу экзегетики, матери {\itshape всех} научных дисциплин. Волнительно следить за~итальянцем, переписывающим заново в~своих Диалогах книгу природы математической форме\footnote{{\itshape Elizabeth Eisenstein}. The Printing Press as~an Agent of~Change (Cam­bridge: Cambridge University Press, 1979).}. Увлекательно изучать, как это делал я~в течение двух лет, иголки, царапающие цилиндры физиографов; видеть, как расставляют ловушки, для того чтобы заставить то, о~чем говорят, писать (\hyperlink{par:3.1.5}{3.1.5}) и~говорить напрямую с~теми, кого хотят убедить. Эти странные тексты, которые являются не~священными писаниями, но~записями, произведенными внутренностями крысы или открытыми сердцами собак, странным образом очаровательны. Все они очень красивы, я~согласен. Они говорят об~огромной работе и~большой сноровке, но~они не~чудотворны. Нет ничего нематериального в~том, как бесконечно рассыпаются переплеты, щелкают ручки, стрекочут принтеры и~скрипят стилусы. Нет ничего нематериального в~этой одержимости письмом, записыванием, диаграммами и~спектрограммами.
	\end{itemize}	

\paragraph{4.3.5}\hypertarget{par:4.3.5}{} Обращены ли~они к~природе? Что бы~это могло значить? Посмотрите на~них! Они опираются на~то, что они написали и~на~разговоры друг с~другом внутри своих лабораторий. Посмотрите на~них! Их~единственным принципом реальности является тот, который они установили сами (\hyperlink{par:1.2.7}{1.2.7}). Посмотрите на~них! <<Внешние>> референты, которые они создали, существуют только внутри их~мира (\hyperlink{par:1.2.7.1}{1.2.7.1}).


\paragraph{4.4.1}\hypertarget{par:4.4.1}{} Все что локально всегда таковым и~остается. Ни~один тип работы не~является {\itshape более} локальным, чем какой-либо другой, пока его не~завоюют (\hyperlink{par:1.2.4}{1.2.4}) и~не~заставят оставить след. После над ним могут работать {\itshape в~его отсутствии}.
	\begin{itemize}
	\item 
	Африканского охотника, который покрывает десятки квадратных миль, и~который научился распознавать {\itshape сотни} {\itshape тысяч} знаков и~меток, называют <<локальным>>. Но~про картографа, который научился распознавать {\itshape несколько сотен} знаков и~индексов, склонившись над несколькими ярдами карт и~аэрофотоснимков, говорят, что он~более универсален, чем охотник, и~обладает глобальным видением. Кто из~них скорее потеряется на~территории другого? До~тех пор, пока мы~не проследим длинную историю превращения охотника в~раба, а~картографа в~господина, у~нас не~будет ответа на~этот вопрос. Между глобальным и~локальным нет тропинки, поскольку глобального не~{\itshape существует}. Вместо этого у~нас есть географы, самолеты, карты и~Ежегодные международные съезды геодезистов (International Geodesic Years).
	\end{itemize}	


\paragraph{4.4.2}\hypertarget{par:4.4.2}{} <<Общие идеи>> могут быть выстроены, но~сделать это не~более и~не~менее трудно, чем построить сеть железных дорог. Мы~должны заплатить за~<<общие идеи>>. Мы~не можем переехать с~одного стола на~другой с~помощью понятия <<стол>>. Для того чтобы переехать нам потребуется сеть, поддержание которой стоит так же~дорого, как и~поддержание железнодорожной сети с~ее~стрелочниками, замечательными железнодорожниками, бухгалтерами и~сигналами. 
	\begin{itemize}
	\item 
	Ученые очень хорошо понимают принцип <<приватизации прибыли и~национализации убытков>>. Они заставляют нас поверить, что они думают и~что идеи ничего не~стоят, но~они просят нас заплатить за~их лаборатории, лектории и~библиотеки (\hyperlink{par:4.1.9}{4.1.9}).
	\end{itemize}	


\paragraph{4.4.3}\hypertarget{par:4.4.3}{} Когда овладевают серией дислокаций и~соединяют их~в сеть, становится возможным перемещаться из~одного места в~другое, не~замечая той работы, которая связала их~вместе. {\itshape Одна} <<дислокация>> как кажется <<потенциально>> содержит в~себе все остальные. Я~рад назвать жаргон, который используется для того, чтобы проникнуть в~эти сети, <<теорией>> до~тех пор, пока подразумевается, что он~является подобием указателей и~меток, которые мы~используем для того, чтобы найти {\itshape обратный путь}.
	\begin{itemize}
	\item 
	Есть жаргон торговцев финикийского побережья, портовых грузчиков, финансистов, людей в~белых халатах, которые читают в~световых годах и~взвешивают в~пикограммах. Как все они могут понять друг друга? У~них нет общих целей. Они не~движутся вдоль одних и~тех же~силовых линий и~не~манипулируют одними и~теми же~следами. То, что мы~называем <<теорией>> реально не~более и~не~менее чем карта метро в~метро (\hyperlink{par:2.1.7}{2.1.7}).
	\end{itemize}	


\paragraph{4.4.4}\hypertarget{par:4.4.4}{} <<Универсальность>> так же~локальна, как и~все остальное. Универсальность существует только <<in potentia>>. Другими словами она не~существует до~тех пор, пока мы~не будем готовы заплатить высокую цену за~постройку и~поддержание дорогостоящих опасных связей. 
	\begin{itemize}
	\item 
	Если все случается локально и~только один раз (\hyperlink{par:1.2.1}{1.2.1}) и~если ни~одно место не~может быть редуцировано к~другому, тогда каким образом одно место может содержать в~себе другое? Не~обвиняйте меня в~номинализме. Все части армии {\itshape могут} быть соединены со~штабом. Офицеры ВВС {\itshape могут} работать с~картой мира размером три на~четыре метра. Все часы мире {\itshape могут} быть синхронизированы, если установлено универсальное время. Я~просто хочу, чтобы стоимость создания этих универсалий и~те~узкие каналы, по~которым они циркулируют, были включены в~счет.
	\end{itemize}	


\paragraph{4.4.5}\hypertarget{par:4.4.5}{} Так вы~верите, что применение математики к~физическому миру это чудо? Если так, тогда я~предлагаю вам полюбоваться другим чудом; я~могу путешествовать по~всему миру с~моей кредиткой <<American Express>>. Вы~скажите о~втором: <<Это всего лишь сеть. Если вы~выйдете за~ее пределы хотя бы~на дюйм, то~ваша кредитка не~будет иметь никакой ценности>>. Именно так. Это то, что я~говорю о~математике и~науке, {\itshape не~больше и~не~меньше}. 
	\begin{itemize}
	\item 
	Уравнение второй степени имеет область распространения, которую можно нанести на~карту, как и~все остальное. Его изобретение, перевод и~инкорпорацию в~другие практики можно проследить тем же~самым образом, каким мы~регистрируем распространение сбруи, кормового руля, галстука-бабочки, регулятора хода часов или IQ~тестов. Но~мы не~можем удержаться от~того, чтобы делить профессии на~две кучи. Одни жестко встроены свои контексты, в~то~время как другие летают как духи вне контекста. Я~хочу похоронить этих духов на~дне их~сетей, чтобы не~дать им~вновь вернуться после наступления темноты и~преследовать нас.
	\end{itemize}	

\paragraph{4.4.5.1}\hypertarget{par:4.4.5.1}{} <<Универсальное>> более не~может поглотить партикулярное так же, как исторические полотна могут заменить собой натюрморты. Теории не~могут быть абстрактными, если они таковы, то~это название указывает на~стиль, как абстрактная живопись. 
	\begin{itemize}
	\item 
	Когда кто-либо говорит мне об~универсальном, я~всегда спрашиваю какого оно размера, кто проецирует его и~на~какой экран. Я~также спрашиваю сколько людей обслуживают его и~сколько стоит оплата их~труда. Я~знаю, что это дурной тон, но~король голый и~выглядит одетым только благодаря тому, что мы~верим в~универсальное.
	\end{itemize}	

\paragraph{4.4.6}\hypertarget{par:4.4.6}{} Каким образом достигаются <<абстрактность>>, <<формализм>>, <<точность>> <<чистота>>? Как сыр~--- посредством фильтрования, сепарирования, формовки и~выдержки. Или как бензин~--- посредством очистки, крекинга, дистилляции. Нам нужны сыроварни и~очистительные заводы. Все это дорогие процессы и~грязные, вонючие профессии.

\paragraph{4.4.6.1}\hypertarget{par:4.4.6.1}{} {\itshape Работа} абстракции не~более абстрактна, чем работа могильщика; {\itshape ремесло} формализатора не~менее формально, чем ремесло мясника; работа по~очищению не~более чиста, чем работа санитарного инспектора. Сказать, что какие-либо процедуры являются чистыми, формальными или абстрактными, значит путать глагол с~прилагательным. Мы~можем также сказать, что дубление выдублено, фильтрование отфильтровано или логика логична.


\paragraph{4.4.7}\hypertarget{par:4.4.7}{} Быть абстрактным более не~в нашей власти, мы~можем лишь говорить надлежащим образом~(\hyperlink{par:2.2.1}{2.2.1}).


\paragraph{4.4.8}\hypertarget{par:4.4.8}{} Сети тонки, хрупки и~разрежены. Мы~читаем и~пишем внутри них. Мы~способны убедить, только расширяя сеть, иными словами редуцируя масштаб всего того, что было абсорбировано. В~результате несколько людей сидящих за~столом в~одной комнате, могут видеть (survey) все. {\itshape Что может быть проще? }Здесь не~из-за чего поднимать шум. 

%\subparagraph{Интерлюдия VIII: В~которой маленький пример повседневной социологии показывает, что такое меры.}

\paragraph{4.5.1}\hypertarget{par:4.5.1}{} В~научных профессиях, также как и~во~всех других, мы~учимся тому, как увеличить нашу силу локально (\hyperlink{chap1}{Глава 1}).


\paragraph{4.5.2}\hypertarget{par:4.5.2}{} Прибавление силы, достигнутое в~лаборатории, проистекает из~того факта, что большим количеством маленьких объектов манипулировали множество раз, что эти микрособытия могли быть записаны, что они при желании могли быть прочитаны вновь весь процесс в~целом может быть записан так, чтобы люди его прочитали. Для этого, требуются умения и~много денег, но~колдовство здесь не~при чем. 
	\begin{itemize}
	\item 
 Не~важно туманности ли~это, кораллы, лазеры, микробы, Валовые Национальные Продукты или результаты I.\,Q. теста. Не~важно являются ли~они <<бесконечно большими>> или <<бесконечно малыми>>. О~них {\itshape с~уверенностью} можно говорить только тогда, когда они будут перенесены в~маленькое место, где ими может управлять небольшое количество людей, и~их~заставят показать знаки~--- кривые, фигуры, точки, лучи или полосы~--- которые настолько просты, что согласие является возможным. Мы~можем только запинаться относительно всего остального.
	\end{itemize}	

\paragraph{4.5.2.1}\hypertarget{par:4.5.2.1}{} Правило достаточно просто: если мы~хотим увеличить нашу силу, пойдем тысячей на~одного в~тех вопросах, которые окупятся в~одном шансе из~ста. 
	\begin{itemize}
	\item 
	Представьте себе бациллу антракса, которая в~течение миллионов лет жила, будучи спрятанной в~толпе своих сородичей. В~один прекрасный день она обнаруживает себя одиночестве вместе со~своими детьми под слепящем светом микроскопа, над которым возвышается огромная борода Пастера. Кроме мочи ей~больше негде жить (\hyperlink{chap1}{Глава 1}). Это хороший пример переворота в~балансе сил. Разве точность всегда не~вырастает из~таких переворотов? В~действительности требуется слепота веры, чтобы игнорировать испытания сил, которые имеют место в~пыточных камерах науки~--- биотестерах, тензиметрах, линейных акселераторах, прессах, иглах, стилусах, вакуумных насосах, калориметрах. Оставаться слепым перед лицом этих испытаний равносильно <<отчаянному сопротивлению этому {\itshape вопросу}>>! Те, кто верит в~<<науку>> вместо этого являются настоящими мучениками.
	\end{itemize}	

\paragraph{4.5.3}\hypertarget{par:4.5.3}{} Итак, они более уверены в~себе чем другие? Конечно! Они множество раз испробовали свои аргументы на~маленьких моделях и~сделали все возможные ошибки. Очевидно, что они более уверены, чем те, у~кого есть только одна попытка.
	\begin{itemize}
	\item 
	Уважаемый эксперт ничем не~отличается от~презираемого всеми политика. Эксперт делает большое количество скрытых маленьких ошибок и~уверенно появляется из~укрытия {\itshape в~самый последний момент}. Политик совершает по-настоящему большие ошибки и вынужден действовать у всех на глазах. Здесь решения принимаются {\itshape до} ошибок (\hyperlink{par:3.6.3}{3.6.3}). Все люди одинаковы~--- в~равной степени правдивы, и~в равной степени лживы. Иначе как бы~они могли существовать?
	\end{itemize}	 

\paragraph{4.5.4}\hypertarget{par:4.5.4}{} Единственный способ быть сильным вновь, это воспроизвести отношения сил, которые некогда были выгодны. {\itshape Нет такой вещи как предсказание.} Предсказание это повторение чего-то, что уже имело место, в~увеличенном или уменьшенном масштабе. Только волшебники верят в~то, что они могут предсказывать будущее. 
	\begin{itemize}
	\item 
	Если мы~находим чудодейственным тот факт, что невакцинированная овца умирает Пуйи-ле-Фор или что Вояджер II~прошел между кольцами Сатурна в~установленный момент времени, тогда нам следовало бы~считать смерть Гамлета в~последнем акте столь же~удивительной. Ни~одно предсказание не~является чем-то большим, чем менеджмент сцены, который учит тому, как повторять генеральную репетицию~--- хотя он~и не~избавляет от~страха сцены и~тревожного ожидания. До~тех пор пока дело касается прогнозов Пастер, Шекспир и~НАСА оказываются неотличимыми. Если бы~им нужно было импровизировать или предсказывать, они бы~бессвязно бормотали как Пифия, так же~как это делаем мы, когда покидаем убежища наших профессий. И~Шекспир, вероятно, был бы~менее непоследовательным, чем любой другой. В~театре доказательства, или обычном театре все режиссеры одинаковы, в~равной степени лживы и~правдивы. Каким образом они могут различаться?
	\end{itemize}	

\paragraph{4.5.5}\hypertarget{par:4.5.5}{} Узнать это можно только посредством испытания сил. <<Знание>> это положение на~данном участке фронта. Оно не~простирается дальше. Как это возможно? (\hyperlink{par:1.1.0}{1.1.0}).
	\begin{itemize}
	\item 
	Ученые говорят, что приходят к~выводам в~лаборатории, <<при прочих равных условия>>, но~потом они забывают [об этом], предпочитая путешествовать магическим образом другие места, и~продолжая заниматься законотворчеством, как если бы~они все еще были дома.
	\end{itemize}	


\paragraph{4.5.6}\hypertarget{par:4.5.6}{} Ничего нельзя узнать за~пределами сетей, организованных и~управляемых ноу-хау (\hyperlink{par:1.3.7}{1.3.7}), но~эти сети могут быть расширены.


\paragraph{4.5.7}\hypertarget{par:4.5.7}{} Нет такой вещи как <<знание>> (\hyperlink{par:4.3.2}{4.3.2}), но~есть возможность осознавать (realize), то~есть делать реальным, понимать. 
	\begin{itemize}
	\item 
	Тайна {\itshape adequatio rei et~intellectus} есть просто расширение лаборатории. Если мы~не верим магию, это расширение становится видимым, но~если мы~обращаем массив слабостей чудодейственную власть, это расширение скрывается. У~<<науки>> нет внешнего (\hyperlink{par:4.3.5}{4.3.5}), но~только узкие галереи, которые позволяют лабораториям расширяться и~пробираться места, которые могут быть далеко. 
	\end{itemize}	

\paragraph{4.5.7.1}\hypertarget{par:4.5.7.1}{} Ничто не~покидает [пределы] сетей, и~в меньшей степени [это может сделать] ноу-хау, но~кто сомневается в~том, что сеть, которая оплачивает издержки, может расшириться? <<Докажи мне, что это вещество, которое так хорошо действует в~Париже, столь же~хорошо в~предместьях Тимбукту>>.

<<С какой стати? Есть универсальный закон>>.

<<Я не~хочу просто {\itshape поверить} в~это. Я~хочу {\itshape увидеть} это>>.

<<Просто подожди пока я~построю лабораторию и~я докажу это тебе\ldots>>

Спустя несколько лет и~миллионов долларов я~своими собственными глазами увидел доказательство того, о~чем я~просил в~новехонькой лаборатории. Я~выхожу, проезжаю несколько миль, и~ставлю вопрос снова:

<<Докажи мне, что\ldots>> 

	\begin{itemize}
	\item 
	Когда люди говорят о~том, что знание является <<универсально истинным>>, то~мы должны понимать, что это так же, как с~железными дорогами, которые существуют по~всему миру, но~только имеют ограниченную протяженность. Перейти к~утверждению того, что локомотивы могут двигаться за~пределами своих узких и~дорогостоящих путей это совсем другое дело. Тем не~менее, маги пытаются ослепить нас с~помощью <<универсальных законов>>, которые, как они утверждают, сохраняют свою силу даже в~разрывах между сетями!
	\end{itemize}

\paragraph{4.5.7.2}\hypertarget{par:4.5.7.2}{} Каким образом можно распространить ноу-хау? Так же~как радиоприемники, которые делаются в~Гонконге, или как таблицы умножения! Должны быть покупатели и~продавцы, учителя и~коммерческие каналы, представители и~книги, которые считаются заслуживающими доверия. 
	\begin{itemize}
	\item 
 Мы~говорим, что законы Ньютона могут быть обнаружены в~Габоне и~что это достаточно удивительно, так как это далеко от~Англии. Но~я видел камамбер в~супермаркетах Калифорнии. Это тоже достаточно удивительно, так как Лизьё далеко от~Лос-Анджелеса. Либо есть два чуда, каждым из~которых следует восхищаться одинаково, либо нет ни~одного.
	\end{itemize}	

\paragraph{4.5.7.3}\hypertarget{par:4.5.7.3}{} Люди обычно говорят о~<<научной истине>> шепотом. Однако всегда существовало лишь три способа почитать ее: согласованность~--- <<это логично>>; репрезентация~--- <<это соответствует>>; эффективность~--- <<это работает>>. Эти три выражения служат для того, чтобы просто указывать на~степень распространения сети. 
	\begin{itemize}
	\item 
 На~кухонной латыни мы~бы сказали {\itshape adequatio laboratorii et~laboratorii, adequatio laboratorii et~alius laboratoris, adequatio laboratorii et~vulgi percoris}.
	\end{itemize}


\paragraph{4.5.8}\hypertarget{par:4.5.8}{} Одна форма ноу-хау не~более <<истинна>> чем другая. Она не~более и~не~менее истинна, чем кофейник, дерево или лицо ребенка. Вот она, на~мгновение установившаяся линия сил (\hyperlink{par:1.1.6}{1.1.6}). Слово <<истинный>> это приложение, добавляемое к~конкретным испытаниям силы, чтобы ослепить тех, кто все еще может в~них сомневаться. 
	\begin{itemize}
	\item 
	Рационалисты смеются над ордалиями, которые делают победителя в~схватке правым. Однако, они ежедневно коронуют победителей научных споров, утверждая, что у~них более чистые сердца и~более рациональные умы! Есть две меры, два стандарта (\hyperlink{par:4.2.3}{4.2.3}).
	\end{itemize}	


\paragraph{4.5.9}\hypertarget{par:4.5.9}{} Мы~можем сказать, что все, что сопротивляется реально! (\hyperlink{par:1.1.5}{1.1.5}) Слово <<истина>> делает только лишь маленькое дополнение к~испытанию сил. Это не~много, но~это производит впечатление могущества (\hyperlink{par:2.5.2}{2.5.2}), которое спасает от~испытания то, что не~выдержит [проверки]. 
	\begin{itemize}
	\item 
	Релятивисты и~идеалисты никогда не~были в~состоянии удерживать свои позиции достаточно долго (\hyperlink{par:1.3.6}{1.3.6}), поскольку утверждения выходящие из~лабораторий противостоят, сопротивляются и~потому являются реальными (\hyperlink{par:2.4.7}{2.4.7}). Но~они правы: это не~основание для того, чтобы верить сказкам.
	\end{itemize}	

\paragraph{4.5.10}\hypertarget{par:4.5.10}{}Если нечто сопротивляется, то~оно создает у~тех, кто испытывает его оптическую иллюзию относительно того, что есть некий видимый и~описываемый объект, который вызывает это сопротивление. Но~объект это следствие, а~не~причина. Иллюзия исчезает, когда фронт борьбы смещается и~осторожно появляется вновь, когда фронт опять стабилизируется. 
	\begin{itemize}
	\item 
	<<Реальные миры там>>~--- это следствия стабильных силовых линий, а~не~причина их~стабилизации.
	\end{itemize}	

\paragraph{4.5.11}\hypertarget{par:4.5.11}{}Мы можем перформировать, трансформировать, деформировать и~таким образом формировать и~информировать самих себя, но~мы не~можем {\itshape описывать} что-либо. Другими словами не~существует репрезентации, кроме как в~театральном и~политическом смыслах этого термина. 
	\begin{itemize}
	\item 
	Трудность в~отношении <<наук>>, возможно, возникает из-за того факта, что работа, осуществляемая руками, приносит записи, которые читаются глазами. Возможно, эпистемология~--- это смешение чувств. Мы~следуем за~ослепленным взглядом, но~забываем о~руках, которые пишут, комбинируют, монтируют. Однако не~существует <<теории>>, <<созерцания>>, <<умозрения>>, <<предвидения>>, <<видения>> и~<<знания>>. Солнце Платона никогда не~припекало и~не~обращалось в~небе. Но~внутри сетей есть электроны, лампы накаливания, и~проекторы, которые потребляют электричество и~есть объекты так же, как и~кое-что еще. Такие лампы не~окружены ореолом тайны. Они включены в~свои розетки реальными руками.
	\end{itemize}	


\paragraph{4.6.1}\hypertarget{par:4.6.1}{} Почему нас должно удивлять, что те, кто собрали избыток сил и~присоединились со~своим влиянием к~конфликту, где никто не~имел преимущества, должны победить?


\paragraph{4.6.2}\hypertarget{par:4.6.2}{} Когда мы~не можем победить только благодаря своим собственным силам, мы~говорим о~тех, кем мы~повелеваем как <<власти>>, а~о соотношении сил как о~<<знании>>. Наши оппоненты могут быть в~состоянии сопротивляться сложению <<сил>>, но~не превосходству <<знания>> над <<властью>>.
	\begin{itemize}
	\item 
	Вот как можно объяснить разделение сил, с~которым мы~столкнулись с~самого начала (\hyperlink{par:1.1.5.2}{1.1.5.2}, \hyperlink{par:4.1.9}{4.1.9}). Это различение обозначает не~нечто явное, а~стратагему, которая умножает силы десятикратно, описывая некоторые из~них как <<науку>>. <<Наука>> сродни мечу Бренна, брошенному на~весы. Да, {\itshape vae victis}, поскольку их~объявят <<нелогичными>>, <<плохими>> и~<<неразумными>>. <<У неимеющего отнимется и~то, что имеет>>\footnote{<<Ибо всякому имеющему дастся и~приумножится, а~у неимеющего отнимется и~то, что имеет>> (Матфей 25:29).}. 
	\end{itemize}	

\paragraph{4.6.2.1}\hypertarget{par:4.6.2.1}{} Если бы~мы могли объяснить <<науку>> в~терминах <<политики>>, то~тогда бы~не было наук, так как их~развивают именно для того, что находить союзников, новые ресурсы и~пополнение войск. 
	\begin{itemize}
	\item 
	Вот почему социология науки так прирожденно слаба. Огюст Конт, отец сциентизма социологии, изобрел забавную систему двойной бухгалтерии. Наука~--- это не~политика. Это политика, осуществляемая {\itshape другими средствами}. Но~люди возражают <<наука не~редуцируется к~власти>>. Совершенно верно. Она не~редуцируется к~власти. Она предлагает другие средства. Но~на это опять возражают, что <<эти средства по~природе своей не~могут быть предвидены>>. Совершенно верно. Если бы~их можно было предвидеть, то~они бы~уже были использованы противостоящей силой (power). Что может быть лучше новой формы власти, которую никто не~знает, как использовать? Призывайте резервы! Мое почтение Шейпину и~Шафферу\footnote{{\itshape Steve Shapin, Simon Schaffer}. Leviathan and the Air-Pump (Prince­ton: Princeton University Press, 1985).}.
	\end{itemize}	

\paragraph{4.6.3}\hypertarget{par:4.6.3}{} Теперь, когда нас больше нельзя одурачить с~помощью этих маневров, мы~видим делегатов (\hyperlink{par:3.1.3}{3.1.3}), кем бы~они ни~были, говорящих от~имени других акторов, какими бы~они не~были. Мы~видим, как они бросают в~битву одно за~другим соединения своих союзников, отчасти колеблющихся, отчасти воинственных. 

	\begin{itemize}
	\item 
	Первый делает успехи, следуя за~своими микробами; второй~--- за~разгневанными рабочими; третий~--- за~своими китами, о~нуждах и~численности которых он~знает, которых он~хочет спасти; четвертый~--- за~своими батальонами; пятый~--- за~своим Кораном и~нефтедолларами; шестой~--- за~важными деловыми кругами, которые он~представляет; седьмой~--- за~своим бульдозером; еще один~--- за~своими овцами и~собакой. Все они выстроены в~боевые порядки и~проранжированы в~соответствии с~номерами, под которыми они были зачислены. Все они устанавливают, что является реальным на~линии фронта их~испытаний. Если мы~пытаемся разделить эту толпу на~людей и~не-человеков или на~<<политическое>> и~<<научное>>, то~тогда мы~совершаем ошибку~--- мы, я~настаиваю, мы~совершаем предательство (\hyperlink{par:4.7.0}{4.7.0}).
	\end{itemize}	

\paragraph{4.6.4}\hypertarget{par:4.6.4}{} Но~что бы~эти энтелехии, которые вступили в~наши конфликты, сказали, если бы~они могли говорить {сами за~себя}? <<То же~самое>>, поскольку их~заставляют говорить. Какие могут быть сомнения, когда блестящие демонстрации принуждают нас признавать это ежедневно? 
	\begin{itemize}
	\item 
	Иногда люди говорят о~<<природе>>, ссылаясь на~толпу рабов и~покоренных актантов, которых заставили замолчать, или когда говорят о~приказаниях, отдаваемых классом исследователей, которые в~свою очередь пляшут по~дудку горстки <<великих мыслителей>>. Однако крайне маловероятно, что силы ведут себя таким образом. В~конце концов, только два или три процента ВНП нескольких стран циркулирует внутри разреженных и~хрупких сетях <<науки>>. Мы~с таким же~успехом могли бы~редуцировать все путешествия к~сети авиалиний (\hyperlink{par:2.1.8.1}{2.1.8.1}). Актор должен достичь гегемонии, чтобы говорить в~единственном числе о~<<природе>> или <<реальном мире там>>. Гегемония является причиной, а~не~следствием <<мира>> в~единственном числе.
	\end{itemize}

\paragraph{4.6.5}\hypertarget{par:4.6.5}{} Но~что бы~могли сказать бесчисленные актанты зачисленные в~наши конфликты и~наши блестящие демонстрации, если бы~они могли говорить сами за~себя? Мы~не имеем понятия. Не~потому что они непознаваемы (\hyperlink{par:1.2.12}{1.2.12}), и~не~потому что они неописуемы (\hyperlink{par:2.2.3}{2.2.3}), но~потому что никто не~пытался, или скорее потому что те, кто пытался, стали слабее чем были до~этого.


\paragraph{4.6.6}\hypertarget{par:4.6.6}{} Мы~все еще мало знаем <<объективно>>. Мы~знает нечто только потому, что одни силы растут за~счет других. Мы~не имеет ни~малейшего представления о~том, что связывает силы вместе до~тех пор, пока они не~начнут действовать как пробы и~факты в~наших лабораторных конфликтах (\hyperlink{par:1.3.1}{1.3.1}).


\paragraph{4.6.7}\hypertarget{par:4.6.7}{} Как только мы~редуцируем редукцию <<наук>>, мы~вынуждены признать, что <<знание>> может существовать только на~уровне следов~--- во~всех смыслах этого термина. 
	\begin{itemize}
	\item 
 Мы~часто различаем между знанием прошлого и~современного мира (\hyperlink{par:3.2.5}{3.2.5}, \hyperlink{par:3.3.0}{3.3.0}). Это Великий Разлом, который не~позволяет нам видеть, что все эти знания обладают одним и~тем же~двигателем и~одной и~той же~общей формой: они не~заинтересованы в~вещах самих по~себе, в~том, чтобы проследить {\itshape их} пути; они касаются только человека и~видоизменений, которым человек может быть принудительно подвергнут. Как мы~обычно говорим, они <<социальны, слишком социальны>>. Образно говоря, мы~могли бы~сказать, что древние кривды и~современные правды относятся друг к~другу как две революции одной спирали. Несомненно, первая меньше, чем вторая, но~обе они обращаются за~помощью к~обществу. 

	Однако они различаются, явно различаются. Эти различия не~имеют никакого отношения ни~к критической строгости, с~которой получаются [эти знания], ни~к наличию данных. Различие состоит просто в~их~размере. В~прошлом поддерживались только малые коллективы. К~<<вещами>> проявляли интерес только с~той целью, чтобы усмирить их. Это знание теперь называют ложным, потому что оно слишком маленькое. Со~строительством больших Левиафанов стало необходимым интересоваться большим количеством вещей в~течение более длительного времени, чтобы быть более точными, более педантичными и~проникнуть в~гущу еще большего количества сил с~помощью еще большего количества лабораторий. Но~цель осталась той же: это все еще был человек, который должен был быть реформирован, деформирован, трансформирован и~информирован. Да, то~знание, которое мы~считаем новым, является столь же~антропо{\itshape{морфным}}, как и~его предшественники. Нет, оно является таковым даже в~большей степени! Поскольку стало необходимым завоевывать большие количества людей, стало важным наносить удары еще сильнее. Итак, мы~восхищаемся объективностью доводов, которые мы~создали? Но~в чем мы~хотим быть правыми для того, чтобы наносить удары столь сильно и~жестоко? Нужны ли~тому, кто не~желает никого убивать факты столь же~твердые как биты? 

	Как насчет того, чтобы совершить такую странность и~последовать за~вещами туда, куда они нас приведут? Кто может сказать по~совести, что сейчас больше людей, которые бы~были заинтересованы в~том, чтобы фланировать вдоль {\itshape их} пути, чем это было в~прошлом? Сделать такое означает, что мы~слабы, а~не~сильны. Это означает отъезд без мысли возвращении. Или, если мы~все-таки вернемся, это значит, что мы~придем с~пустыми руками; без добычи, трофеев, коллекций, статей или диссертаций. Можем ли~мы по~совести сказать, что мы~видели больше людей, которые ведут себя таким образом? 

	Идеалисты были правы: мы~можем знать, только в~той мере, в~которой мы~приближаем вещи к~нам самим. Но~они забыли добавить, что вещи необходимо собрать вместе, чтобы свергнуть нас. Крылатые ракеты вращаются по~орбите Левиафанов и~рано или поздно падают, чтобы произвести впечатляющий побочный эффект. Коперниковская революция была совершена посредством игнорирования всего остального, и~потеряно было практически все. Нам осталась лишь магия~--- наука и~колдовство, будущие войны определенное количество восхитительного знания добытого, вопреки нам всем, на~пересечении антропоморфизма и~объективности.

 Я~не~говорю этого, потому что я~хочу потопить нашу единственную спасательную шлюпку. Я~говорю это, потому что я~хочу предотвратить кораблекрушение, или, если уже поздно, сделать возможным выжить после кораблекрушения.
	\end{itemize}	



\paragraph{4.7.1}\hypertarget{par:4.7.1}{} Поскольку существуют только узы слабости, нет двух способов обучения~--- один их~которых академический, человеческий, рациональный или нововременной, а~другой народный, естественный, дезорганизованный или древний. Есть только один способ. Мы~всегда учимся одним и~тем же~способом, без того чтобы идти напрямик, предвидеть или когда-либо покидать сети, которые мы~выстроили. Мы~совершаем каждую ошибку столько раз, сколько необходимо, чтобы продвинуться от~одного пункта к~другому. Мы~никогда не~сможем делать что-то лучше (\hyperlink{par:1.2.1}{1.2.1}). 
 Мы~никогда не~сможем идти быстрее. Мы~никогда не~будет видеть яснее. 

	\begin{itemize}
	\item 
	Науки всегда критиковали от~имени более высоких форм знания, которые являются более интуитивными, непосредственными, человеческими, глобальными, теплыми, развитыми, духовными или искусными. Мы~всегда хотели критиковать науку, утверждая, что некая альтернатива лучше, добавляя апелляционный суд к~суду первой инстанции, обращаясь к Богу богов за~тем, чтобы он~уязвил гордость ученых и~приберег тайну вещей для скромных и~смиренных. Но~нет знания более высокого, чем знание наук, потому что нет шкалы знания и, в~конечном счете, вообще знания. Нам следует растворить все споры об~<<уровнях знания>> в~низшей форме знания, единственной форме имеющейся у~нас. Не~метафизика, а~инфрафизика. Как мы~уже сказали, нам никогда не~удастся возвыситься над необузданным политиканством (\hyperlink{par:3.6.0}{3.6.0}).
	\end{itemize}	


\paragraph{4.7.2}\hypertarget{par:4.7.2}{} Нет такой вещи как высшее и~низшее знание. Если мы~вообще хотим сохранить эти термины, мы~должны сказать, что некоторые формы знания <<выше>> чем другие, потому что обладатели высшего знания вознесли себя при молчаливом попустительстве обладателей низшего знания (\hyperlink{par:4.4.0}{4.4.0}).


\paragraph{4.7.3}\hypertarget{par:4.7.3}{} Холодны ли~<<науки>>? Строги? Бесчеловечны? Объективны? Скучны? Аполитичны? Нововременны? Эти недостижимые качества просто приписываются им~их врагами, которые таким образом надеются их~заклеймить (Интерлюдия IV). Горячие? Беспорядочные? Неистовые? Антропоморфные? Антропоцентричные? Пристрастные? Дикие? Мифичные? Нет, эти термины не~описывают их~также. Разреженные и~хрупкие и~в первую очередь разреженные. Каков их~специфический знак? Отсутствие особых примет.
	\begin{itemize}
	\item 
 Я~упрекаю эти плохо изученные агрегаты, которые мы~называем науками, не~за то, что они слишком рациональны, но~скорее за~не понимание природы своих природ. Давайте редуцируем их~к тем сферам, которые они занимают, и~наконец-то избавимся от~магии. С~самого начала эпистемология следовала за~науками по~пятам, пытаясь быть: ПЕРИ-, МЕТА-, ПАРА-, ИНФРА-, СУПРА-научной. Но~это не~понимание сути предмета. Политика, несомненно, все еще лучшая модель испытаний слабостей, и~никогда еще не~была столь подходящей как, когда мы~обнаружили, что исследователь ведет себя как делегат молчаливых толп атомов, микробов или звезд. В~этом случае мы~видим исполнителя, законодателя и~судью, который слишком долго уклонялся даже от~самых элементарных форм демократии.
	\end{itemize}	

\paragraph{4.7.3.1}\hypertarget{par:4.7.3.1}{} Те~из нас, кто хочет конвоировать <<науки>> обратно к~их~должному месту обитания, больше рационалисты, чем большинство ученых людей, которые хотят распространить их~<<вдвойне>>. Мы, по~крайней мере, знаем о~стоимости работы, включенной в~умножение тех мест обитания.
	\begin{itemize}
	\item 
	Гностики должны понять правильно: я~не~пытаюсь сделать их~жизнь легче.
	\end{itemize}	

\paragraph{4.7.4}\hypertarget{par:4.7.4}{} Коль скоро не~существует другого мира, совершенство обитает в~этом. Абсолютное знание обнаруживается в~этом мире, коль скоро не~существует более уровней знания. Те~же самые люди, которые устанавливают уровни знания, оказываются потом теми, кто впадает в~отчаяние от~невозможности достичь вершины: те~же самые редукционисты, которые попеременно то~опьянены властью, то~ослаблены немощью, оказываются то~высокомерными, то~скромными. Все испытания силы являются цельными, законченными и~точными именно в~той степени, в~которой это возможно. {\itshape Они не~приблизительны}. Не~являются они и~смутными, конвенциональными или субъективными. До~тех пор пока не~будут установлены новые отношения силы, они не~хороши и~не~плохи. Совсем не~утратив определенности, мы, в~конце концов, обнаружили, что привело к~иллюзии знания без неопределенности.


\paragraph{4.7.5}\hypertarget{par:4.7.5}{} Поскольку существует не~два способа познания, а~только один, существуют с~одной стороны те, кто склоняется перед силой аргумента, а~с другой~--- те, кто понимает только насилие. Демонстрации~--- это всегда демонстрации силы (\hyperlink{par:3.1.8}{3.1.8}), а~линии силы это всегда мера реальности, ее~единственная мера (\hyperlink{par:1.1.4}{1.1.4}). Мы~никогда не~преклоняемся перед разумом, но~скорее перед силой.


\paragraph{4.7.6}\hypertarget{par:4.7.6}{} Веря в~обратное, мы~позволяем определенным линиям силы и~определенным аргументам господствовать в~сетях, к~которым они собственно принадлежат. Мы~создаем могущество (\hyperlink{par:1.5.1}{1.5.1}), и~таким образом ослабляем других. 
	\begin{itemize}
	\item 
	Существуют, говорят они, благоразумные люди, которые уступают только силе аргумента, и~остальные, которые неблагоразумны, и~которые слепо подчиняются силе без понимания. Мне никогда не~приходилось встречать кого-либо, кто не~презирал бы~неблагоразумных людей, и~кто не~верил бы~в то, что это презрение воплощает благодетель.
	\end{itemize}

\paragraph{4.7.7}\hypertarget{par:4.7.7}{} Коль скоро <<правота>> отделяется от~<<могущества>>, или <<разум>> от~<<силы>>, правота и~разум оказываются ослабленными, потому что мы~больше не~понимаем их~слабости, мы~постепенно овладеваем единственным способом стать справедливым благоразумным, который доступен для презираемых. Эти две утраты освобождают поле для нечестивцев. Я~называю это преступлением, единственным преступлением, которое нам необходимо в~этом эссе.
	\begin{itemize}
	\item 
	Человек, который уступает твердости маленького аргумента только после сотен испытаний и~тестов, ошибок и~починок в~его лаборатории, тем не~менее, утверждает, что те, кого тестировали и~испытывали, ничего не~понимают и~размышляют как болваны. Несмотря на~то, что он~не умеет говорить ясно, в~тот момент, когда он~выходит из~двери своей лаборатории, он~в возмущении обнаруживает, что <<не каждый понимает такой простой аргумент>>. Его возмущение подпитывает презрение. Поскольку он~презирает дураков вокруг него, он~забывает об~одной вещи, которая заставляет его уступать силе аргумента: его лаборатории, месте, где он~сам подвергается испытаниям. Это порочный круг. Чем более глупы другие, то~больше он~верит в~то, что он~может <<думать>> и~тем меньше он~состоянии увидеть то, как он~научился. Чем больше он~расширяет могущество разума за~пределы силы, тем более ослабленным оказывается разум.
	\end{itemize}


\paragraph{4.7.8}\hypertarget{par:4.7.8}{} Противопоставление правоты и~могущества преступно, потому что оно расчищает поле для нечестивцев, делая вид в~это время, что защищает его при помощи силы справедливого. Но~то, что является правым, лишено силы кроме как <<в принципе>>. Не~будучи в~состоянии гарантировать, что то, что является правым, является сильным, люди вели себя так, будто бы~то, что было сильным, было порочным. Сильные просто заняли место, которое освободили те, кто без задней мысли презирали их.
	\begin{itemize}
	\item 
 В~результате понятной перестановки Макиавелли и~Спинозу стали считать аморальными, даже несмотря на~то, что они были правы в~том, что отказывались разделять могущество и правоту. Однако настоящее изложение отличается от~<<Богословско-политического трактата>> Спинозы. Времена изменились. Экзегетика религиозных текстов теперь заменена экзегетикой <<научных>> записей. По~этой причине я~рассматриваю это эссе как <<Научно-политический трактат>>. И~все же, объект один и~тот же. Мы~все еще в~самом начале экзегезиса, и~связь между наукой и~демократией истончилась в~ходе <<научных войн>>. Подобно Спинозе мы~выглядим жестокими с~тем, чтобы быть честными.
	\end{itemize}	

\paragraph{4.7.9}\hypertarget{par:4.7.9}{} Мы~страдаем не~от нехватки души, разума, науки или справедливости, а~от~излишка всех этих дополнений, которые присоединяются к~отношениям силы, чтобы замедлить рост могущества и~сделать слабого немощным. Если бы~у слабого против них был бы~только набор слабостей, который я~описал, то~они бы~замарали руки и~трансформировали бы~его по~своему нраву. 
	\begin{itemize}
	\item 
	<<Noli me~tangere>>\footnote{Не прикасайся ко~мне {\itshape(лат.).}}~--- это слова магов, которые хотят в~одно и~тоже время быть мертвыми, и~живыми; и~там и~здесь; и~сильными и~рациональными; и~сильными хорошими.
	\end{itemize}	

\paragraph{4.7.10}\hypertarget{par:4.7.10}{}Поскольку есть только один способ познания, а~не~два~--- испытание отношений между силами~--- мы~никак не~можем избежать одной ошибки, нелепости или преступления. Мы~не можем избежать одного эксперимента или одной попытки сократить путь. Даже {\itshape думать} об~обратном~--- значит сбить себя с~толку преступными иллюзиями. 
	\begin{itemize}
	\item 
 Во~скольких ядерных войнах мы~должны повоевать прежде, чем мы~уступим силе аргумента, что это никоим образом не~может быть способом ведения наших дел? Послушайте, это очень просто. Мы~никогда не~станем лучше, чем те, кто просто убедили себя в~пустяках, имея под рукой все, что им~нужно, имея надлежащее питание, будучи хорошо одетыми и~соответствующим образом обученными. Как много ошибок они совершат прежде, чем они начнут расставаться с~мельчайшими предрассудками? Десятки, сотни, тысячи? Как много войн потребуется для того, чтобы убедить пять миллиардов мужчин и~женщин? Десять? Сто? Иначе, до~тех пор массы не~смогут думать быстрее яснее, чем те, кто находится в~лаборатории.
	\end{itemize}	

\paragraph{4.7.11}\hypertarget{par:4.7.11}{}Те, кто думают, что они могут делать что-либо лучше и~работать быстрее всегда будут делать хуже, потому что они забудут поделиться своими выдающимися средствами познания и~испытания. Они будут верить в~то, что они достаточно сделали, когда они <<распространили>> причины, коды, результаты. В~действительности, все это погибнет, как только будет извлечено из~презираемых сетей, которые позволяют им~быть сильными.
	\begin{itemize}
	\item 
	Когда Вольтер хотел высмеять религию, он~обычно подписывал свои письма <<ecrelinf>> <<истребить бесчестных>>. Религия пережила свои худшие времена, более чем худшие. Сегодня мы~обнаруживаем себя в~такой же~позиции. Мы~никогда не~смогли бы~вообразить такой источник чудес, энтузиазма, теплоты и~откровения, который бы~сравнился с~тем, что мы~вульгарно называем <<науками>>. И~тем не~менее, пока тысячелетие не~подошло концу, мы~должны подписывать наши письма словом <<ecrelinf>>. Для того, чтобы обладать знанием в~следующем тысячелетии, чтобы мочь говорить о~точности, без того, чтобы на~не обозвали облученными, мы~должны спасти знание от~<<наук>>, точно так же, как божественное было спасено от~пустой оболочки религии. При помощи божественной любви мы~должны были истребить все, что было религиозного в~нас. При помощи любви знания мы~должны выпутаться из~<<наук>>. Мы~не можем сопоставить Galileo и~крылатые ракеты, таким же~образом, каким Нагорная Проповедь столь долго противопоставлялась Инквизиции. Меня не~интересует апологетика. В~<<науке>> так же, как и~в <<религии>> более чем достаточно протестантов, мистиков, интегристов, анабаптистов, фундаменталистов светских Иезуитов. Никто из~них меня не~интересует, потому что все они хотят реформировать или обновить те~плохо сконструированные целостности, <<науки>>. Все они ищут пути примирить непримиримое, и, занимаясь этим, они делают для меня непостижимым то~единственное, что я~хочу понять. Если крылатые ракеты настигают меня в~винограднике, то~я не~хочу преклоняться перед <<разумом>>, <<ошибающейся физикой>>, <<людской глупостью>>, <<жестокостью Бога>> или <<Realpolitik>>. Я~не~хочу обращаться к~путанным объяснениям, говорящим о~могуществе, когда причина моей смерти кроется в~силе фактов. В~те~несколько секунд, которые разделяют иллюминацию от~радиации, я~хочу быть агностиком настолько, насколько это возможно для человека, который присутствует при смерти первого Просвещения, агностиком настолько, насколько это возможно для человека, который достаточно уверен в~божественном и~знании, что осмеливается верить в~рождение нового Просвещения. Я~не~уступлю им; я~не~буду верить <<наукам>> наперед; и~после я~не~потеряю веру в~знание, когда одно из~отношений силы, в~которое внесли свой вклад лаборатории, разорвется над Францией. Не~вера, и~не~отчаяние. Я~буду настолько агностиком и~настолько честным, насколько это будет возможно.
	\end{itemize}