\chapter{Антропологики}

\paragraph{3.1.1}\hypertarget{par:3.1.1}{} Как обстоят дела? (How do~things stand?) О~каких актантах мы~говорим? Эти энтелехии, чего они хотят? Они сами борются за~то, что бы~ответить на~эти вопросы. Выбрать ответ, значит усилить одного и~ослабить другого. 
	\begin{itemize}
	\item 
	Каждый актант создает для себя целый мир (\hyperlink{par:1.2.8}{1.2.8}). Кто мы? Что мы~можем знать? На~что мы~можем надеяться? Ответы на~эти громкие вопросы определяют и~изменяют их~формы и~границы (\hyperlink{par:1.1.6}{1.1.6}).
	\end{itemize}

\paragraph{3.1.2}\hypertarget{par:3.1.2}{} Я~не~знаю, как обстоят дела. Я~не~знаю ни~кто я, ни~чего я~хочу, но~{\itshape другие} говорят от~моего имени, что знают, другие, которые определяют меня, связывают меня, заставляют меня говорить, интерпретируют то, что я~говорю, и~вербуют меня. Будь я~штормом, крысой, скалой, озером, львом, ребенком, рабочим, геном, рабом, бессознательным или вирусом они шепчут мне, они предлагают, они навязывают интерпретацию того, кто я, и~кем бы~мог быть.

%\subparagraph{Интерлюдия IV: Объясняющая, почему вещи-в-себе хорошо обходятся без какой-либо помощи с~нашей стороны.}

\paragraph{3.1.3}\hypertarget{par:3.1.3}{} Те, кто говорит всегда говорят о~тех, кто не~говорит сам. Они говорят о~нем, об~этом, о~нас, о~вас \ldots о~том, кто это, что оно хочет, когда случится нечто иное. Те, кто говорят, связаны с~теми, о~ком они говорят многими способами. Они действуют как делегаты, переводчики, аналитики, толкователи, предсказатели, наблюдатели, журналисты, гадалки, социологи, поэты, представители, родители, опекуны, пастыри, любовники. 
	\begin{itemize}
	\item 
	Гоббс говорил о~<<персоне>>, <<маске>> или <<актере>>, когда рассуждал о~тех, кто говорит от~имени безмолвных. Существует множество масок и~не~все из~них известны хранителям Музеев Антропологии.
	\end{itemize}

\paragraph{3.1.4}\hypertarget{par:3.1.4}{} Каждый актант решает, кто будет говорить и~когда. Есть те, кому он~позволяет говорить, те, от~имени которых он~говорит, те, к~кому он~обращается. Наконец, есть те, кого заставили молчать или те, кому позволяют общаться исключительно жестами или
симптомами. 
	\begin{itemize}
	\item 
	Энтелехии не~могут быть поделены на~<<одушевленные>> и~<<неодушевленные>>, <<людей>> и~<<не-человеков>>, <<объект>> и~<<субъект>>, поскольку это разделение является одним из~тех самых способов, с~помощью которого одна сила может соблазнить другие. Мы~можем заставить каменных истуканов ходить, отрицать наличие души у~черных, говорить от~имени китов, или заставить Полюса проголосовать. Акторов всегда {\itshape можно заставить сделать} нечто подобное, даже если то, что они бы~сделали или сказали, если бы~их предоставили самим себе, является тайной. (Возможно, они вообще не~были бы~<<черными>>, <<китами>>, <<Полюсами>> или <<истуканами>>.)
	\end{itemize}

\paragraph{3.1.5}\hypertarget{par:3.1.5}{} Сила практически всегда окружена властями (мощностями, powers)~--- голосами, которые говорят от~имени безмолвных толп (\hyperlink{par:1.5.0}{1.5.0}). Эти власти определяют, соблазняют,
используют, проектируют, перемещают, считают, инкорпорируют и~прерывают эту силу. Вскоре становится невозможным различить (\hyperlink{par:1.5.1}{1.5.1}) между тем, что говорит сама сила, что она говорит о~себе, что о~ней говорят власти, и~что представляемые этими властями толпы хотели бы, чтобы они сказали. 
	\begin{itemize}
	\item 
	Заключая союзы со~словами, текстами, бронзой, сталью, местами или эмоциями, мы~приходим к~различению форм, которые можно классифицировать, по~крайней мере в~мирное время. Но~эти классификации не~могут надолго устоять перед мародерством других акторов, которые раскладывают вещи иначе.
	\end{itemize}

\paragraph{3.1.6}\hypertarget{par:3.1.6}{} Все что угодно может быть редуцировано к~тишине и~все можно заставить говорить. Таким образом, любая сила может обратиться к~неистощимому источнику акторов, {\itshape от~лица которых можно говорить}.
	\begin{itemize}
	\item 
	Этнологи показали нам, как можно заставить говорить останки, свернувшееся молоко, дым, предков или ветер. Их~робость помешала им~увидеть, как другие гораздо ближе к~дому заставляют говорить окаменелости, химические осадки, промокательную бумагу, гены и~торнадо. Разумеется, психоаналитики говорят о~словоохотливом <<бессознательном>>, но~его репертуар обедняется и~комбинируется в~соответствии с~очень малым количеством правил. Вдобавок ко~всему, психоаналитики склонны говорить, что подсознательное имеет только <<субъективный>> смысл. Таким образом, все, что нам нужно сделать, это почитать <<Таймс>> и~увидеть как много акторов помимо бессознательного, которых заставляют говорить бесконечным количеством иных способов. Тут легионы ангелов мобилизованы для того, чтобы пресечь зло, там тысячи страниц распечатаны на~компьютере для того, чтобы остановить АЭС; на~следующей странице молчаливое большинство доведено до~крика от~имени нерожденных детей; несколькими страницами ранее мертвых возвернули к~жизни для того, чтобы остановить осквернение кладбища; на~последней полосе представители китов прервали смертельную миссию Японского судна.
	\end{itemize}

\paragraph{3.1.7}\hypertarget{par:3.1.7}{} По~определению не~существует {\itshape преданных} (верных) представителей (\hyperlink{par:2.2.1}{2.2.1}), поскольку они говорят то, что их~избиратели не~говорили и~говорят вместо них (\hyperlink{par:3.1.3}{3.1.3}). Всякая власть, поэтому, может быть {\itshape редуцирована к~своему простейшему выражению}. Все что необходимо, так это то, чтобы все акторы, от~имени которых говорит власть, высказались по-очереди. Тогда каждый из~акторов скажет сам, чего он~хочет без цензуры и~без подсказки. У~сил, которые редуцируют и~провоцируют друг друга, нет жалости. <<Ты говоришь от~их имени, но~что если я~поговорю с~ними сам, что они скажут мне?>>

\paragraph{3.1.8}\hypertarget{par:3.1.8}{} Существует только один способ, с~помощью которого актор может доказать свою власть. Он~должен заставить тех, от~чьего имени он~говорит, {\itshape говорить} и~показывать, что они говорят {\itshape одно и~тоже}. Как только это будет сделано, актор может сказать, что он~не говорит, но~четно <<выражает>> (канализирует, транслирует) взгляды других. 
	\begin{itemize}
	\item 
	Профсоюзы проводят демонстрации при помощи своих членов так же, как лаборатория Скиннера проводит демонстрации при помощи крыс. В~каждом из~случаев должно быть видно, что участники демонстрации и~крысы сами говорят то~же самое как и~тогда, когда их~побуждали говорить. А~что касается ангелов и~бесов, то~есть тысяча способов обнаружить их~знаки~--- очевидцы, стигматы или чудеса~--- которые смягчат  ожесточенное сердце.
	\end{itemize}

%\paragraph{3.1.9}\hypertarget{par:3.1.9}{}

\paragraph{3.1.10}\hypertarget{par:3.1.10}{} Поскольку представитель всегда говорит {\itshape нечто другое}, чем те, которых он~побуждает говорить, и~поскольку всегда существует необходимость вести переговоры относительно сходства и~отличия (\hyperlink{par:1.2.1}{1.2.1}), постольку {\itshape всегда есть место} для спора о~правильности всякой интерпретации. Сила всегда может втереться между спикером и~теми, кого он~заставляет говорить. Она всегда может заставить их~сказать что-нибудь еще. 
	\begin{itemize}
	\item 
	Участники демонстрации не~говорили, что они хотят сорокачасовую рабочую неделю~--- просто их~пришло тысячи; крысы не~говорили, что у~них есть условные рефлексы~--- они просто цепенели от~ударов током. Следовательно, другие могут вмешаться. Присутствие рабочих можно перевести, сказав, что <<им заплатили профсоюзы>>, а~неподвижность крыс можно интерпретировать как <<экспериментальный артефакт>>.
	\end{itemize}

\paragraph{3.1.11}\hypertarget{par:3.1.11}{} У~этих споров нет {\itshape естественного завершения}. Они всегда могут возобновиться (\hyperlink{par:3.1.6}{3.1.6}). Единственный способ прекратить эти споры, это не~дать другим актантам сбить с~пути тех, кто был завербован, и~превратить их~в предателей. В~конечном счете, интерпретации всегда стабилизируются посредством массива {\itshape сил}.

\paragraph{3.1.12}\hypertarget{par:3.1.12}{} Сила становится могущественной, только если она может {\itshape говорить за} других; если она может заставить тех, кого она заглушила, {\itshape говорить} тогда, когда она вынуждена продемонстрировать свою мощь; и~если она может вынудить тех, кто бросил ей~вызов, {\itshape признать}, что она действительно говорит то, что сказали бы~ее союзники. 
	\begin{itemize}
	\item 
	Профсоюзы не~могут запретить своим противникам, правым интерпретировать демонстрации по-другому. Скиннер не~может воспрепятствовать тому, что его <<дорогие коллеги>> интерпретировали его эксперимент иным способами. Если бы~они могли, они, конечно же, сделали бы~это, но~поскольку все так, как есть, они не~могут. Другие бы~уничтожили их, если бы~они попытались.
	\end{itemize}

\paragraph{3.2.1}\hypertarget{par:3.2.1}{} Каково положение дел? Что будет дальше? Каково соотношение сил? Используя множества, которые они заставляют говорить, некоторые актанты становятся достаточно влиятельными для того, чтобы определять, на~недолгий срок и~локально, для чего все это. Они делят актантов, сортируют их~по ассоциациям, называют сущности, наделяют эти сущности волей или функцией, направляют эти воли или функции к~цели, принимают решение о~том, как определить, что эти цели были достигнуты и~так далее. Мало-помалу они все увязывают вместе. Все придает силы энтелехии, у~которой нет сил, и~целое становится <<логичным>> и~<<устойчивым>>~--- другими словами {\itshape сильным} (\hyperlink{par:2.1.8}{2.1.8}).
	\begin{itemize}
	\item 
 Я~не~пытаюсь избежать ответа на~вопрос: <<Каков баланс сил?>> Однако мы~должны расчистить почву для того, чтобы {\itshape все} ответы смогли проявиться.
	\end{itemize}

\paragraph{3.2.2}\hypertarget{par:3.2.2}{} Ни~один из~актантов, мобилизованных для закрепления союза, не~прекращает действовать в~своих интересах (\hyperlink{par:1.3.1}{1.3.1}, \hyperlink{par:1.3.4}{1.3.4}). Все они продолжают плести свои интриги, создавать свои группы, служить другим господам (хозяевам), волям и~функциям. 
	\begin{itemize}
	\item 
	Силы всегда склонны к~восстанию (\hyperlink{par:1.1.1}{1.1.1}); они одалживают себя, но~не отдают (\hyperlink{par:1.5.1}{1.5.1}). Это верно для дерева, которое прорастает снова; для саранчи, которая уничтожает урожай; для рака, который поражает других их~же собственным оружием, для мулл, которые разрушили Персидскую империю; для сионистов, которые ослабили власть мулл; для треснувшего бетона электростанции; для голубых акриловых красок, которые поглощают другие пигменты, для льва, который не~следует пророчествам оракула~--- все они имеют иные цели и~судьбы, которые не~могут быть {\itshape суммированы}. В~тот момент, когда мы~поворачиваемся спиной, наши ближайшие друзья встают под другие знамена.
	\end{itemize}

\paragraph{3.2.3}\hypertarget{par:3.2.3}{} Каким образом можно удержать от~разговоров тех, от~чьего имени мы~говорим? Каким образом те, кто был рекрутирован благодаря счастливой случайности, могут быть сцементированы в~единый блок? Как можно усмирить мятежников и~диссидентов? Есть ли~где-нибудь хоть {\itshape одна} сущность, которая не~должна решать эти проблемы? Ответ всегда один и~тот же, поскольку существует только один источник силы: тот, что проистекает из~объединения (\hyperlink{par:1.3.2}{1.3.2}). Но~как могут быть ассоциированы мятежники? [Этого можно добиться] {\itshape Путем поиска дополнительных союзников}, которые заставят других держаться вместе, и~так далее, до~тех пор, пока градиент неопределенных объектов не~подойдет к~концу и~не~сделает авангард альянса способным к~сопротивлению и~таким образом реальным (\hyperlink{par:1.1.2}{1.1.2}).
	\begin{itemize}
	\item 
	Понятие системы бесполезно для нас, поскольку система это конечный продукт наладки, а~не~исходная точка (\hyperlink{par:2.1.4}{2.1.4}). Для того, чтобы системы существовали, сущности должны быть четко определены, тогда как на~практике никогда не~бывает так; функции должны быть ясны, тогда как большинство акторов не~определились, хотят ли~они командовать или подчиняться; обмен эквивалентами между сущностями и~подсистемами должен быть оговорен, тогда как повсюду ведутся споры о~пропорции и~направлении обмена. Системы не~существуют, но~систематизация довольно распространена; повсюду есть силы, которые обязывают других играть так, как они играли всегда (\hyperlink{par:1.1.13}{1.1.13}). 
	\end{itemize}

\paragraph{3.2.4}\hypertarget{par:3.2.4}{} У~каждого актора, когда он~ассоциирует элементы, есть выбор: расширяться дальше, не~боясь (risking) диссидентства и~диссоциации, или усилить согласованность и~прочность, но~не идти слишком далеко. 

\paragraph{3.2.5}\hypertarget{par:3.2.5}{} Вполне определенное положение дел~--- это работа {\itshape многих сил}. Они не~согласны ни~в чем и~ассоциируются только посредством длинных сетей, в~которых они непрестанно говорят, не~будучи в~состоянии резюмировать друг друга. Они смешиваются, но~они не~могут выйти за~собственные пределы, чтобы понять то, что связывает их, противостоит им, и~резюмирует их. Однако, несмотря ни~на что, сети усиливают друг друга и~сопротивляются разрушению. Надежные и~все же~хрупкие, изолированные и~все же~переплетенные, ровные и~все же~скрученные вместе энтелехии их~странных фабрик. Вот как мы~представляли себе <<традиционные миры>>, как бы~далеко назад мы~ни смотрели. 
	\begin{itemize}
	\item 
 Я~не~говорю о~<<культуре>>, потому что это слово было припасено Западом для описания одной из~обособленных сущностей, конституирующей <<человека>>. Силы не~могут быть поделены на~<<человеческие>> и~<<нечеловеческие>>. Я~не~говорю об~<<обществе>>, поскольку ассоциации, которые меня интересуют, не~ограничены теми немногими, которые допускает понятие <<социального>>. Кроме того, я~не~говорю о~<<природе>>, потому что тех, кто говорит от~имени групп крови, хромосом, водяных испарений, тектонических плит или рыб можно лишь временно и~локально отличить от~тех, кто от~имени крови, мертвых, реки, ада, и~рыбы. Я~бы~согласился на~слово <<бессознательное>>, если бы~мы были достаточно восприимчивыми для того, чтобы обозначать им~вещи-в-себе.
	\end{itemize}

\paragraph{3.3.1}\hypertarget{par:3.3.1}{} Для того, чтобы распространятся далеко, не~теряя связности, актанту нужны верные союзники, которые принимают то, что им~говорят, идентифицируют себя со~своей причиной, выполняют все функции, которые им~определены, и~без колебаний приходят на~помощь к~актанту, когда их~зовут. Поиск таких идеальных союзников занимает пространство и~время тех, кто хочет быть сильнее, чем другие. Как только актор нашел {\itshape в~известной степени более верного союзника}, он~может заставить другого союзника в~свою очередь быть {\itshape более верным}. Он~создает градиент, который обязывает других союзников принять форму и~сохранять ее~до поры до~времени (\hyperlink{par:1.1.12}{1.1.12}).
	\begin{itemize}
	\item 
 Мы~потратили много времени на~поиски чего-либо оказывающегося жестче для того, чтобы придать форму тому, что мягче~--- камня, который служит наковальней, биопробы измеряющей уровень эндорфинов в~крови, языка коровы, позволяющего вирусу проникнуть в~костный мозг, закона, сдерживающего аппетит лоббистов, лоббистов для того, чтобы изменить закон. Слово <<технология>> неудовлетворительно, потому что оно слишком долго было ограничено исследованием тех силовых линий, которые принимали форму гаек и~винтов.
	\end{itemize}

\paragraph{3.3.2}\hypertarget{par:3.3.2}{} Если мы~хотим не~дать силам трансформироваться, в~тот момент, когда мы~поворачиваемся спиной, мы~должны избегать того, чтобы поворачиваться спиной! Власти всегда мечтают о~том, чтобы быть повсюду, даже если они далеко или их~уже давно нет. Как они могут присутствовать, когда другие силы вытеснили их~(\hyperlink{par:1.2.5}{1.2.5})? Как они могут расширяться, когда все локализует их? Как они могут быть там и~где-либо еще, сейчас и~навсегда? О, потенция мифа о~потенции! Все, что в~[данный момент] помогает данной структуре сохраняться после момента ухода силы, будет помогать [и далее].

\paragraph{3.3.3}\hypertarget{par:3.3.3}{} Когда сила нашла союзников, которые позволяют ей~надежным способом скрепить ряды других сил, она может вновь расширяться. Именно потому, что верные [союзники] скреплены такими прочными связями, сила может уходить без страха. Даже, когда ее~там нет, все произойдет так, как если бы~она там была. В~конечном счете, это просто собрание сил, которые работают на~нее, но~без нее.
	\begin{itemize}
	\item 
	Иногда мы~называем эти махинации сил <<механизмами>>. Это неудачно выбранный термин~--- потому что он~подразумевает, что все силы механические, тогда как большинство из~них таковыми не~является; потому что он~придает особое значение несущим конструкциям (hardware) за~счет более мягких отношений; и~потому что он~предполагает, что они сделаны человеком и~искусственны, несмотря на~то, что их~генеалогия является именно тем, что поставлено на~карту.
	\end{itemize}

\paragraph{3.3.3.1}\hypertarget{par:3.3.3.1}{} Обретение могущества (potency) всегда является делом противопоставления сил друг другу. Власть, которая проистекает из~всего массива, затем приписывается {\itshape последней} силе, захваченной всеми остальными. 
	\begin{itemize}
	\item 
	Причина, по~которой я~говорил о~силе с~самого начала должна быть теперь ясна. Это была не~попытка {\itshape распространить} технические метафоры на~философию. Напротив, сила машин и~автоматизмов лишь редко и~локально достигает своей цели. Только тогда, когда мы~игнорируем все остальные силы, благодаря которым они оказываются {\itshape последними в~цепи}, мы~можем говорить о~<<технике>>. Мотор, урчащий под кожухом это лишь одна из~возможных форм, принимаемых заговором сил. Дизель надеялся оптимизировать производительность социальных организаций так же, как он~это сделал для двигателя внутреннего сгорания. Должен был быть такой же~мотор, такое же~исследование, такая же~оптимизация: давление, смесь, регенерация, производительность.
	\end{itemize}

\paragraph{3.3.3.2}\hypertarget{par:3.3.3.2}{} В~этих махинациях нет ничего особенного, не~считая этого макиавеллевского предписания: собрать как можно большее количество союзников {\itshape внутри} и~оставить тех, в~ком мы~сомневаемся {\itshape снаружи}. Благодаря этому мы~имеет новое разделение между жестким и~мягким.
	\begin{itemize}
	\item 
	Те, кто захвачен этим разделение говорит о~<<техническом>> и~<<социальном>>, не~осознавая того, что <<социальное>> может быть оставлено в~стороне как стружка из-под рубанка плотника. Каждый проект может быть прочитан как еще один <<Государь>>: скажи мне, каковы твои допустимые отклонения, исходные пункты, твои калибровки, патенты, от~которых ты~уклонился и~уравнения, которые ты~выбрал, и~я скажу тебе, кого ты~боишься, на~чью поддержку ты~надеешься, кого ты~решил избегать или игнорировать, и~над кем ты~желаешь господствовать\footnote{{\itshape Mick{\`e}s Coutouzis}. Soci{\'e}t{\'e}s et~techniques en~voie de~d{\'e}placement: Ie~transfert d'un village solaire des Etats Unis en~Cr\^ete. (Paris: Universite Dauphine, 1983).}.
	\end{itemize}

\paragraph{3.3.4}\hypertarget{par:3.3.4}{} И~все же~вы не~можете удержать силы от~того, чтобы играть друг против друга. Нет заговора, колдовства, логики, аргумента или машины, которая бы~уберегла мобилизованных актантов от~того, чтобы кипеть и~пениться, когда находятся в~поисках других целей и~альянсов. Самая безличная машина переполнена в~большей степени, чем пруд с~рыбой.
	\begin{itemize}
	\item 
	Вопреки Лейбницу, в~ходе часов также есть пруды полные рыбы и~рыбы полные прудов. Конечно, всегда возможно найти людей, которые скажут, что машины холодны, безличны, бесчеловечны или стерильны. Но~посмотрите на~чистейший сплав: его также предают повсюду, как и~все остальные наши альянсы. Запад всегда верил в~то, что моторы <<чисты>> также как аргументы <<логичны>>, а~слова <<буквальны>>. Вот что капитан сказал Крузо прямо перед самым кораблекрушением: <<Остерегайся чистоты. Это язва (vitriol) души>> (Интерлюдия IV).
	\end{itemize}

\paragraph{3.3.5}\hypertarget{par:3.3.5}{} Для того, чтобы существовать самому, актант должен программировать других актантов таким образом, чтобы они не~смогли его предать (\hyperlink{par:3.3.3}{3.3.3}), несмотря на~тот факт, что они близки к~этому (\hyperlink{par:3.3.4}{3.3.4}). Есть только один способ разрешить это затруднительное положение: так как ни~одна индивидуальная связь не~является надежной, то~актанты должны поддерживать друг друга; в~тот момент, когда множество связей выстроено ярусами, они становятся реальностью.
	\begin{itemize}
	\item 
	Так как не~существует ничего кроме слабостей, то~власть это всегда впечатление. Однако данное впечатление это все что требуется дл~того, чтобы изменить форму вещей, ин{\itshape{формируя}} или в{\itshape{печатляя}} их. Это то~самое чудо, которое следует объяснить.
	\end{itemize}

\paragraph{3.3.6}\hypertarget{par:3.3.6}{} {\itshape Мы~всегда неправильно истолковываем силу сильного.} Хотя люди приписывают ее~безупречности актанта, она существует неизменно благодаря иерархизированному массиву слабостей.

%\subparagraph Интерлюдия V: Где мы~с большим удивлением узнаем, что не~существует такой вещи как современный мир.

\paragraph{3.4.1}\hypertarget{par:3.4.1}{} Как нам следует говорить обо всем том, что сохраняет единство? Должны ли~мы говорить об~экономике, законе, механизмах, языковых играх, обществе, природе, психологии или о~системе, которая охватывает их~всех?
	\begin{itemize}
	\item 
 В~фильмах про Джеймса Бонда всегда есть одна черная кнопка, которая может свести на~нет все махинации злого гения, кнопка, которую герою, переодетому техником удается нажать в~конце. Замаскированный и~переодетый в~белый халат, философ достигает точки, где совпадают абсолютное могущество и~абсолютная хрупкость.
	\end{itemize}

\paragraph{3.4.2}\hypertarget{par:3.4.2}{} Это не~предмет {\itshape экономики}. Она использует эквиваленты, не~зная того, кто установил эквиваленты, и~бухгалтерию, не~зная того, кто измеряет и~считает. Экономика ступает в~дело {\itshape после} того, как инструменты измерения были расставлены по~местам~--- инструменты, которые делают возможным определять цены и~вступать в~обмен. Будучи далекой от~того, чтобы освещать испытания сил, экономика маскирует и~подавляет их. В~лучшем случае она является способом регистрировать эти испытания, как только они стабилизируются.
	\begin{itemize}
	\item 
	Стоит установить инструмент измерения, и~мы~можем вести экономику и~считать, экономить и~копить. Другими словами, мы~можем убеждать и~обогащаться. Но~экономисты не~говорят о~том, каким образом эти инструменты были установлены в~первую очередь.
	\end{itemize}

\paragraph{3.4.2.1}\hypertarget{par:3.4.2.1}{} Всеобщая экономия~--- подсчет удовольствия, генов или прибыли~--- невозможна. Необходимо бы~было раскрыть тех, кто ведет переговоры, тех, кто платит, тех, кто выиграл и~проиграл, во~что обошлись вознаграждения и~когда следует закрыть счет. 

\paragraph{3.4.3}\hypertarget{par:3.4.3}{} Это не~предмет {\itshape права}. Оно представляет собой храповый механизм, который, как и~любой другой (\hyperlink{par:1.1.10}{1.1.10}), позволяет актанту сделать временное занятие позиции необратимым. Закон делают сильным не~только тексты, но~также паралич тех, кто не~осмеливается пересечь то, что по~их убеждению <<потенциально>> заложено в~его скрижалях (священных писаниях); то~есть разрыв между законом и~силой или законом и~фактом. Если мы~обладает такой властью, то~мы можем устрашить других и~распространить себя в~новые места, невзирая на~противодействие. Сила закона проистекает не~из него самого, но~от нищей презренной толпы, дающей ему силу факта: нравов, слов, дубинок полицейских, надежд, администраций, стен, телексов, файлов, финансов, зол.

\paragraph{3.4.4}\hypertarget{par:3.4.4}{} Это не~вопрос {\itshape машин} или {\itshape механизмов}. Они никогда не~существовали без механики, изобретателей, финансистов и~машинистов. Машины это тайные объекты желания актантов, которые настолько эффективно приручили силы, что они более не~выглядят как
силы. Результат состоит в~том, что актанты подчиняются даже тогда, когда сил там нет (\hyperlink{par:3.3.3}{3.3.3}).
Многие люди грезили о~таких машинах, которые можно было бы~распространить на~все виды отношений, но~этот сон всегда преследует ночной кошмар: восстание саботирующих актантов, которые расставляют ловушки для самый гладко работающих машин. Сила машин черпается из~других сил, которые становятся их~частью~--- сил, которых другие презирают и~подавляют; сил, которые представляют собой разобщенную плебейскую толпу из~низших классов.

\paragraph{3.4.5}\hypertarget{par:3.4.5}{} это не~вопрос {\itshape языка} или языковых игр (\hyperlink{par:2.3.0}{2.3.0}, \hyperlink{par:2.4.3}{2.4.3}, \hyperlink{par:2.4.4}{2.4.4}). Слова не~обладают властью, но~заимствуют свою силу у~компромиссов, которые далеки от~<<изящной словесности>> (художественной литературы).

\paragraph{3.4.6}\hypertarget{par:3.4.6}{} Это не~предмет {\itshape науки}. Если бы~аргументы были суверенными, они бы~обладали всем могуществом (potency) подагрического монарха, заточенного в~разрушающемся замке. Если наука и~развивается, то~это потому, что ей~удалось убедить множество актантов сомнительного происхождения одолжить ей~их силу: крыс, бактерий, промышленников, мифов, газ, червей, специальных сплавов, страстей, учебников, мастерских~\ldots толпу невежд, чья помощь отрицается даже в~то~время, когда она используется. 
	\begin{itemize}
	\item 
	Высшая школа фактов нередко также является и~школой высокомерия (самонадеянности). Просвещение ведет к~наиболее грубой форме обскурантизма.
	\end{itemize}

\paragraph{3.4.7}\hypertarget{par:3.4.7}{} Это не~вопрос {\itshape общества}. Значение понятия <<социальное>> постоянно сжимается~--- теперь оно сведено до~уровня <<социальных>> проблем. Это то, что остается, когда остальное было поделено властьимущими (powerful); все, что не~является ни~экономическим, ни~техническим, ни~правовым, ни~каким-либо еще оставляется социальному. Неужели мы~действительно надеемся связать все вместе с~помощью такой обедненной версии социального? Как майонез, который никто не~берет, оно грозит прокиснуть. <<Социальное>>~--- его акторы, группы, стратегии~--- слишком тесно идентифицируется с~человеческими существами для того, чтобы обратить внимание на~беспомощную нечистоту и~аморальность альянсов.
	\begin{itemize}
	\item 
	Если бы~{\itshape социология} (как подсказывает ее~имя) была наукой об~ассоциациях, а~не~наукой о~социальном, к~которой она была редуцирована в~девятнадцатом веке, тогда, возможно, мы~были бы~счастливы назвать себя <<социологами>>.
	\end{itemize}

\paragraph{3.4.8}\hypertarget{par:3.4.8}{} Это не~вопрос {\itshape интерсубъективных отношений}. Только в~наши дни мы~можем надеяться найти людей убогих настолько, чтобы пытаться объяснить ядерные реакторы, национальные государства или биржевые сделки на~основе <<интеракций>>. Психология и~ее~брат психоанализ, полагают себя богатыми в~их~безграничной нищете. Об~этой точке зрения не~следует говорить ничего, кроме того, что ее~не следует придерживаться. Уклончивый (who shrinks) психоаналитик (shrink) не~может распространяться по~поводу того, чтобы объяснить остальных (\hyperlink{par:2.5.6.2}{2.5.6.2}).
	\begin{itemize}
	\item 
 В~глубинке всегда находились пристанища для тех, кто хочет строить соборы из~спичек или шариковых ручек.
	\end{itemize}

\paragraph{3.4.9}\hypertarget{par:3.4.9}{} Это не~проблема {\itshape природы} (\hyperlink{par:3.2.5}{3.2.5}). Попробуйте осмыслить эти серии: пятна на~солнце, тальвеги\footnote{Тальвег (нем. {\itshape Talweg}, от~{\itshape Tal} <<долина>> и~{\itshape Weg} <<дорога>>)~--- линия, соединяющая наиболее пониженные участки дна реки, долины, балки, оврага и~др. вытянутых форм рельефа. Тальвег в~плане обычно представляет собой относительно прямую или извилистую линию. В~более широком смысле тальвег~--- дно долины~--- {\itshape Прим. перев.}}, антитела, спектральный анализ углерода; рыба, подстриженные живые изгороди, пустынные пейзажи; <<le petitshape pan de~mur jaune>>, горные ландшафты нарисованные китайской тушью, лес трансептов\footnote{Трансепт (от позднелатинского {\itshape transeptum} из~лат. {\itshape trans} <<за>> и~лат. {\itshape septum} <<ограда>>)~--- поперечный неф в~базиликальных и~крестообразных в~плане храмах, пересекающий под прямым углом основной (продольный) неф.~--- {\itshape Прим. перев.}}; львы, которые ночью превращаются в~людей, богини матери из~слоновой кости, тотемы из~черного дерева.

Видите? Мы~не можем редуцировать количество или гетерогенность альянсов таким способом. {\itshape Природы} смешиваются друг с~другом и~с <<нами>> так основательно, что мы~не можем надеяться отделить их~и открыть чистые, уникальные истоки их~власти (Интерлюдия IV).

\paragraph{3.4.10}\hypertarget{par:3.4.10}{} Это не~проблема {\itshape систем} (\hyperlink{par:3.2.3}{3.2.3}). Поскольку люди знают, что исток власти не~покоится в~чистоте сил, они помещают его в~<<систему>> чистых сил. Это мечта постоянно возрождается. Право привязывается (is attached) к~экономике, биологии, языку, обществу, кибернетике\ldots Нарисованы красивые ящички, дополненные красиво расставленными стрелочками. К~несчастью для тех, кто делает системы, акторы остаются неподвижными достаточно долго, чтобы сделать групповое фото; ящики переполняются; стрелки изгибаются и~разрываются; право просачивается в~биологию, которая растворяется в~обществе. Нет, альянсы выковываются не~в отношениях {\itshape между} приятными и~абстрактными сторонами, а~в беспорядочном и~разнородном конфликте, который ужасен для тех, кто почитает чистоту.

\paragraph{3.5.1}\hypertarget{par:3.5.1}{} Мы~всегда неправильно истолковываем действенность сил: мы~приписываем им~то, что им~лишь одолжили (\hyperlink{par:1.5.1}{1.5.1}). Мы~считаем их~чистыми, тогда как они были бы~совершенно немощны, если бы~это случилось. Когда мы~смотрим на~то, как они работают, мы~хлам, который не~может быть подытожен (приведен к~единому знаменателю). Всякая сеть разрежена, пуста, хрупка и~гетерогенна. Она становится сильной только, если она расширяется и~выстраивает в~боевой порядок слабых союзников. 
	\begin{itemize}
	\item 
 С~чем мы~можем сравнить слабости, которые создают силу? С~макраме. Есть ли~такой узел, который связывает человека с~человеком, нейроны с~нейронами, лист железа с~листом железа? Нет. Веревка для этого Гордиева узла еще не~была свита. Но~каждый день мы~видим собственными глазами макраме из~нитей разных цветов, материалов, происхождений, длин, к~которым прикреплены наиболее ценимые нами блага.
	\end{itemize}

\paragraph{3.5.2}\hypertarget{par:3.5.2}{} Можем ли~мы все сети описать одним и~тем же~способом? Да, потому что <<мира Нового Времени>> нет.
Годами этнографы говорили, что невозможно изучать <<примитивные>> или древние народы, если мы~разделяем право, экономику, религию, и~все остальное. Напротив они доказывали, что эти слабо связанные смешения могут быть поняты, только если мы~пристально посмотрим на~места, семьи, обстоятельства и~сети. Но~когда они говорят о~своих собственных странах, он~прибегают к~разделению сфер и~уровней.
 

\paragraph{3.5.3}\hypertarget{par:3.5.3}{} <<Мир Нового Времени>> это наклейка на~кнопке, которая объединяет в~себе абсолютное могущество и~абсолютную немощь (\hyperlink{par:3.4.1}{3.4.1}). Гетерогенное и~локальное применение слабостей становится системой властей с~такими престижными именами как природа, экономика, право и~техника.
	\begin{itemize}
	\item 
	Как те, кто ненавидит мир Нового Времени, так и~его фанатичные приверженцы, изобрели для его описания больше терминов, чем благочестивец нашел для прославления имя Господа. На~каждый из~этих призывов они отвечают либо <<Изыди, сатана>>, либо <<Услышь мои молитвы>>:

		\begin{verse}
		мир Нового Времени \\
		секуляризация  \\
		рационализация \\
		анонимность \\
		разочарование \\
		меркантилизм \\
		оптимизация \\
		дегуманизация \\
		механизация \\
		вестернизация \\
		капитализм \\
		индустриализация \\
		постиндустриализация \\
		технологизация \\
		интеллектуализация \\
		стерилизация \\
		объективизация \\
		американизация \\
		сциентизация \\
		общество потребления \\
		одномерное общество \\
		бездушное общество \\
		современное безумие \\
		современность \\
		прогресс
		\end{verse}

	<<Услышь мою молитву>>. <<Изыди, Сатана>>. Каждый из~этих лозунгов скрывают работу, проводимую силами, и~делают невозможным антропологию того, что существует здесь и~сейчас. Хотя это действительно очень просто: не~существует модерного мира, а~если и~существует, то~это просто стиль, как когда мы~говорим <<стиль модерн>>.
	\end{itemize}


%\subparagraph{Интерлюдия VI: В~которой автор, теряя самообладание, утверждает, что те, кто осуществляют редукцию~--- предатели.}

\paragraph{3.5.4}\hypertarget{par:3.5.4}{} К~счастью мир больше не~расколдован как это было раньше, машины больше не~полируются, мышление не~является больше точным, а~обмены не~организованы лучше.
Как мы~можем говорить о~<<модерном мире>>, когда его действенность зависит от~идолов: денег, закона, разума, природы, машин, организаций или лингвистических структур? Мы~уже использовали слово <<магия>> (\hyperlink{par:2.1.11}{2.1.11}). Поскольку истоки власти <<модерного мира>>истолковываются неправильно, и~действенность приписывается вещам, которые ни~двигаются, ни~говорят, мы~еще раз можем сказать о~магии (\hyperlink{par:4.1.0}{4.1.0}).

\paragraph{3.5.5}\hypertarget{par:3.5.5}{} То, что мы~с удовольствием называем <<другими культурами>> имеет множество секретов; наша же~возможно имеет только один. Вот почему <<другие культуры>> представляются нам таинственными и~стоят того, чтобы их~познавал, в~то~время как наша~--- выглядит равным образом непознаваемой и~лишенной загадки. Секрет, который является тем единственным, что отличает нашу культуру от~остальных, состоит в~том: что она и~только она одна не~является одной культурой из~многих (рядовой культурой, такой же~культурой, как и~многие другие). Наша вера в~модерный мир возникает из~этого отрицания. Для того чтобы избавиться от~нее, мы~все должны соединить вместе то, что мы~обычно разделяем, когда говорим о~сами себе. Мы~должны стать антропологами нашего собственного мира.

\paragraph{3.6.1}\hypertarget{par:3.6.1}{} О~чем все это? Каково положение дел? Некто говорит от~имени тех, кто не~говорит ничего, и~отказывает моим просьбам, определяя меня к~немым. Если его отказ убедит меня, я~больше не~смогу разобраться почему [он отказал], поскольку он~заставил слишком
много помощников поддержать его.

\paragraph{3.6.2}\hypertarget{par:3.6.2}{} Все происходит так, как будто нет никаких испытаний сил, но~только странная фантазия: <<человек>> <<открывающий>> <<природу>>!

\paragraph{3.6.3}\hypertarget{par:3.6.3}{} Только в~политике люди готовы говорить об~<<испытаниях сил>>. Политики~--- это козлы отпущения, жертвенные агнцы. Мы~высмеиваем, презираем и~ненавидим их. Мы~соревнуемся в~суждении их~продажности и~некомпетентности, узколобости, их~схем и~компромиссов, их~ошибок, их~прагматизма или достатка реализма, их~демагогии. Считается, что только в~политике испытания сил определяют форму вещей (\hyperlink{par:1.1.4}{1.1.4}). Только политиков считают бесчестными, блуждающими в~темноте (who are held to~grope in~the dark). 
	\begin{itemize}
	\item 
	Потребуется нечто подобное отваге для того, чтобы предположить, что мы~{\itshape никогда не~сможем быть лучше}, чем политик (\hyperlink{par:1.2.1}{1.2.1}). Мы~противопоставляем его некомпетентность компетенции того, кто хорошо осведомлен, строгости ученого, ясновидению провидца, озарению гения, незаинтересованности профессионала, мастерству ремесленника, вкусу артиста, прочному здравому смыслу простого человека с~улицы, чутью индейца, проворству ковбоя, который стреляет быстрее собственной тени, виду и~самообладанию самодовольного интеллектуала. И, тем не~менее, никто не~может быть лучше, чем политик. Тем другим просто есть, где спрятаться, когда они делают свои ошибки. Они могут возвратиться и~попробовать снова. Только политик ограничен одним выстрелом и~обязан стрелять на~публике. Я~сомневаюсь, что кто-либо способен на~то, чтобы думать более точно или видеть хоть сколько-нибудь дальше, чем самый близорукий конгрессмен (\hyperlink{par:2.1.0}{2.1.0}, \hyperlink{par:4.2.0}{4.2.0}).
	\end{itemize}

\paragraph{3.6.3.1}\hypertarget{par:3.6.3.1}{} То, что мы~презираем как политическую <<посредственность>> есть набор компромиссов, к~которым мы~от нашего имени принуждаем прийти политиков.
	\begin{itemize}
	\item 
	Если мы~презираем политиков, мы~должны презирать самих себя. Пеги ошибался. Он~должен был сказать: <<Все начинается политикой, но, увы, вырождается в~мистику>>.
	\end{itemize}

\paragraph{3.6.4}\hypertarget{par:3.6.4}{} Некто, задыхаясь, говорит другим, которые понимают только то, что хотят услышать. Это история о~тех, кто обнаруживает себя посредством загадок и~симптомов. Время от~времени те~о ком говорят, вмешиваются в~разговор, взбешенные из-за того, что их~предали. Иногда те, кто ведут разговор, останавливаются, будучи злыми из-за того, что они не~понимают или их~не поняли. Колеблясь, участники разговора ощупью продвигаются от~полумеры к~компромиссу. Они собирают силы, которые проверяют методом проб и~ошибок и~объединяют их~во временные союзы. Когда результат и~по~нраву, они связывают свою судьбу с~более прочными материалами. Мало по~мало силы
растут, от~объединений к~соглашению, от~одного недопонимания к~другому, до~того момента, когда другие более многочисленные и~умелые не~сокрушат их.
	\begin{itemize}
	\item 
	Макиавелли и~Спиноза, которых обвиняют в~политическом <<цинизме>>, были самыми великодушными из~людей. У~тех, кто убежден, что может достичь чего-то лучшего, чем плохо переведенный компромисс между слабо связанными силами, всегда получается еще {\itshape хуже}.
	\end{itemize}

\paragraph{3.6.5}\hypertarget{par:3.6.5}{} Хотя это может прозвучать странно, наша связь с~большинством сил, от~лица которых мы~говорим, является, вероятно, не~более тесной, чем связь профсоюзного деятеля с~рабочими, которых он~представляет, или управляющего директора с~его акционерами. Я~говорю здесь о~наших грезах, в~той же~мере в~какой он~наших крысах,
желудках или машинах. 
	\begin{itemize}
	\item 
 В~конечном счете политика является приемлемой моделью до~тех пор, пока она распространяется на~политику вещей-в-себе (\hyperlink{par:4.5.0}{4.5.0}).
	\end{itemize}

\paragraph{3.6.6}\hypertarget{par:3.6.6}{} Миры, вероятно, больше походят на~Рим, нежели на~компьютер. Или скорее так, превосходно задуманный компьютер следует коллажом из~переставленных и~заново использованных руин, великолепным Римским беспорядком\footnote{{\itshape Tracy Kidder}. The Soul of~a New Machine (London: Allen Lane, 1981).}. Каждая энтелехия подобна двору Пармы.
	\begin{itemize}
	\item 
	Бальзак сказал о~<<Пармской обители>> Стендаля, что она была <<Государем>> девятнадцатого века. Ни~секреты сердца, ни~секреты двора не~являются грандиозными~--- ни~грандиозными, ни~жалкими, но~нередуцируемыми, смещаемыми и~предаваемыми.	
	\end{itemize}	